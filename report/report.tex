\documentclass[11pt]{report}
%\documentclass[12pt]{article}
\usepackage[top=2.5cm, bottom=2.5cm, left=2.5cm, right=2.5cm]{geometry}
\pagestyle{plain}
\linespread{1.7}


%\usepackage[margin=1.5cm]{geometry}
\usepackage{graphicx}
\usepackage[font=small,font = sl, labelfont=bf]{caption}
%\usepackage{epstopdf}
\usepackage{float}
%To create colored tables
%\usepackage[table,x11names]{xcolor}
\usepackage[dvipsnames]{xcolor}

\usepackage{wrapfig}
\usepackage{subcaption}
\usepackage{dirtytalk}
\usepackage{mathtools}
\usepackage{hyperref}

\usepackage{booktabs}
\usepackage{makecell}
%To create unnumbered section
\usepackage{blindtext}

% Packager for Spread-LaTeX
\usepackage{booktabs, multirow} % for borders and merged ranges
\usepackage{soul}% for underlines
%\usepackage[table]{xcolor} % for cell colors
\usepackage{changepage,threeparttable} % for wide tables

\usepackage{comment}
%To write mathematical letters such as R, C etc..
\usepackage{amsfonts}

\usepackage{amsmath}
%To create a list of item in the same row
\usepackage[inline]{enumitem}

%To create a "note" space
\newlist{notes}{enumerate}{1}
\setlist[notes]{label=Note: ,leftmargin=*}

\hypersetup{
    colorlinks,
    citecolor=black,
    filecolor=black,
    linkcolor=black,
    urlcolor=black
}

%\newcommand{\virgolette}[1]{``#1''}
%\title{Titolo Tesi}

\author{Federico}

\graphicspath{{/home/federico/magistrale/Imagess/}}


\urlstyle{same}
%\hypersetup{
%colorlinks=true,
%linkcolor=black,
%citecolor=blue,
%}



\begin{document}

\begin{titlepage}
\begin{figure}[t]
\centering
\includegraphics[scale=0.55]{cherubino_pant541}
\end{figure}

\begin{center}
	\textbf{University of Pisa \\ Department of Physics E. FERMI\\ Master degree in physics\\}
	\vspace{20mm}
    {\LARGE{\bf Functional connectivity measures using machine learning techniques}}
\end{center}

\vspace{36mm}
\begin{minipage}[t]{0.47\textwidth}
	{\large{\bf Supervisors:\\ Prof. Alessandra Retico\\ Prof. Piernicola Oliva}}
\end{minipage}\hfill\begin{minipage}[t]{0.47\textwidth}\raggedleft
	{\large{\bf Candidate: \\ Federico Campo}}
\end{minipage}

\vspace{60mm}
\hrulefill
\\\centering{\large{\bf Academic year: 2021/2022}}

\end{titlepage}


\tableofcontents

\chapter*{Abstract}
\addcontentsline{toc}{chapter}{Abstract}

Autism spectrum disorder (ASD) is a neurodevelopmental disorder manifesting from the early ages and involving behavioral and cognitive impairments. So far, no univocal biomarker have been detected to identify this disorder, and the most accurate diagnostic tool remains a behavioural and attitudinal test.
In this work we focus on brain functional connectivity data to study the difference between patients with ASD and healthy controls. To tackle this challenge, we  employed machine learning and deep learning models.
Our database are Magnetic Resonance scans provided by the ABIDE dataset consisting on structrural and functional scans of approximately 2200 subjects collected from different medical centers.
One half of them belong to normal individuals and the other half to people with diagnosed ASD.

Starting from these data, the workflow of this thesis can be summarised in the following steps:

\begin{itemize}
\item MRI and fMRI data preprocessing and normalization to a common template, and extraction of temporal series of different brain areas for each patient from the fMRI scans.
\item Creation of a connectivity map for each patient. This connectivity matrix is created computing a correlation coefficient for every pair of timeseries of a single patient. Correlation coefficients were computed using two different approaches: a linear correlation of the two timeseries known as Pearson correlation, and a correlation based on time-frequency analysis of a timeseries pairs by means of wavelet transform.
\item Implementation of a harmonization procedure to correct for potential biases and distinctive traits of data, linked to acquisition scan procedures employed in a medical centre which can differ between different centres.
\item Study and classification of connectivity matrices with deep learning methods. The two different connectivity matrices described above are compared to assess which one is able to yield the best separability between Controls and ASD.
Classification of raw data and harmonized data implemented with different procedures are compared, and in addition a harmonization implemented inside a deep learning model through an adversarial learning technique was tested.
\item Deep learning model is compared to a Random Forest classifier, a standard machine learning model, to determine if a deep model brings a benefit to this kind of analysis.
\item After the definition of the appropriate harmonization procedure, we implemented an interpretation method called SHAP to explain which features contributed the most to the final outcome of machine learning classification, and from them we determined the most relevant brain areas that allow a discerning between controls and ASD
\end{itemize}

\chapter*{Organization of contents}
\addcontentsline{toc}{chapter}{Organization of contents}

\textbf{Background:} In this chapter we explain relevant theory arguments that lay the groundwork to all the thesis work.
In chapter \ref{chap:introduction} we outline the main problems associated with the Autism Spectrum Disorder, from the biological development to the social isseues it generates. We briefly review some studies carried on so far to study this challenging issue. We also provide a brief overview of the main techniques employed to study this issue in this work: from data acquisition to data analysis.

Chapter \ref{chap:mri_introduction} provides an overview of the basic physical principles of MRI imaging and the main acquisition sequences. It also includes a description of the physical and biological principles underlying fMRI diagnostic tool which is the means through which data we are going to employ are collected.

Chapter \ref{chap:machine_learning} provides a brief explaination of machine learning and deep learning models diving into the mathematical formulation of the main concepts. We aim to give the idea of the structure of a machine learning models, the learning procedure, and the important metrics for a proper evalutation of model performances.
Ultimately, we describe a common way to reduce data dimensionality keeping as much information as possible and reducing the number of uninformative or reduntant features from data.

Chapter \ref{chap:shapley_values} describes the theory behind an important strategy to inspect a deep model  and understand the processes that brought it to a certain output given some input data.

\textbf{MATERIALS \& METHODS:} This section describes the instruments we used to run the analysis: dataset, preprocessing steps, and data preparation employed to obtain correlation matrix.

Chapter \ref{chap:dataset} describes and analyse ABIDE dataset: a publicly available collection of more than 2000 scans of healthy and ASD patients, collected from different medical centers.

Chapter \ref{chap:image_preprocessing} provides an overview of the most important preprocessing steps to normalize and align all these scan to a common template to run meaningful statistical analysis and to exctract brain timeseries form different brain areas aligned to a common coordinate system. We describe C-PAC: the software employed to this image preprocessing and how we used it to conduct our processing.

Chapter \ref{chap:connectivity_coefficients} describes the two methods we used to calculate connectivity coefficients and create a functional connectivity matrix for each patient. We describe the computation of correlation using Pearson correlation, and illustrate the formalism of wavelet transform for time-frequency analysis and the process to extract a correlation coefficient from time-frequency data of two timeseries.

Chapter \ref{chap:harmonization_theory} describes the regression model upon which harmonization procedure is based. This is an analytical method to remove inter-site variability from our data and avoid biases towards the acquisition source.

Chapter \ref{chap:domain_adversarial_theory} describes the structure and the general idea beyond a domain adversarial neural network: a deep neural network that aims
to make Control/ASD classification following a procedure that makes data site-independent to classify following a procedure that makes the classification Control/ASD output sites-independent.

Chapter \ref{chap:shap} describes the main aspects of an important algorithm for machine learning model explaination called SHAP. We discuss the general idea behind its implementation and important quantities to determine the contribution of a feature to the output of a machine learning model.

\textbf{Implementation \& RESULTS:} In this last section, the results obtained applying methods to our materials are presented. We illustrate the results of harmonization applied to our data and classification result of Pearson-based correlation coefficients and wavelet-based coefficients and in the end we study the most relevant feature from our data, that played an important role in discrimination of Control/ASD.
Chapter \ref{chap:harmonization_results} Presents the results of the analytical harmonization to our data to have a visual understanding of how this process affects a feature distribution.
Chapter \ref{chap:results} shows all the classification results obtained with a deep neural network on harmonized and non-harmonized (raw) data and the results of the domain adversarial neural network and compares them with results obtained using a simpler Random Forest classifier. Results of data dimensionality reduction are presented as well to assess if this is a meaningful procedure to reduce dimensionality on this dataset.
At the end, chapter \ref{chap:shap_results} shows the implementation of a feature importance assessment procedure and the results obtained with machine learning in the previous chater. We show what feature contributed the most to the outcome of a prediction, and then to the discrimination of healthy and ASD patients, and subsequently we study what brain areas are mainly involved in this discernment.



\addcontentsline{toc}{chapter}{BACKGROUND}
%\chapter*{BACKGROUND}


\chapter{Introduction}\label{chap:introduction}
\section{Autism spectrum disorder} \label{chap:autism}
Autism spectrum disorder (ASD) is a neurological disorder that is becoming a growing social issue, drawing more and more attention expecially during the last decades, for its social impact on families and society and the raise in number of diagnosed cases.
ASD is classified as a neurodevelopmental behavioural disorder \cite{guze-1995} \cite{who-1993} and refers to a broad range of conditions manifesting as deficits in social communication and interaction such as reduced sociability or empathy, repetitive behaviours, resistance to changes, and sometimes even speech difficulties.\cite{rapin-2008}
%\cite{hyman-2020}
To the present days, ASD is diagnosed through a comportamental assessment test since there is not a definite biological test.
Early signs start appearing by the age of 2 - 3, ages at which children usually start interacting with parents and with other children.
Parents are asked to answer a set of questions about every-day behaviour of their children like the checklist for Autism in Toddler \cite{robins-2009} consisting on a list of question such as \textquotedblleft When you point at something, does your child watch in that direction? \textquotedblright or \textquotedblleft does your children show interest towards other children? \textquotedblright.
However, are not uncommon cases where ASD is discovered only after the adolescence.
Currently ADI-R (Autism Diagnostic Interview - Revised) and ADOS (Autism Diagnostic Observation Schedule) are considered the ‘gold standard’ tools for diagnosis of ASD \cite{ozonoff-2015} \cite{lecouteur-2008}.

Nevertheless, besides these behavioural test, to achieve a more accurate diagnosis, medical and biological informations should be taken into account as well.
For a better treatment of this condition, it would be of great importance to make an early diagnosis, in order to provide in time the help and support people need.
Hopefully, treating children when this disorder is still it in its early stage, could bring to a regression of this condition.

The main biological factors that bring to the developing of autism aren't very clear so far, and is acquiring more consensus between neuroscientists and medical doctors that there is not just a single cause, but a concatenation and coexistence of various triggering factors.
Among them, the most strongly suspected are identified to be genetic and environmental, where environmental is to be intended as exposure to air pollution.
%or certain pesticides (even though is difficult to demonstrate a strong relation)
Regarding these environmental causes, it is suspected that during pregnancy, expecially during the first days of embrional development (day 20-24 of gestation), exposition of the mother to air pollution is linked to an increased risk for her child to develop this disorder.\cite{ratajczak-2011}

Genetic origin of this disorder is assiciated with a rare gene mutation, such as a deletion, duplication or inversion, and even though most of the mutation that increase the risk of developing autism are not been traced.
It has been assessed though, that it has an high inheritance trait. From studies on twins it has been assesses that between omozygotes twins there is around 90 \% or probability that both of them develop ASD, while for heterozygote this probability falls to around 7\% for men and 1-2\% for women
\cite{freitag-2007}
%tesi di elisa???.

%Ambiental factors such as prenatal air pollution or certain pesticides are also suspected to be associeted with a increase of the risk to develop this disorder, even thouh the etiology is way more complex and it can't be assessed so accurately.
%Currently ASD is diagnosided by syntoms-based tests concerning behaviour assessment.
From a neurological point of view, ASD may affect a brain both in its structure and in its functionality.

Structural studies usually focus on volumentric and morphometric analyses to examine differences in brain anatomy.
It has been studied how autism could alter the symmetry between the two emispheres \cite{postema2019} of the brain. In children it has been observed an increase in total brain volumne as well asan enlargment of the left superior temporal gyrus, but this trend is not well defined for older ages.\cite{riddle-2017}.


Functional neuroimaging researches mainly focus on impaired functionality. Different studies pointed out a reduced information processing due to synaptic disfunction that manifests in a reduced or altered brain functional connectivity.
Functional connectivity measures aim to describe statistical dependencies between brain areas in time, by studying temporal correlation between different brain parts even between those which are spatially separated.
Altered brain activity is found by different but controversal studies because both hyper- functional connectivity and hypo-FC were detected in ASD patients, while some found a the coexistence of an hyper- and hypo-functionality between different areas of the brain.
In particular it appears functionality abnormalities to have an age-dependent trend since over connectivity is usually observed in young children while under connectivity in adolescences and adults.\cite{supekar-2013} \cite{spera-2019}.

\section{fMRI as investigation tool}

In the last decades, different approaches to study ASD have been proposed, to find a relation between different brain areas involved in lower or higher functional connectivity.
Different diagnostic tools are being used to investigate this matter, from different perspectives and at different scales, such as magnetic resonance imaging (MRI), functional-MRI (fMRI), or electroencephalography (EEG).

Magnetic resonance imaging is employed to obtain images of brain both for structural studies and for functional through the  technique of functional MRI (fMRI).
An other tool employed to study temporal signal is the EEG which differs from fMRI because of the type of signal it aims to detect and because of the different resolution, since with EEG is possible to study signal in scale time comparable with the neuronal activation time (timescale $\sim \mu s$), while with fMRI we look for a signal due to the hemodynamic response following a neuronal activation (timescale $\sim s$).


fMRI is a diagnostic tool that investigates physiological changes linked to blood flow variations across the whole brain structure.
Measures of blood properties are strictly linked with measures of the neuronal activity: an increase of neuronal activity leads to a boost in blood flow and an increase of vessel's size within a specific brain area, because of the bigger demand of oxigen and other nutrients to the neurons in that area.

The different neural connections are identified by measuring the BOLD fluctuations, (acronym for blood-oxigen-level dependent) from different areas of the brain, using the signal difference between oxy (diamagnetic) and deoxyhemoglobin (paramagnetic).
%Oxyhemoglobin appears then brighter than deoxyhemoglobin
These spontaneous fluctuations in the BOLD signal during resting state are considered a strong indicator for the assessment of the properties of the brain system.

fMRI can be performed during the resting state of a patient or during a task.
We refer to these procedures as rs-fMRI and task-based fMRI, and while task-based fMRI is a good instrument to investigate the local activation of a single brain area during a task, resting state fMRI is drawing more attention for the investigation of ASD.
It was proven to be suitable for examining functional connectivity among all the brain regions, and highlight the difference between a normal brain and one affected by this disorder.

Resting state fMRI is an acquisition carried out while the patient is in a relaxed state, and is not performing any active task, but he is simply asked to stay still with eyes closed or open while fixating a cross.
%either resting or, more in general in a task-negative state.
Following this setup, the brain is at rest and its activity is not perturbed by active tasks, such as moving an arm, or passive actions, like being exposed to different visual stimuli, which are common activities for task-based functional connectivity, a different acquisition setup to investigate BOLD fluctuations between subjects in a task-stimulated state and control states.
Resting state setup revealed to be a powerful tool to investigate the intrinsec generated brain activity and study the altered functional connectivity networks in subjects with mental disorder

\section{Machine learning to deal with non-linear problems}\label{sec:intro_ml}
During the study of characteristic traits between control and ASD patients, different attributes are involved and affect data in a strong and evident way, the most important of which are age, sex and FIQ (full intellective quotient).
Age, for example, affects both structural and functional data, with an observed overgrown of the brain volume and hyperconnectivity, in toddlers, that are both traits that tend to decrease going in on age.
Moreover, on control subjects, even though it is not well characterised, seem to show a decreasing of brain structure and functional connectivity in older ages \cite{zhangC-2016}.
Sex is the other feature that gives characteristic traits to data.
Studies show that some brain areas exhibit an increased functional connectivity in female.
And, from a combined analysis of both age and sex, it appears that in men some brain structures show more pronounced aging effect than women \cite{coffey-1998}.

In addition, if we limit our focus just to functional data, the eye status at scan plays an important role.
Functional connectivity with open steady eyes results to be different from closed eyes, with strong differences and higher connectivity in different brain areas between the two coohrts of subjects.\cite{costumero-2020}.

Furthermore, an other aspect to be take into account regarding closed eye patients, is that during the scan acquisition, they may fall asleep, and this would heavily modify the functional brain activity, resulting in a modified functional connectivity pattern.

It appears clear that the distinction of ASD patients among normal ones is not a straightforwand task and a simple univariate analysis would not be sofficient, because of the presence of different confounding factors contributing to the characterization of data.

To take on this challenge, one of the most promising tools that allow us to deal with complex, non-linear problems is the use of machine learning algorithm to study data extracted from fMRI such as pairwise correlations between different regions of the brain, or image of brain themselves from MRI scans to studies with convolutional neural networks (CNN).
%One of the most promising and powerful tool to tacke non-linear problems is machine learning.
Machine learning belongs to artificial inteligence tools and make use of alghorithms that learn from data, and modify their parameters with the aim of recognising distinctive properties from data and make predictions or classification, on new unseen data, trying to gradually reduce the error and make amore accurate predictions.

A relevant aspect that makes machine learning and artificial neural networks so popular is the ability to deal with non-linear problems, by introducing several non-linear functions during training.
This allows to obtain non-linear outputs from each input data, which would hopefully help in the distinction between characteristic traits of healthy people and of ASD patients.

As a drawback, a restraint of machine learning is that, to perform well, an algorithm needs to be trained on datasets of large dimension, because the bigger the dataset, the more the algorithm is able to generalize information and the better are its performances on new data.
This is true for all the kind of machine learning algorithms and expecially for Deep Neural Networks which are the principal kind of algorithms we are going to employ in this work.

\section{The need for a harmonization procedure}

To achieve the goal of obtaining a large dataset, particularly in medical field where data are not easily available from a single centre or are not in a sufficient amount to perform a large-scale analysis, data from different acquisition centers need to be put together.
In the last years, this issue is becoming increasingly popular and several multicenter medical datasets such as the Human Connectome Project, the Alzheimer’s Disease Neuroimaging Initiative (ADNI) or Autism Brain Imaging Data Exchange (ABIDE) were created to obtain large collections of data to make significant statistical analysis.

Unfortunately, this procedure brings with it a downside regarding the unavoidable bias towards the site that data belong to.
This bias is a result of hardware and scan procedure differences between centres and it is not feasible to ask to all the medical centers across the world to uniform to a single common acquisition protocol.

To work out this problem and try to uniform inter-scan variability, we need a harmonization procedure to remove site-dependent information, leaving all the rest of important information unchanged.

In this work, two different approaches are proposed: an analytical harmonization and a deep learning approach.

Analytical harmonization is a procedure that modifies data with shifts and rescaling to remove only inter-site related effects in multi-site data, while trying to preserve all other information as biological-related effects.

The second approach, the deep learning one, tries to make the classification of control/ASD data extracting from input data both information related to the class (Control/ASD) and to site.
Then it uses the latter to remove the bias linked to sites before making a prediction.


\chapter{Principles of Magnetic Resonance Imaging}\label{chap:mri_introduction}
 In this chapter we report some basic principles that underlie the Nuclear Magnetic Resonance Imaging.
 Since the matter can become very complex, we only give some glimpses of what are the main principles behind this technique and the main acquisition sequences employed in diagnostics.

\section{Physical principles}\label{sec:mri_introduction}
Nuclear magnetic resonance imaging, is a non-invasive imaging technique widely employed in radiology to obtain anatomical images or to study physiological processes, taking advantage of its flexible sensitivity to different tissues.

\begin{wrapfigure}[25]{l}{0.4\textwidth}
\centering
\includegraphics[width=1\linewidth]{mri/precessione_}
\caption{Representation of a angular moment vector $\vec{\mu}$ under a static magnetic field $\vec{B_0}$ and the resulting moment $\vec{\tau}$  which causes the precession}
%\label{fig:precession}
\end{wrapfigure}


The physical principle of this technique is the interaction of a nuclear spin of any tissue molecule, with an external magnetic field $\textbf B_0$.
Molecules that contribute the most to the signal are water, and specifically, hydrogen nuclei are the dominant source of signal in MRI.

When a nucleus is subjected to an external magnetic field, the interaction would make the nuclei align with the magnetic field, but, since protons and nuclei have an intrinsic spin, and hence an angular moment $\overrightarrow{\mu}$, they start precessing.

Precession occurs because of the torque created by the interaction of a static magnetic field with the angular moment of a spinning nucleus.
The relation is given by

\begin{equation}
\overrightarrow{\tau} = \overrightarrow{\mu} \times \overrightarrow{B_0}
\end{equation}

Precession occurs around the symmetry axis given by the direction of the magnetic field $\overrightarrow{B_0}$ as schematically shown in figure \ref{fig:precession}.

When a nucleus starts this rotatory movement, the characteristic frequency $\omega_0$ is uniquely determined by the properties of the nucleus itself and by the strength of the magnetic field; the relation is given by the Larmor equation and the frequency $\omega_0$ is called \textbf{Larmor frequency}

%\begin{equation}\label{eq:larmorfrequency}
\[
\omega_0 = \gamma B_0
\]
%\end{equation}
where  $\gamma$ is a constant called \emph{gyromagnetic factor} and is it characteristic of each nucleus.
The gyromagnetic factor of a proton in water has a value of $\gamma = 2.7\cdot 10^8 \frac{rad\cdot s}{T}$.
For a typical scanner with a magnetic field of 2-3 T, the Larmor frequency assumes values of $\sim MHz$ just below radio waves.

When protons in a body are exposed to a magnetic field, they can assume two states in relation to the direction of the external field: parallel and anti-parallel (as an example in figure \ref{fig:precession} a parallel configuration is represented).

\textcolor{ForestGreen}{
Proton with different orientation with respect to the magnetic field have different energies.
The potential energy of a particle with magnetic moment $\vec{\mu}$ immersed in a magnetic field $\vec{B_0}$ is given by
\begin{equation}
U = -  \vec{\mu} \cdot \vec{B_0}
\end{equation}
If we suppose $\vec{B_0}$ along the z axis, it will have only component along z, so the magnetic moment will just concern $\mu_z$.
Recalling the relation between the angular moment and the magnetic moment $\vec{\mu} = \gamma \vec{J}$, and the quantization of the angular moment $J_z = m_s \hbar$.
For a proton $m_s$ can be either $-\frac{1}{2} \text{ or } +\frac{1}{2}$ so its energy when subjected to a magnetic field can be written as
\begin{equation}
U = -\vec{\mu} \cdot \vec{B_0} = - \mu_z B_z = - \gamma m_s \hbar B_z = - m_s \hbar \omega_0 = \pm \frac{1}{2} \hbar \omega_0
\end{equation}
For a thermodynamic system, with a given temperature T, the probability to find particles with energy $\epsilon$ is given by the Boltzmann distribution
\begin{equation}
P(\epsilon) = \frac{1}{Z} e^{-\frac{\epsilon}{kT}}
\end{equation}
where k is the Boltzmann's constant.
Using this relation we can compute the difference between the number of proton in the two states parallel and anti-parallel
\begin{equation}
\Delta N = N^+ - N^- = \frac{N}{Z}(e^{-\frac{\hbar \omega_0}{kT}} - e^{\frac{\hbar \omega_0}{kT}}) \approx \frac{N}{2}\frac{\hbar \omega_0}{kT}
\end{equation}
}

%If we compute the difference between the number of proton in these two states we can determine the net magnetization of a body, expressed by a vector pointing parallel or anti-parallel to the direction of $\overrightarrow{B_0}$.

This net magnetization can be represented by a single vector pointing parallel or anti-parallel to the direction of the external field and if we want to detect a signal related to this magnetization, we have to perturb this vector from its equilibrium state.
The perturbation would cause the precession of this magnetization vector around the direction of the magnetic field, with its own larmor frequency.
This precession will produce a changing flux which can be detected by an external coil.
% only if it is perturbated from its equilibrium state.
To kick this vector away from its equilibrium state, a second external magnetic field is employed under the form of a radiofrequency impulse, with a frequence resonating with the Larmor frequence of the precessing spins.

A common way to apply this radiofrequency impulse is with the purpose of flipping the magnetization vector and hence the spins in a direction usually perpendicular to the original along $\overrightarrow{B_0}$ one, which we can suppose to be along $\hat z$ axis.

We create this way a magnetization component onto the x-y plane, aligned with the direction of the radiofrequency pulse.
This magnetization vector on thte transverse (x-y) plane is called \emph{transverse magnetization}.
After this radio frequency pulse, protons will start rotating in the x-y plane, but, since this impulse is not persistent, after it is switched off, they gradually lose their initial energy and tend to realign along the z-axis.
This process of realigning along the z-axis is called longitudinal relaxation (longitudinal in relation to the original $B_0 \hat z$ direction).
%In other words, we create a transverse magnetization by applying a rf-pulse, so that when we have a transverse magnetization, we lost the longitudinal magnetization because the vector is flipped and lie on the x-y plane.
%Gradually, though, the transverse vector lose its magnitude in the x-y plane because of the gradually loss of energy of protons and the longitudinal magnetization reacquire its previous magnitude (longitudinal in relation to the original $B_0 \hat z$ direction).

The evolution of the longitudinal and transverse magnetization are modeled with the introduction of two time constant $T_1 \text{ and } T_2$ respectively.
Longitudinal relaxation occurs because the system goes from an higher energy state (when it is flipped on the x-y plane) to a state of thermodynamic equilibrium with its surroadings, regaining its previous magnitude.

This process follows an exponential decay in time shown in equation \ref{eq:spinlattice}, with a time constant $T_1$, which for the physical reason underlying it, is also called spin-lattice relaxation time.

\begin{equation}\label{eq:spinlattice}
M_z(t) = M_0 (1-e^{-t/T_1})
\end{equation}


The transverse magnetization vector follows a slightly different evolution. Before introducing the T2 parameter associated with its evolution, we establish a notation to indicate this magnetization vector: since it lies on the x-y plane it has two component so we can write $\overrightarrow{M_\perp} = M_x \hat x + M_y \hat y$.
When studying the evolution of transferse magnetization an other effects has to be taken into account, namely the spin-spin interaction between protons.
If we consider a physical system where proton are immersed in a magnetic field, each proton interacts with this external field plus all the small fields created by the surroading protons and their associated spins. This leads to the presence of a local field for each proton, slightly different from the external magnetic field $B_0$.
These different local fields affect the evolution of the transverse magnetization because they cause a relative dephasing of protons between each other.
According to this, if immediately after the rf-impulse all the spins can be imagined as aligned to the x axis, in time they start to fan out. This effect speeds up the loss of the total transverse magnetization on the x-y plane.

If we study this evolution from the rotating reference frame, (rotating with the same Larmor frequency of protons) the module of the transverse magnetization follows an exponential decay in the form
\begin{equation}
M_\perp(t) = M_\perp(0) e^{-t/T_2}
\end{equation}.

Since T2 constant comprises the spin-spin interaction process plus the spin-lattice interaction which cause the realignment of spins along the $\hat z$ axis, it is always smaller than T1; and to give an idea of the order of magnitude for protons in huma tissue, if T1 ranges from 10 to 100 milliseconds, T2 is usually of the order of 10 milliseconds.


In a real pyisical system, however, there's and additional dephasing source, coming from external field inhomogeneities.
This effect is often taken into account by the introduction of a different decay time $T_2'$ which along with $T_2$ bring to a overall time costant given by
\begin{equation}
\frac{1}{T_2*} = \frac{1}{T_2}+\frac{1}{T_2'}
\end{equation}


\section{Image acquisition and k-space}

%fem
The physical process underlying signal acquisition is Faraday induction, according to which an electrical potential called electromotive force is generated by the variation of a magnetic flux over time $emf = -\frac{d\Phi}{dt}$ being $\Phi$ the varying flux through the receiving coil.

A simple setup in MRI to generate a varying flux over time is the application of a single rf pulse with an angle of $\pi/2$. The variation of flux occurs while tipped spins are rotating in the x-y plane and the magnetization vector is relaxing towards the longitudinal axis.
This induced emf signal can be detected by properly oriented and tuned coils. This simple experiment is referred as Free induction Decay (FID) and is usually performed in any MRI scan to tune rs coils and optimise system response.

Usually the varying flux we are interested in is the one along the transfersal plane. The reasons behind this are mainly two:
\begin{enumerate*}
\item the weak signal along the longitudinal axes would be saturated by the strong magnetic field of the main $B_0$ coil.
\item The signal coming from the transverse magnetization $M_\perp$ is representative of all the information we need for the analysis of a material: T1, T2 and the proton density $\rho$.
\end{enumerate*}

\hfill

To properly create an image based on these information we need to spatially encode each signal received by the changing magnetic vector of the object.
In order to relate a signal with a spatial position, in addition to the initial static magnetic field $B_0$ we have to place a second field which causes a controlled local modification of the magnetic field $B_0$. This additional field $B'$ needs to be non-uniform and have a lower magnitude of $B_0$ for each point.

Its distribution follows a spatial gradient so that the total magnetic field along $\hat z$-axis is given by the sum of this two contributes, and the signal contains space-varying frequency components according to equation \ref{eq:larmorfrequency} which can be rewritten as
\[
\omega(z) = \gamma B(z)
\]
being z the spatial coordinate and where $B(z)$ is the total magnetic field now given by the relation
\begin{equation}
B(z, t) = B_0 + z\cdot G(t) \qquad G(t) =\partial B'/\partial z
\end{equation}

This gradual changes in magnetic field causes different parts of the body along the z axis to have different larmor frequencies.
This gradient along the longitudinal direction of the static field $B_0$ is usually referred as Slice Selection gradient $G_{ss}$.
Choosing the z axis as the direction of the slice selection, we acquire images on the transversal x-y plane.


When acquiring an image, data are collected under the form of a matrix called k-space.
K-space is a coordinate system used to store spatial frequencies information. From these informations we can retrieve the usual MRI image (containing spatial, anatomical informations) by applying the inverse fourier 2D transform.

In a k-space matrix, low spatial frequencies, corresponding to large object across the whole real image, are encoded at the center of the matrix and high spatial frequencies corresponding to small objects and finer detailes, are encoded in the peripheries.

The construction of k-space is done step by step in relation to the gradient applied time by time, producing a trajectory on the k-space.

To acquire spatial frequency information from an image we need to introduce two kind of new gradients (one for each direction): a readout (or frequency encoding) gradient and a phase encoding gradient.

The frequency encoding gradient acts on one direction in the transversal plane; let's identify this direction as the x axis for a clearer visualization.
Just like the slice selection gradient it consists of a linear changing magnetic field to modify the Larmor frequency of the spins along the x axis. According to the acquisition technique we want to perform, it can be applied forward and after a while, reversed. This allows a refocusing of all the dephased spins due to spin-spin interaction; we will see more in detail this concept when we will discuss acquisition sequences, for now, we just need to know that the refocusing of all the dephased spins occurs after a time interval called echo time (TE).

The phase encoding gradient acts the exact same way of the previous one, but it acts on the y axis. It is switched on and acts by modifying the larmor frequency but it is just a temporary change. When it is switched off, all the spin along this direction contrinue precessing all at the same frequency, but with a relative phase between them.

The combined action of these two gradients allow us to acquire different lines of the k-space.
%Each spatial line of k-space is acquired after a time interval called \textbf{Repetition time (TR)}.


For further and more detailed information we refer to the book \cite{brown-2014}

\section{Acquisition sequences}
Depending on the type of analysis we want to carry out, we would like to focus on some magnetic properties rather than others, for eaxmple, for a an anatomical, structural image, we may be interested in a good distinction of different tissue, while in a functional scan we are interested in detecting chemical fluctiations.

When talking acout acquisition sequences two main parameters drive the main differences between different techniques.
For the moment we limit to define them but a contestualized use of them can help to clarify their meaning.
This can be found in the following sections describing different acquisition sequences.

\begin{itemize}
\item \textbf{Repetition time (TR)} is the time interval between one rf-pulse and the next.
\item \textbf{Echo time (TE)} is the time interval between a rf pulse and the echo peak
\end{itemize}

Changes in TR, TE and rf pulses characteristics (such as angle or numbers), allow to perform different acquisition sequences and emphasise one of the three fundamental parameters $T_1, T_2$ or the proton density $\rho$ of different tissues.

As a general rule
\begin{itemize}
\item T-1 weighted images are acquired with small values of TR. This avoid that all nuclei are back to their longitudinal position, allowing a better tissue contrast and TE has to be very short to avoid contribution due to T2-related effects.
\item T2-weighted images are acquired with high values of TR ($TR \gg T1$) and values of TE comparable with T2 time constant.
\item To enhance the spin density contrast TR has to be chosen as long as possible while TE needs to be short.
\end{itemize}


\subsection{Spin-echo sequence}

%Sometimes, though, this additional $T_2'$ is so small compared to $T_2$, that dominates over it, resulting in a rapid loss of dephasing information.
%This lost phase information can be recovered employing a specific rf pulse sequence called \textbf{spin-echo acquisition sequence}.
Spin echo sequence is one of the most used acquisition sequence. It lets us retrieve lost information due to spin dephasing caused by field inhomogeneities.

The spin echo method employes two rf pulses: the first with an angle of $\pi/2$ with respect to the $\overrightarrow{B_0}$ direction and the second of $\pi$ in relation to the direction of the first impulse.
All the sequence can be summarised by the following steps:
\begin{enumerate}
% Add a figure like on page 123
\item The first radio-frequency pulse is applied with an angle of $\pi/2$ in relation to the direction of the external magnetic field $\overrightarrow{B_0}$. This process as mentioned in section \ref{sec:mri_introduction} causes the spin to flip onto the transversal plane x-y.
Just as an example we imagine they are flipped into the x direction. They gradually start dephasing both because of spin-spin interactions and small external field inhomogeneities, which are different from point to point.
Because of this, they start fanning out, some rotate faster and some are delayed in relation to the average magnetization vector.

\item At the instant t = $\tau $ the second rf impulse is sent. It is created with an angle of $\pi$ in relation to the first impulse, so if the first was along the $\hat x$ direction, this point towards $-\hat x$. This has the effect of turning all spins with a phase $\phi$ to a phase $\pi - \phi$.
\item These dephasing spins continue accumulating a phase but since they were flipped of 180 degrees, the same process according to which they were previously fanning out, push them to converge, untill they all meet again when the relative phase return to zero. This results in a recreated transverse magnetization vector in the opposite direction of the first one created after the $\pi/2$ impulse.
\end{enumerate}

The time interval covering the application of the first impulse $\pi/2$ to the instant at which phases turn back to zero is called \textbf{echo time} $T_E$.
Between the application of the $\pi$ pulse and the echo time, the acquisition sampling of k-space starts.


This method is an effective strategy to remove the nuisance effect due to field inhomogeneities associated to $T_2'$, but do not reduce the effect of T2 due to local fields and spin-spin interactions. The reason behind this is that static field inhomogeneities remain the same even after the $\pi$ impulse, so they act the same way even after the impulse, while local fields which cause spin-spin interactions change and the rate at which they accumulate phase changes with them. In general no refocus strategy is possible to correct for this effect.
Fortunately though, this is not a critical issue for liquids since the time interval over which data are collected is usually smaller than the T2 time constant.



\subsection{Gradient Echo sequences}
The sequence employed to acquire functional imaging is the Echo Planar Imging (EPI) which is a particular sequence from the family of gradient echo sequences.
%A different acquisition sequence is the gradient echo.
Gradient echo sequences differ from spin echo because they allow a faster acquisition.
They use a single excitation impulse with a flip angle less than 90 deg, this allows a minor time interval to make spins back to their longituginal component as they are not completely flipped on the transverse plane.

The transverse component just created decay and dephase according to T2*.
%If along the z-axis a gradient magnetic field is added to the primary magnetic field $B_0$, spins dephase faster because of this additional field perturbation.
At this point in an spin echo sequence, the $\pi$ impulse would be applied; in its place, in GE sequences, the gradient along the x axis is reversed.
This gradient inversion causes the spins to rephase but what is relevant is that this gradient inversion doe not act neither on dephasing due to field inhomogeneities nor on dephasing due to chemical shifts, but just on dephasing due to the gradient field.
This is the most important aspect that makes GE different from SE: the relaxation is dictated by $T_2*$ and not just by $T_2$
In an gradient echo sequence, the echo peak lies on T2* decay curve while in a spin echo, it lies on T2 decay curve;
Thanks to this property, this image acquisition is susceptible to any chemical variation such as that due to hemoglobin shift.



\subsection{Echo Planar Imaging (EPI)}\label{sec:epi}
EPI refers to a technique to acquire all the 2D k-space after a single rf impulse. This is achieved by applying intermittent small phase encoging gradients called blips: triangular gradients that are switched on and off between gradient echoes in a multi echo acquisition.

To discuss this method is can be useful to give a look at picture \ref{fig:epi_gradients} where all the gradients in play are illustrated, and discuss line by line what they represent
\begin{enumerate}
\item A single rf impulse with an angle of 90 degrees is applied to the body under examination
\item Together with the rf impulse, a slice of the body is selected, corresponding to a position along the z axis, by applying the slice selection gradient
\item Now we focus on the single 2D slice to begin the k-space data acquisition starting from the bottom line of the diagram in figure \ref{fig:epi_kspace}: both gradient of phase encoding and frequency encoding are switched on.
\item The frequency encoding (readout) gradient is reversed to create the first echo
\item When the first echo occurs, the first line of k-space (along the kx direction) is acquired. In figure \ref{fig:epi_kspace} this lines corresponds to the horizontal bottom line
\item Now we have to move to higher values of ky: to accomplish so, the phase encoding gradient performs a small blip: it is switched on for a brief time and then switched off. This dephase spins a little further so that we move up along the ky direction in figure \ref{fig:epi_kspace} and we are ready to acquire the other line of k-space
\item the frequency encoding gradient is reversed again and the second horizontal line along kx is acquired.
\item This process is iterated: these last two steps are repeated untill all the k-space is covered and the slice data is fully acquired.
\end{enumerate}

This acquisition technique allows the acquisition of a single slice in a short time (around 50-100 ms). The whole k-space for a single slice is acquired sequentially, following a snack-like pattern and the number of k-space line acquired after each single rf impulse is named differencly according to the scan brand factory: examples are \emph{EPI factor} or \emph{Echo train lenght}.

Since slice data are collected after gradient echo inversions, this imaging technique if sensitive to the $T_2^*$ time constant which makes it suitable for functional imaging where we are interested in chemical shifts within a body. For this reason this kind of imaging is often called $T_2^*$-weighted imaging.
In fact, the period of the Readout gradient is the echo time $T_E$, and is modulated such that is sensible to $T_2^{\ast}$ setting $T_E \approx T_2^{\ast}$.




\begin{figure}
\begin{subfigure}{0.5\textwidth}
\includegraphics[width=\textwidth]{mri/epi_gradients_}
\caption{}
\label{fig:epi_gradients}
\end{subfigure}
\begin{subfigure}{0.5\textwidth}
\includegraphics[width=\textwidth]{mri/epi-kspace}
\caption{}
\label{fig:epi_kspace}
\end{subfigure}
\caption{}
\label{fig:epi}
\end{figure}



\section{Functional MRI (fMRI)}\label{sec:fmri}

Functional Magnetic Resonance Imaging is a non-invasive diagnostic tool which aims to measure the neural activity of different parts of the brain.
The vast majority of scan acquisition is carried out usi EPI sequences (section \ref{sec:epi}) and consist of several 3D volumes acquired each 1.5-2 seconds for the entire duration of a scan which lasts from 5 to more than 8 minutes depending on the scanner model.


The signal it detects comes from the blood oxygenation level fluctuations following a neuronal activation, and for this reason this kind of analysis is referred as BOLD (blood oxygenation level-dependent) imaging.

Blood can in fact be portrated as a colloidal mixture where blood cells constitute around 40-45 \% of volume. \cite{brown-2014}

From a physical point of view the variation in oxygen content in blood affects the local magnetic susceptibility of the blood.
Signal coming from blood cells is mainly due to the presence (in each blood cell) of several molecules of hemoglobin: a protein containing 4 \textquotedblright heme groups \textquotedblleft each one including an iron atom which binds an oxygen molecule and carries it throughout veins and capillaries.
Hemoglobin can be in two states: oxyhemoglobin and deoxyhemoglobin, these two molecules differs in the presence of the bounded oxygen molecule which reflects in differences in their magnetic susceptibility as oxyhemoglobin has diamagnetic properties while deoxyhemoglobin is paramagnetic.
This difference is due to unpaired iron electrons in the deoxyhemoglobin which leads to an unshielded molecule against the external magnetic field.
The presence of deoxy-hemoglobin causes local field inhomogeneities, leading to a reduction of the main signal coming from water molecules.

This signal weakening is due to spins going out of phase between each other more quickly, because of these additional field inhomogeneities, which results in reducing the total magnetization, this means that nuclei would lose their magnetization faster than the typical $T_2^*$ decay constant.
For this reason if the scenner is tuned to the $T_2^*$ relaxation time, it woule be possible to appreciate this chemical shift between oxygenated and deoxygenated areas. This gives the alternative name of BOLD-images as $T_2^{\ast}$-weighted images. \cite{huettel-2009}

This way, since the presence of deoxy-hemoglobin causes a weakening of the signal, when a brain area is activated, it demands a greater amount of oxygen carried by oxyhemoglobin molecule which results in an increasing of signal.

The process that from a neuronal stimulus leads to a measurable blood signal is governed by the hemodynamic response function (HRF) shown in figure \ref{fig:hrf}. This figure represent the blood response immediately after a neuronal stimulus (which occurs on a timescale of $\approx$ 10-100 $\mu$s). The BOLD signal takes $\approx$ 4-6s to reach a peak and a total of 20-30 s to return to its zero baseline.

\begin{figure}[h]
\centering
\includegraphics[width=8cm]{mri/hrf}
\caption{Evolution of BOLD signal of an adult brain during time. At the instant t=0 the neuronal stimulus occurs; after that, it takes around six seconds to reach a maximum of signal, and a total of more than twenty-five seconds to return to its baseline value}
\label{fig:hrf}
\end{figure}


Since we are interested in these slow signal trends, the TR of 1.5-2 seconds is the minimum necessary to have an analysis sensitive to these signals.
According to the Shannon-Nyquist sampling theoreme, the maximum detectable frequency in a signal is equal to 1/2 sampling frequency. For this reason with a TR of 1.5 seconds for example, we are able to detect signals with a period $\geq 3$ seconds.




\chapter{Machine learning}\label{chap:machine_learning}

%Machine learning is a branch of Artificial Intelligence (AI) that aims to learn from data and gradually improve its accuracy.
Machine learning is a branch of Artificial Intelligence (AI) that aims to learn from data and apply this knowledge to operate on other, new, data such as making predictions or classifying them, and by training training process, gradually improve its accuracy in the task it performs.

Algorithms are trained to make classifications or predictions and, as a rule of thumb, the bigger and homogeneous is the dataset to train the algorithm, the better accuracy and generalization (namely perform well on different datasets) can be achieved.

According to the analysis we aim to perform, and the type of dataset we are working on, machine leraning algorithms can be divided into two macro areas: supervised and unsupervised learning.

From a dataset, consisting on a collection of data, each one containing a certain number of features, an \textbf{unsupervised} algorithm goal is trying to learn the principal properties from the dataset's structure and to extract information from data without human labour to annotate and label each data.
Analysis with unsupervised algorithms include clustering or dimensionality reduction.
Clustering is maybe the most simple and intuitive example of an unsupervised learning: it is basically a classification process through which unlabeled data are reorganised and classified into subgoups according to some common properties or distance measures that arise from the analysis.
In this way, data of each group, called cluster, share a certain degree of similarity in feature probability distribution.
%A common algorithm for clustering is k-clu?
%modificatoo
Dimensionality reduction is another common example of unsupervised learning process: it is performed with the task of finding an different representation of the data, with a lower dimensionality, and preserving as much information as possible about data.
This can be accomplished either by compressing data into a lower-dimensional space, or by searching the main source of variance across the data and create a representation in such a way that the dimensions of the new representation are statistically independent.



%They usually aim to learn the probability distribution of the features in our data, and to do so, they search for a new representation of the data, often simpler than the initial one.
%Simpler can be associated with
%\begin{itemize}
%\item lower dimensionality of data namely the reduction of the space where data lie of informally a reduction of the number of features in our data
%\item sparse representation of data transform the data in a representation containing zeros in most entries, to try to aligh input data along this new  representation's axis. Usually resulting in an increasing of data dimensionality to avoid information lost deriving from setting most entries to zero
%\item Independent representaiton focus on the main sources of variance across data, trying to create statistically independent representation ofa initial data.
%\end{itemize}

On the other hand, when we are dealing with supervised learning each input data is associated with a label, specifying the class which data belong to, so the algorithm is to learn the common features for each class trying to predict the label given the corresponding input, or to classify them according to the right label.
Generally speaking, given an input data x and an associated label y, they try estimating p(y|x). These algorithms are referred as supervised because of the human labour needed to label each input data.

A subset of Machine Learning is Deep Learning, the main difference lies on the algorithms structure and the learning techniques, as an instance, traditional machine learning methods consist on algorithms like Random Forest, Support Vector Machines (SVM), K-nearest neighbor (KNN) and several more models, while deep learning models include Artificial neural networks (ANN), Convolutional neural networks (CNN) and more.
As we will see in furter paragraphs, they have a common structure similar to a neural structrure, made by different neurons connected to each other.




\section{Random forest}
Da rivedere un paio di frasi.
Random Forest classifier belongs to the category of \emph{ensemble} classifiers, and is a common machine learning algorithm.
It combines multiple decision tree models to reduce overfitting of data and achieve a better generazability to create a more powerful model.
Therefore, to properly understand a random forest algorithm, we need to start understanding how a Decision Tree classifier works.

\paragraph{Tree classifiers} \hfill

A decision tree aims to learn distictive traits from input data by asking yes/no questions. Focusing on the valus of a single input feature, the classifier splits the dataset based on the value of that feature according to an if/else statement like \textquotedblleft if feature\_i $>$ a \textquotedblright.
Doing so each branch of a decision tree consists of a question which splits the dataset into two smaller sub datasets.
Considering different features, the process is then recursively repeated untill it reaches an end point called leaves, corresponding to the ultimate partition, containing a single data point belonging to a single class.
The goal of this tree is to construct branches so that the partition are informative about the class labels.
In a practical way, given a dataset X made of different data samples, each one containing n features, the algorithm construct a tree following these steps:
Starting from the top node called root, it searches among the n features the one which allows the best split between classes and split the dataset into two subsets, each one constituting a node. This split is performed according to a pre-defined objective function we want to maximize throughout the tree construction.
Usually these kind of functions regard some kind of impurity measures such as the difference between the impurity of a parent node and the sum of the impurity of the child.
The lower the impurity of the child, the bigger the Information Gain.
This splitting process is repeated for each node in order to achieve the best information gain for each split, and
%Then repeats the previous step and split each subset in order to have the best information gain about categories for each split.
the process is iterated and the tree is grown untill a single node is left.
The important features according which a tree is forked at each node, can be visualized and if we plot the tree structure.
This property is one of those which make decision trees so popular: they are easy to interpret and understand since their structure and the information of each node can be plotted to have a visual feedback of what is going on during the growing process.
Other positive sides of decision trees are, their easy to use implementation which requires little data preparation and their rapidity since their complexity goes like $O(n_{features}n_{samples}^2log(n_{samples}))$ \cite{scikit-2011}

%decision trees are among the most popular classifiers for different reasons: theey are implemented in several packages for different languages, they are easy to use since they require little data preparation, and are simple to understand and interpret, since they can even be visualized.


Three common impurity measures or splitting criteria are commoply used in decision trees: Gini impurity, entropy and classification error.
%We are not focusing on these functions, a more detailed explanation can be found in \cite{raschka-2019}.
%Gini Impurity is the frequency with which a randomly chosen element in the dataset is incorrectly classified if it were randomly labeled according to the class distribution in the dataset.
As an example we report the Gini impurity measure. If a node contains a sub-dataset Q with a total number of data n, belonging to different classes k, the gini impurity measure is obtained by the simple relation
\begin{equation}
H(Q)=\sum_k p_{k}(1-p_{k}) = 1- \sum_k p_k^2
\end{equation}
Where $p_k$ is the fraction of data in Q belonging to class k.
%\begin{enumerate}
%\item Starting from the top node called root, it searches among the n features the one which allows the best split between classes
%\item Split the node by asking a question about the feature that would provide the best classes split
%\item Iterare point (1) and (2) until a single data point is left
%\end{enumerate}

An example of a decision tree dealing with flower classification is shown in figure \ref{fig:decisiontree}.
We choose this figure because is easier to understand how a decision trees works with this example where each feature is labeled with an intuitive name and contains a physical quantity easily comparable with everyday life such as petal lenght expressed in centimeters.

\begin{figure}
\centering
\includegraphics[width=0.8\textwidth]{ml/decisiontree_example.png}
\caption{A graphical representation of a Decision Tree for a classification task of the Iris dataset consisting on three different classes of iris: Setosa, Versicolour, and Virginica. Each node of the tree shows the criterion used for its split, the gini values, the total number of samples in that node and the main class that data belong to}
\label{fig:decisiontree}
\end{figure}

Instead of looking at the whole tree, we may be interested in what feature contributed the most to the final output, to this end, a decision tree usually weigths feature with a number from 0 to 1 according to how much a feature contributed to the impurity decrease along the tree.
\cite{raschka-2019} \cite{muller-2017}

Usually though, one of the main drowbacks of decision trees is that the iterating process towards the reaching of a leaf, can bring to the creation of a over-complex model that overfits the training data, that is, the model accurately learned the training dataset but is not able to generalize well to a test dataset.
To prevent overfitting, one possible strategy is to early stop the iteration towards the leaf, and leave a node with more than a single sample left.
An other strategy, which leads to the construction of a Random Forest, is the creation of a stronger model, more prone to generalize data reducing overfitting.

\paragraph{Random Forest} \hfill

A Random Forest is a collection of decision trees where each tree is slightly different from the others, each tree tends to overfit, but it does so in a different way, so that we can reduce overfitting by averaging different results.
Random forest gets its name because of the insertion of randomness during the construction of these different trees.
The process whereby a Random Forest is built, given an input dataset X with N different samples, there are two main steps where randomicity plays an important role: when selecting the number of samples on to which built a tree n\_trees, and when selecting the number of feature to use to grow each tree.
In summary, a random forest, once selected the number of tree to be created:
\begin{enumerate}
\item Randomly select n$\leq$N samples from the input dataset, this operation is called bootstrap.
\item Grow a decision tree from this bootstrap sample, but to each split, it randomly limit the number of feature available and compute the best split on this remaining subset of feature. This avoids correlations between trees resulting in better performances
\item Repeat steps 1) and 2) n\_trees times.
\item Once all the trees are created, if we want to make a prediction, we put together all the trees, and use each one of them to make predictions.
The final prediction of the random forest is assessed by majoring vote: the predicted class will then be the one predicted by the majority of classifiers.
\end{enumerate}

%This steps are implemented in the Python's package Scikit-Learn


%Randomness within trees of a single model is introduced with three strategies:
%\begin{enumerate*}
%\item Randomly select some input data and buiding a tree with this subset of data
%\item Randomly select a different subset of features at each split
%\item A combination and implementation of both the previous strategies results in a model in which each tree has reduced predictive power, but when put %all together, it results in an improved predictive power with a drastical reduction of overfitting.
%\end{enumerate*}

Introducing this kind of randomness in a model results in a lower correlation between models and this brings to a reduced variance between outputs.
Informations regarding features importance can be extracting from a random forest classifier by extracting them from each tree, and by averaging the impurity decrease from all the decision trees in the forest.



\section{Deep Learning: ANN}

We discussed in the previous paragraph one of the most important machine learning algorithm, but, as mentioned before, there's a subset of machine learning algorithm with a charactristic common structure, thanks to which they get their name as Deep learning algorithm.
The structure of algorithms belonging to this family are inspired by a brain neuronal structure, and even if in a simplified way, they try to emulate the learning process of a brain.
These algorithms own the adjective \textquotedblleft deep \textquotedblright to their structure: they are organised in layer, each one containing several foundamental unit called neuron.
Neurons between layers are connected to each other like synapsis transmit a signal between neurons in a biological brain.
For this structure which reminds a neuronal brain structure,they are caller artificial neuronal networks
%Some kind of algorithms are called Artificial Neural Networks (ANN) and, such as a neuronal structure, they are made of several neurons organized into layers; each layer contains certain amount of neurons each one linked to each others.

\begin{wrapfigure}{r}{0.4\textwidth}
\centering
\includegraphics[width=1\linewidth]{ml/perceptron_example}
\caption{A schematic representation of a perceptron: this perceptron receives an input vector t, with n features $x_1, ... x_n$, each one weighted with a different weight $w_1, ... w_n$ and a offset $w_0$, and compute a linear combination of them and returns an output, activated by a step function: it returns 0 if the linear combination's value doesn't reach a certain threshold, or returns the value itself if it does.}
\label{fig:perceptron}
\end{wrapfigure}

The foundamental unit which constiture a neuronal network is an artificial neuron, one of the most relevant example of artificial neurons is the \emph{perceptron} shown in figure \ref{fig:perceptron}.

A perceptron is the foundamental unit of a supervised deep learning algorithm: it takes as input a data, for example an n-dimensional array $\textbf{x} = \left\{ x_1, x_2, ..., x_n\right\}$ and returns an output: a real number computed by applying a function to a linear combination of all the inputs $ y = f(z) = f \left ( \sum_i w_{i} x_i + b \right )$.
The function that determines the output is called \emph{activation function}, and can be either a simple step function, or a slightly more complex function we will briefly discuss in section \ref{sec:activation_functions}.



\begin{wrapfigure}{r}{0.4\textwidth}
\centering
\includegraphics[width=1\linewidth]{ml/ann_example}
\caption{Schematic representation of an artificial neural network containing an input layer, three hidden layers and an output layer: a single vector contained n features (input 1, ... input n) is given as input to the first Input layer i. All the inputs are linked to every neurons of the first hidden layer, which would compute a linear combination of its inputs. All the hidden layers are linked to the other hidden layers, and at the end, n different outputs are returned.
}
\label{fig:ann}
\end{wrapfigure}


A deep neural network is a hyerarchical organization of neurons into layer, connected to each other.
Input data are passed to the first input layer where each neuron acting like a perceptron, produces an output using the activation function which is set the same for every neurons of the layer, but can differ from the activation function of other layers.
Once collected each and every output from all the first layer neurons, they are passed to the second layer and become the input to each neuron belonging to this layer.
This is reiterated through all the layers up to the final output layer.
As schematized in figure \ref{fig:ann} an artificial neural network is mainly composed of three parts: an input layer, come middle layers also  called \emph{hidden layer} and a final layer. Each neuron of a layer is linked to all the neurons of the previous and next layer.
The connection between two neurons is weighted, according to an initial setup of the network during which weights are randomly set.

When reach the end of this process, we obtain the output: a numerical value related to every single input data $\textbf{x}$.

Using matrix formalism the entire input-output process can be written as

$\textbf{y} = \sum_j^n w_{ij} x_j + b_i $ where $\textbf{x} \in \Re^n \ \textbf{y} \in \Re^m $ being m the output dimension given by the number of neurons in the output layer.


During the training process, given an input, the algorithm calculate an output (prediction) and this process is referred as  \emph{forward propagation}.
This prediction is then compared with the actual value this output should be, and during the \emph{backpropagation} the algorithm modifies its weights in order to minimize the difference between the actual and the predicted value.

The goal of a machine/deep learning algorithm is to learn from data using a train dataset, and generalize it in order to perform well on an unseen dataset called Test dataset.
Performing well means to produce a low error on the Test dataset after minimizing the error on train dataset.


To fully define a neural network we just need some parameters called hyperparameters, the principals of which are:

\begin{itemize}
\item Number of layers
\item Number of neuron for each layer
\item Activation function for each layer
\item Number of epochs (or iterations)
\item Learning rate
\end{itemize}




One of the most important concept in machine learning when characterising a network is its Capacity.
It refers to the level of complexity that a model is able to learn.
This is strictly linked to one of the critical issue in machine leraning: the concept of underfitting and overfitting.

Underfitting occurs when the model isn't able to learn the required amount of informations during train; this usually happens for shallow network, when the model has a small number of parameters in regard to the number needed to explain the input features, having low parameters results in poor performances because the model isn't able to learn the underlying structure of a complex dataset.

On the contrary, overfitting occurs when the model has too many parameters compared to those required to learn the input features.
What happens then is that the model memorizes all the data it sees during train but it is not able to generalise them.
As a result, the model performs well on the Train dataset and has low performances in an unseen dataset, called Test datset.

As an example in a two-dimensional space, if we aim to fit a dataset consisting of some points sampled from a quadratic curve, using three different functions: a linear function, a quadratic function and a polynomial function with grade equal to the number of data points.
As shown in figure \ref{fig:underoverfitting} the linear model is not able to describe the data distribution while the polynomial function has too many parameters, and it is perfectly able to fit our distribution, but it would perform very poorly to describe a dataset sampled from the same quadratic distribution, furthermore, if the number of parameters (the grade of the polynomial in this case) is greater or equal to the number of data points, we obtain that an infinite number of different curves are suitable to fit our data, and finding the best one, which performs well on the test dataset is an hard task.
We should therefore pay attention when choosing the number of parameters on a model to avoid under- or more likely over- fitting.


\begin{figure}
\begin{subfigure}{0.4\textwidth}
\includegraphics[width=\textwidth]{ml/underoverfitting_mod}
\caption{Underfitting, overfitting and appropriate fit in a 2D dataset: three models with different capacities are trained on blue data, if the model is too simple, in this case if the fit function has a few free parameters, it is not able to properly fit data, on the other hand, if it has too much parameters in respect to how much we would need to fit our data, it perfectly match train data distribution, passing through each data point, but performs very poorly on the test dataset (orange points). Whit an appropriate number of parameters, (middle figure) the model is able to fit train data, and generalizes well to Test data.
}
\label{}
\end{subfigure}
\begin{subfigure}{0.4\textwidth}
\includegraphics[width=\textwidth]{ml/underoverfitting_grafico}
\caption{}
\label{}
\end{subfigure}
\caption{General, qualitative trend of a loss function vs. capacity of model: the more capacity it has, the more is able to reduce the error on train set (blue dotted line) with a tendence to overfit train data. Beyond a certain capacity the error on the test dataset (red line) increases because overfitting of training data results in a lower ability to generalize to new data.
}
\label{fig:underoverfitting}
\end{figure}

\paragraph{Regularization strategies} \hfill
Some regularization procedures are often implemented to avoid overfitting, the strategy is to build a model with a capacity slightly higher than the necessary in order to perform well on the training dataset, and then, to avoid overfitting, implement some regularization techniques to achieve good generalization performances.

Some of the most popular are Dropout and Batch-Normalization

\begin{itemize}
\item Dropout is a strategy which allows us to obtain the same effect of training train different model with our input data and take the average output at the end without paying high computational costs.
It is usually applied to hidden layers' neurons and accomplish this task by randomly dropping a certain fraction of hidden neurons and their connections, during each training cycle. The percentage of neuron to drop, correspons to the probability p for a single neuron to be dropped and it is set by the user.
A visual representation of what occurs is illustrated in figure \ref{fig:dropout}.
When discarding some neurons in a layer, the remaining neurons needs to rescale their weights to account the missing connection; doing so, every neuron cannot rely on the the input of all the preceding neurons and the network is forced to learn more robust patterns from the data.
\item BatchNormalizzation is a regularization scheme that has been commonly adopted since its introduction in 2015. It is based on the observation that a neural network works and performs better when its input are normalized because this prevents the saturation of its neurons.
A neuron can in fact saturate and settle to a certain value because of an high input, this causes the neuron outputs value to be always close to the asymptotic end of its activation function, resulting in a biased and less accurate prediction.
What is essentially done is then a simple scaling of each neuron's input: for a layer l with d neurons its input $\mathbf{x} = \{x_1^l, x_2^l, ..., x_d^l\}$ is normalized by removing the mean value across all the input data, and divide for their variance $x_i^l \rightarrow \tilde x_i^l = \frac{x_i^{l} - \mathbb{E}[x_i^l]}{\sqrt{Var(x_i^l)}} $

\end{itemize}

\begin{figure}
\includegraphics[width=.5\linewidth]{ml/dropout.png}
\caption{Schematic representation of a 0.5 dropout procedure: in a standard neural network, each neuron is linked to all the others, but if we implement a 0.5 dropout, for each iteration, neurons on hidden layers have a 50\% probability to be dropped and excluded from the computation of outputs}
\label{fig:dropout}
\end{figure}

\paragraph{Cross validation procedures} \hfill
To monitor the evolution of the model's performances during train is a common practise to split the train dataset into two subsets: one that will be the actual train dataset, and a smaller one, called validation dataset used to make constant checkups on how well the model is learning with the train set, by making constant test of its performances.

The validation dataset is used for the fine tuning of hyperparameters and it typically consists of 10-30\% of the whole train dataset.
When assessed the best hyperparameters combination for our model, is possible to actually train the model using the entire Train dataset.
Even if we manage to use the entire Train dataset for the training process,
However, when dealing with small datasets, dividing it into Train and Test can be a non trivial issue because of the shortage of data for a proper train procedure.

To work this out, a prodecure called k-fold cross validation can be implemented, at the cost of increasing computational costs.
It consists in the creation of k different partition of the main whole dataset (before splitting it into train + test).
From these partitions, k-1 are used as train, and the last is used as Test. Moving forward every combination of partitions is used so that each different partition is used as test, and the others as train.
Then for each run k-th there are two subsets (a train and a test) from the original dataset. This way, for each iteration, the Train dataset is different, and the model is tested on a different dataset each time.
In practise what is usually done, is not a sequential partitioning the dataset, to avoid creating folder containing all the same label if the dataset was ordered, but randomically picking data in order to create subsets containing the same proportion between classes as the main dataset.
After that, we can imagine this process like the creation of k different models, each one trained on a different partition(k-1 folds) and tested on the remaining k-th fold, and we take as a result the average score across all these models.


\begin{figure}
\includegraphics[width=0.4\linewidth]{ml/kfold_example}
\caption{Schematic representation of a k-fold cross validation procedure with k = 5: Dataset is partitioned into 5 subsets, four of which are used for training and the remaining one for the testing dataset. during each iteration a model is trained over 4 different folds and tested on the remaining one, untill each fold has been used at least once as part of training and once as test.
}
\label{fig:kfold}
\end{figure}


\section{Activation functions}
The activation function of a neuron, and consequently of a layer defines the output of each neuron belonging to that layer, there are different activation function, linear or non-linear.
A linear activation function outputs a value $f(x) = w^T x + b$ where $w^T$ indicates the transpose of the weigths vector.
In practise, however, it would not be very useful to introduce a linear activation function since we aim to introduce non-linearity in our model to tackle more complex models.
There are several non-linear function available and, depending on what data we are working on, or on what kind of classification task we are performing, some of the most popular are


\begin{figure}[h!]
\centering
\begin{subfigure}{0.3\textwidth}
\includegraphics[width=\textwidth]{ml/relu}
\caption{Rectified linear unit (ReLU) function: it returns zero if its argument is negative and returns the argument itself if it is positive.}
\end{subfigure}
\begin{subfigure}{0.3\textwidth}
\includegraphics[width=\textwidth]{ml/sigmoid}
\caption{Sigmoid function: returns a value between 0 and 1, according to equation \ref{eq:sigmoid}}
\end{subfigure}
\caption{}
\end{figure}




\begin{itemize}

\item The ReLU functions $\phi(z) = max(0, z) = \begin{cases} 0 \ for \ z<0 \\ z \ for \ z>0 \end{cases}$ which returns the maximum value between the input and zero, it essentially puts to zero all negative inputs and leaves the positives untouched.
A modified version of the ReLU function called Leaky ReLU was introduced defined as
$\phi_{leaky}(z) = \begin{cases} \alpha z \ for \ z<0 \\ z \ for \ z>0 \end{cases}$ where $\alpha$ is a coefficient usually $< 1$, typically of the order of $10^{-2}$


\item The sigmoid function, or logistic function is defined as
\begin{equation}\label{eq:sigmoid}
f(x) = \frac{1}{1+e^{-x}}
\end{equation}
 and outputs a real number between 0 and 1. For this reason is a common choice when we have to predict a probability, for example in a binary classification problem it can be interpreted as the probability that output belongs to the class 0 or 1.


\item The softmax function for multiclass classification, gives the probability of the input belonging to each (mutually exclusive) class. The softmax can't be applied independently to each output $z_i$, since it depends on all elements of $\textbf{z}$ The probability of belonging to the i-th class over a total of M classes is
$p(i = i|z) = \phi(z) = \frac{e^z_i}{\sum_{j = 1}^M e^{z_j}}$ being M the total number of classes
\end{itemize}


\section{Loss functions}\label{sec:loss_functions}
%Ricorda: \hat y_i = sigma(x_i) che è il mio output del modello
%Articoli da cui ho preso info: ml for physicists e dispende di baldini

In order to train a network and improve its performances we need to compare the predicted (output) value with the actual value and compute some sort or error, or distance between these two values, and try to gradually minimize it.


The function to minimize is denoted as the \emph{loss function}
Computing the loss, returns a value of how much the prediction is far from the actual label, and then, the network's weights are modified in order to reduce this value and searching for a minimum of this function.


and the main technique to do so, is the likelihooh maximization

The most common loss function is the cross-entropy loss, it relies on the concept of cross-entropy between two distributions $\hat y \ and \ y$ and is defined as
$H(y, \hat y) = -\sum_i^n y_i \cdot log(\hat y_i)$
Where the sum is intended over all the n possible values a variable y can assume (all the n possible classes in a classifiction problem).
%In case of multi-class classification we call this function categorical crossentropy,
If we only have two classes, we are performing a binary classification and the loss becomes, (if we explicit the sum for two $i = \left\{0, 1\right\}$ and take the average value across all the data samples) we obtain the binary crossentropy:
\begin{equation}\label{eq:binary_crossentropy}
J = -\frac{1}{N} \sum_{j= 1}^N y_j log(\hat y_j) + (1-y_j)log(1-\hat y_j)
\end{equation}

The estimation of the minimum of this function can be seen as the maximization of the likelihood:
Given a probabilistic model depending on m different parameters $\theta_1 ... \theta_m$, the likelihood is the probability of observing a value $\hat y_i$ given a set of parameters $\{\theta_i \}_1 ^m$
\begin{equation}
 L(\theta_1, ..., \theta_m | \hat y_i) = P(\hat y_i|\theta_1, ...\theta_m).
\end{equation}
For a subset of observed value we can write the likelihood as $L = P(\hat y_1, ..., \hat y_n | \theta_1, ...\theta_m)$ .

This probability, if we are dealing with independent variables, can be rewritten as the product of the single probabilities, of observing each sample $\hat y_i$ given out model with parameters $\{\theta_i \}_1 ^m$:
\begin{equation}\label{eq:likelihood_product}
L =\prod_{i=1}^n p(\hat y_i|\theta_1, ...\theta_m).
\end{equation}

%Our goal is to perform the maximization of this function during a logistic regression:
%A logistic regression is the estimate of the best model's parameters that models the discrete probability of two events:

In our practical problem if we aim to maximize this function, the parameters to be optimized $\left\{\theta_i \right\}_1^n$ are the weights of the DNN model $\left\{w_i \right\}_1^n$ and our data is a set of N datapoints $\left\{ x_1 ... x_N\right\}$ each one associate with a label  $\left\{ y_1 ... y_N\right\}$ which can be either 0 or 1 which can be indicated as D = $\left\{ (\mathbf{x_i}, y_i)\right\}$.

From now on we will denote our parameters to optimize as $\mathbf{w} = w1, .., wm $
Given an input data $\mathbf{x_i}$  and an activation function (sigmoid for example, which we recall is of the form $\sigma(z) = \frac{1}{1+e^{-z}}$) we model the probability of $x_i$ to belong to the class $y_i = 1$ as:
%or in other words, the probability to observ $y_i$ given an input $x_i$ is given by

\begin{equation}
 P(y_i = 1 | x_i, \mathbf{w}) = \sigma(\mathbf{x_i\cdot w}) = \frac{1}{1+e^{-\mathbf{x_i \cdot w}}}
\end{equation}

%which in terms of a regression can be wrote as 1 - the probability to belong to the negative class namely
And since we are dealing with a binary classification, where $y_i$ can be either 0 or 1, the probability to belong to one class is 1 minus the probability to belong to the other class
\begin{equation}
P(y_i = 1) = 1 - P(y_i = 0)
\end{equation}

Putting this all together for a set of data $D = \{ (\mathbf{x_i}, y_i)\}_1 ^N$ substituting this result in the log-likelihood function we obtain the likelihood of my observations $\hat y_i = \sigma (x_i|w)$, from equation \ref{eq:likelihood_product} becomes
%of observing our data under our model
\begin{equation}
L = P(D|\mathbf{w}) = \prod_{i=1}^N \sigma(\mathbf{x_i w})^{y_i} (1-\sigma(\mathbf{x_i w}))^{1-y_i}
\end{equation}

The best parameters are those that maximize the likelihood:
$\begin{cases} \frac{\partial L}{\partial w_1}(w_1, ..., w_m) = 0 \\ ... \\ \frac{\partial L}{\partial w_1}(w_1, ..., w_m) = 0 \end{cases}$
To compute the derivative of a product is a nontrivial task, hence is much simpler to compute the logarithm (since the logarithm is a monotonic function, so compute the maximum of a function is the same as computing the maximum of the logarithm of that function).

We obtain then
\begin{equation}
log(L) = \sum_{i=1}^N y_i log(\sigma(x_i \mathbf{w})) + (1-y_i)log(1-\sigma(x_i \mathbf{w}))
\end{equation}

which taken with a negative sign, and averaged over all the N data samples corresponds to the cross-entropy function in equation \ref{eq:binary_crossentropy}.


This result obtained for binary classification can be generalized to perform a multi-class classification called Softmax regression.
In a multi-class classification, labels $y_i$ become binary vectors of dimension M, with all but one entry equal to zero, and the only entry equal to 1 specifies the class.

In this case, given a model with parameters $\mathbf{w_k}_{k=0}^{M-1}$ the probability for an input $\mathbf{x_i}$ to belong to class m' is given by the softmax function

\begin{equation}
P(y_{i,m'}= 1 | \mathbf{x_i}, \{\mathbf{w_k} \}_{k=1}^M ) = \frac{e^{-\mathbf{x_i^T w_{m'}}} }{\sum_{m = 1}^{M} e^{-\mathbf{x_i^T w_{m}}}}
\end{equation}
and the relative negative log-likelihood called the \textbf{categorical crossentropy} will be in the form:

\begin{equation}
L = -\sum_{i=1}^{n} \sum_{m= 0}^{M-1} y_{im} log(P(y_{im}=1|x_i, w_m)) + (1-y_{im})log(1-P(y_im=1|x_i, w_m))
\end{equation}

where there's a summation over the index i which indicate the i-th sample data and over the index m, concerning all the possible M classes.


%The algorithm employed by a feedforward network for weights' fine tuning is called \emph{backpropagation} (backward propagation of errors).

\section{Gradient descent and Backpropagation}\label{sec:backpropagation_theory}

To perform the minimization of our loss function, even if it'd be theoretically possible to find a minimum by means of an analythical way, in practise, in all the neural network applocation, the number of weigths is so big that a numerical methods must be employed, the most popular of which is the \emph{gradient descent}.

Let's denote the loss function as $J(w) = \frac{1}{n}\sum_{i = 1}^N E(y_i, \hat y_i)$
Given a generic loss function $J(\textbf{w})$ where $\textbf{w}$ is a vector of weights, the minimum of this function corresponding to the vector $w_0$ can be found by following these simple steps:

\begin{enumerate}
\item Choose a random initial guess for \textbf{$w_0$} and start iterating
\item At iteration i+1 we have a weights vector \textbf{$w_{i+1}$} given by the formula \textbf{$w_{i+1}$} = \textbf{$w_i$} - $\eta \nabla$ J(\textbf{$w_i$})
\end{enumerate}

 where the $\nabla$ indicate the gradient of the cost function with respect to \textbf{$w_i$} components, and $\eta$ is the learning rate of our gradient algorithm, it specifies how big each step will be during the descent toward the minimum of the cost function.

%To find the right variation to the weights we can use the gradient descent algorithm, so we compute the gradient of the cost function $\nabla J(w)$ so that we can compute the new weight $w = w + \Delta w$ where $\Delta w = - \eta \nabla J(w)$ which expressed in component becomes $\Delta w_j = -\eta \frac{\partial J(w)}{\partial w_j} = -\eta \cdot \sum (\phi(z^{(i)}) - y^{(i)} )x_j^{(i)}$.

A drawback of this gradient descent algorithm is that before upgrading the weiths we need to compute all the gradients for each data in input data namely, after the whole dataset is been seen, resulting in a huge computational cost.

An optimized version of this algorithm calle dStchastic Gradient Descent (SDG) is then often used because it has some advantages such as decrease computational cost and, as the name suggests, introduce stochasticity resulting in less chance for this algorithm to get stuck in a local minimum.
It works by approximating the gradient of the cost function calculated with all the input data, with a gradient computed using only a small subset of input data called minibach.

If we have N input data, we can create subsets containing m elements, and obtain N/m subsets.
Therefore, the gradient is computed on a mini-batch and weights are updated and this process is repeated for each mini batch.
%mettere equatione ml_physicists p.17  (19)???
Once the gradient was computed over all the mini batches, this is called an epoch, which is one of the hyperparameters to be set when choosing a model and a training strategy.

%whose difference is that the upgrade of the weigths is sequential, for each training sample, so that the summation is not computed.

Even though SGD already performs quite well, it can be further tuned introducing the concept of momentum.
Momentum is represented by a parameters $0\le \gamma \le 1$ and it takes track of the discending direction by running an average over all the preceding encountered gradients, and helps the algorithm speeding up the descending process if in a certain direction the gradient is persistent.
But in addition we can take into account even the steepness all over the dimension, and to do so we must introduce second order momenta in our algorithm, also called uncentered variance.

To accomplish this, we would ideally need to to calculate the hessian matrix but this comes at the cost of increasing computational cost.
Recently intriduced algorithm can accomplish this task by approximate the calculus of the second momenta.
Doing so, we keep track of the curvature and take big leaps in steepest direction and small steps in flatter ones, allowing us to adaptively change the descending step size according to the shape of our multi-dimensional curve.
One of the most popular among these algorithms is Adam: it accomplish this computational task by making use of two different optimization algorithms: AdaGrad and RMSProp. and adapts the learning rate taking into account both the first and the second moment, leading it to perform better and quicker in findind the minima than simple SGD with momentum algorithms.



After the computation of the loss value, the next step is the process called backpropagation, through which network's weights are updated.
Backpropagation can be summarized in 3 steps:

\begin{enumerate}
\item calculate all the activation of each neuron in a layer
\item calculate the error of the output layer L, which for each neuron will be $\Delta_j^L =\partial E/\partial z_j^L$ being z the output of this neuron
\item Backpropagate the error exploiting the chain rule of derivatives to compute the error of every neuron in the previous layer l $\Delta_j^l = \sum_k \frac{\partial E}{\partial z_k^{l+1}}\frac{\partial z_k^{l+1}}{\partial z_j^l}$
\item Compute the gradient of the function E with respect to model's weights and modify them using equation ..
\end{enumerate}



\section{Dimensionality reduction: PCA}
PCA is a method to perform dimensionality reduction, it is an unsupervised learning algorithm which aims to reduce the dimensionality of input data, preserving as much information as possible,
It does so attempting to learn a new representation of data with lower dimensionality than the initial one and whose element have no linear correlation with each other. It performs then an orthogonal transformation of data, in order to find the direction along which data variance is biggest.

Concretely, if we consider a set of n input data vectors lying in a space $\Re^p$, we can represent them as a matrix $X \in \Re^{n \times p}$ where n is the total number of input data and p is the dimensionality of each data (the number of features in each data vector).

We can suppose without losing generability that feature distribution across our data have zero mean, so that for each feature $x_i$ with i = 1, ..., p, $\mu_i = \mathbb{E}[x_i]=0$.

PCA computes the covariance matrix $\Sigma \in \Re^{p \times p}$ given by $\Sigma(X) = \frac{1}{n-1} X^T \cdot X$, where each entry can be written as $\sigma_{jk} = \frac{1}{n-1}\sum_{i = 1}^n (x^{(i)}_j - \mu_j)(x^{(i)}_k - \mu_k)$, and if we are in the hypotesis of zero mean $\mu_j = \mu_k = 0$, we obtain $\sigma_{jk} = \frac{1}{n-1}\sum_{i = 1}^n (x^{(i)}_j x^{(i)}_k)$.

Since the covariance of a variable with itself is its variance (Cov(a,a)=Var(a)), in the main diagonal we actually have the variances of each initial variable. And since the covariance is commutative (Cov(a,b)=Cov(b,a)), the entries of the covariance matrix are symmetric with respect to the main diagonal, which means that the upper and the lower triangular portions are equal.

We are interested in finding a new representation performing a linear transformation that reduces the covariance between different features.

The eigenvectors of the principal components represent the direction of the maximum variance
and the corresponding eigenvalues, defines their magniture.
 %The principal components of the matrix X are given by the eigenvectors of the matrix $X^T X$ so that we can write $X^T X = W \Lambda W^T$ where $\Lambda$

%PCA finds a representation through a linear transformation z = $X^T W$.

To find them, PCA makes use of Singular Value Decomposition (SVD): a linear algebra analysis which is basically a factorization of a n x p matrix X that (in a case X is a real matrix), to rewrite it as X = $U S W^T$ where U and V are respectively m x m  and n x n orthogonal matrices whose columns are called left- and right-singular vectors of X, and S is a  m x n diagonal matrix.
Diagonal values of S: $S_{ii} = s_i$ are unically determined by X and are called singular values of X.

According this analysis, we can rewrite
\begin{equation}
\begin{aligned}
X^T X & = (U S W^T)^T(U S W^T) \\
& = W S^T U^T U S W^T \\
& = W S^2 W^T
\end{aligned}
\end{equation}
where we used the definition of orthogonal matrix for U $U^T U$ = I.

We therefore rewrite the covariance matrix $Var[X] = \Sigma$ as $\Sigma = \frac{1}{n-1}W S^2 W^T$.

%Ne sei sicuro di questo?????
Here the singular values of X: $\lambda_i$ are related to eigenvalues of the covariance matrix by the relation $\lambda_i = s_i^2/(n-1)$

Using this results if we compute the new data matrix Z = $X^T$ W, its covariance matrix
\begin{equation}
\begin{aligned}
Var[Z] & = \frac{1}{m-1}Z^T Z \\
 & =\frac{1}{m-1} W^T X^T X W   \\
 & = \frac{1}{m-1} W^T W S^2 W^T W \\
 & = \frac{1}{m-1}S^2 .
\end{aligned}
\end{equation}
This shows that when we project the data x to z using the linear transformation W, the resulting transformation has a diagonal covariance matrix which implies that the individual elements of z are mutually uncorrelated.

To reduce the dimensionality of our data, we need to extract the first  $\tilde p$ eigenvalues from the covariance matrix, ordered in a descending order, and collect the corresponding projection matrix $W_{\tilde p}$ made of the corresponding vectors of the righ-singular vectors W.

And the projection of our data from p-dimensional space into $\tilde p$-dimensional space is $\tilde Z = X\tilde W_p$

An important parameters to quantify \say{how well} PCA is able to explain is the \emph{variance explained ratio}, given by the ratio of an eigenvalues and the sum of all the eigenvalues.

\begin{equation}
\frac{\lambda_i}{\sum_{k = 0}^p \lambda_k}
\end{equation}

The singular vector with the largest corresponding singular value is the first principal component, and with decreasing order, the other mutually orthogonal vectors are the next principal components.
In figure \ref{fig:pca_example} is reported a graphic example of two PC extracted from a set of two-dimensional data, along the direction where variance is greater, there's the first principal component and the second component is in the orthogonal direction.

\begin{figure}[h]
\centering
\includegraphics[width=.5\linewidth]{ml/pca_example.png}
\caption{Example of the first two principal components in a 2D dataset}
\end{figure}



%ration lambda_i/sum lambda_i ??

\section{How to assess network's performances}
Once the model has been trained, it is finally applied to the test set to assess the performances of the model to an previously unseen dataset.
A comomon metric for model performances' evaluation is Accurcay: it is simply defined as the ratio between the total number of correct prediction and the total number of predictions (total number of the test's data), so, using T and F for true and false, and P and N for positive and negative, accuracy is so defined
\begin{equation}
Accuracy = \frac{TP + TN}{TP+TN+FP+FN}
\end{equation}
Even though accuracy may seem a good parameter, it isn't the most accurate one, when we're dealing with a unbalanced test dataset, if for example, in a binary classification with A and B classes, if just 2 out of 10 samples belongs to a different class, let's say A and B, and the model predicts every data belonging to the class B, we will get a score of 80 \% accuracy, which is not a thruthful result because the model isn't making any distinction between the two classes.
There are then other ways for quality assessment that overcome this problem. One of this is based on the introduction of two quantities: precision and recall.
Precision is the ratio between true positive and total positive cases (TP + FP), which is a measure of how many positive predicted cases were actually positive, while recall measures the percentage of actual positives that were correctly classified, or in other words it represents the true positive rate.
\begin{equation}
Precision = \frac{TP}{TP+FP} \qquad \qquad Recall = \frac{TP}{TP+FN}
\end{equation}
Unfortunatelly precision and recall are linked in a sense that often enhancing one decreases the other and vice-versa.
%esempio page 283
There's therefore a trade off between recall and precision. If we want for instance a model to score more true positive, risking to have more false positives, we can set a threshold.
To reach less the x\% positive missing means set the recall to (100-x)\%, this operation is called setting the operating point and is done by setting a threshold.
This threshold though is not always estblished at the beginning, and operating point that would be the best suited for our classification goals is not clear a priori.
One way to summarize the information a precision-recall curve can provide us, is to compute the area under the curve, known as \textquotedblleft average precition \textquotedblright
What is done is study the model under all the possible thresholds.
This is achieved by studying the precision-recall curve of which is reported an example in figure \ref{fig:precisionrecall}.
The closer a curve lies on the upper right corner (high precision and high recall), the more correctly the model is working.

Besides the precision-recall curve, a similar curve is usually employed to study different thresholds' effect, is called the receiver operating characteristics curve, usually referred as ROC curve.
% Insert ROC curve computed using the random forest already implemented in my data.
The ROC curve is build from the true positive rate (recall) and the false positive rate (FPR) and shows the evolution of TPR vs FPR.
The ideal curve would be close to the top left (high tpr and low fpr) and the less accurate is the model, the more this curve tend to lay down to the bisector line.

To summarize the model's performances with a single number using ROC's curve information, we compute the Area Under the Curve AUC.
The reference value for an AUC is 0.5 which is obtained when a model is just randomly predicting and it corresponds to a curve laying on the bisector.
The AUC can be explained as the probability that randomly picked point from the positive class, will have an higher score (according to the model) than a randomly picked point from the negative class so that the percentage value of AUC is an estimate of the probability that the model is able to distinguish between the two classes.

An example of ROC curve is shown in figure \ref{fig:roccurve}


\begin{figure}
\centering
\begin{subfigure}{0.4\textwidth}
\includegraphics[width=1.\textwidth]{ml/precisionrecall_example.png}
\caption{Example of a precision-recall curve plotted for different thresholds values}
\label{fig:precisionrecall}
\end{subfigure}
\begin{subfigure}{0.4\textwidth}
\includegraphics[width=1.\textwidth]{ml/roccurve_example.png}
\caption{Example of a ROC curve values of True positive and False positive rates for different thresholds values}
\label{fig:roccurve}
\end{subfigure}
\caption{}
\label{}
\end{figure}



\chapter{Explainable AI: a game theory approach}\label{chap:shapley_values}
As methods to learn pattern from data becomes more complex, they become harder to interpret, deep learning represent an example of a technique to search for nonlinear relations between data, but introducing nonlinearity, makes an outcome difficult to interpret and it ends up considering a machine learning model as a black box without any hint of its inner precesses that occurred when a result is output. But if a model perform well, or bad on a dataset, we would like to know why and what feature influenced the most the outcome of our model.

One state-of-art explainability algorithm is built making use of important concepts from game theory, for this reason in the following section we are going to explain some useful theory underlying explainatory models.


\section{Shapley values}
%Versione definitiva
Game theory is a branch of mathematic related on the study of mathematical models to  conceive social situations among competitive players \cite{ross-2021}

It have had a great development in the XX century expecially during and after the Second world war thanks to the contribution of mathematicians like John Von Neumann, John Nash and Lloyd Shapley.
Game theory mainly focuses on two main research areas: cooperative and non-coperative game.
\begin{itemize}
\item Non-cooperative game concern competition between individual player who don't cooperate each other and can't form coalitions. The main task on non-cooperative games can be summarised as the search for a good strategy for each player. Each player's objective is in fact the maximization of his own utility function.
A big contribution to the development of this theory came from von Neumann, or John Nash with the concept of Nash equilibrium \cite{nash-1950}.

\item Cooperative games concern competition between groups of player forming coalitions, each coalition plays as a single participant and earns a payoff; one of the the main tasks related to cooperative games is to find a way to divide the total utility among the member of a coalition in an equally way proportionally of how much they contributed to the final score.
\end{itemize}
In 1951 Lloyd Shapley in 1951 Introduced a way to compute the exact amount of payoff for each player making use of what were named after him: Shapley values \cite{shap-1951}\cite{shap-1953}.
Shapley introduced its values in the field of coalition games, therefore to properly understand his work, we need to briefly illustrate what a coalition game consists of.

A \textbf{coalition game} involves N players, and different subsets called coalitions created from these N players.
Each subset of players S gain a payoff at the end of the game.
A function $\nu$ maps every subset to its payoff $\nu(S)$ = payoff(S) $\in \Re$.
On the basis of which player had more influence in this final score, we ask how to split this payoff in a fair way between each player of the subset, where \textquotedblleft fair \textquotedblright is to be intended as to each, proportional to his own contribution.
A solution for this problem comes from \textbf{shapley values} $\phi_i(\nu)$.
They are specific for each player i $\in$ N in a coalition S $\subseteq$ N and the idea behind them is the marginal contribution of that player to the final score, where marginal contribution is defined as the difference on the score of the coalition when player i joins the coalition.
In other words they are the difference between the coalition's score with player i and the coalition's score without him $\text{marginal contribution = }\nu (S \cup \{ i \}) - \nu(S)$. \cite{algaba-2021}

Shapley defined his coefficients (Shapley values) for a player i as a weighted average of marginal contribution values, over all the possible subsets that include player i.

The mathematical formulation Shapley introduced for his value is
\begin{equation}
\begin{aligned}
\phi_i(\nu)  & = \frac{1}{N }\sum_{S\subseteq N\ \setminus \{ i \} }  {N- 1 \choose |S|}^{-1} [\nu (S \cup \{ i \}) - \nu(S)] \\
& = \sum_{S\subseteq N\ \setminus \{ i \} }  \frac{S! (N-1-S)!}{N!} [\nu (S \cup \{ i \}) - \nu(S) ]
\end{aligned}
\end{equation}
%That can be interpreted as $\frac{1}{N} \sum \frac{\text{marginal contribution of i to the coalition}}{\text{number of coalitions excluding i of the same size os S}}$
And they exactly represent the amount of reward for each player i.



\addcontentsline{toc}{chapter}{MATERIALS AND METHODS}
\chapter*{MATERIALS AND METHODS}

\chapter{Dataset: ABIDE I $\&$ II}\label{chap:dataset}


Data we are going to work on belong to the ABIDE dataset (Autism Brain Images Data Exchange): a project founded with the aim of investigating ASD using magnetic resonance structural images and resting state fMRI scans.
To address the problem with these two different approaches they collected data acquired over the years from different medical centers, and put them together in a single dataset.

The whole ABIDE dataset was published in two releases: ABIDE I released in August 2012 and containing 1112 patient scans, and ABIDE II released in June 2016 containing 1114 scans.
For each site autism was diagnosed either by gold standard diagnostic instruments, clinical judgment or a combination of clinical gold standard procedures.

ABIDE I includes scans collected from 17 different sites, and the 1112 patients consist on 539 patients with ASD and 573 typical control patients.
ABIDE II includes scans collected from 19 different sites, and the 1114 patients consist on 593 patients with ASD and 521 typical control patients.
Not every site belonging to ABIDE II is different from those of ABIDE I, but, even though some medical centers are the same, the scanner type, or acquisition pipeline and parameters may have been changed during the time interval between the two releases, so in all our analysys, they are regarded as different acquisition sites.

In addition to scan images, ABIDE provides every information related to each patient, as age, sex, intelligence quotient (FIQ), eye status during the scan (open or closed), and every additional clinical information provided by patients.

The vast majority of patients are males as shown in the histogram \ref{fig:mf_site}, for a total amount of 1804 males and 422 females.
Control/ASD patient number is more or less balanced for every site, with the exception of KKI-1 (Kennedy Krieger Institute) that provided two times more controls than ASD patients, and KUL-3 (Katholieke Universiteit Leuven) and NYU-2 (NYU Langone Medical Centre, Sample 2) that only provided ASD cases. For a visual comparison, the number of controls/ASD for each site is displayed on the histogram in figure \ref{fig:controlcase_site}.

\begin{figure}[h]
\centering
\begin{subfigure}{0.9\textwidth}
\includegraphics[width=\textwidth]{abide/controlcase_site.png}
\caption{Histogram of control and ASD subject per site from the whole ABIDE I \& II dataset}
\label{fig:controlcase_site}
\end{subfigure}
\begin{subfigure}{0.9\textwidth}
\includegraphics[width=\textwidth]{abide/mf_site.png}
\caption{Histogram of male and females subjects per site from the whole ABIDE I \& II dataset}
\label{fig:mf_site}
\end{subfigure}
\caption{}
\label{}
\end{figure}


Patients in ABIDE dataset have ages ranging from 4 to $>$ 50 years old, but as shown in the distribution in figure \ref{fig:abideages} the vast majority of participant are younger than 40, precisely $>$ 97 $\%$ of participant are under 40 y.o., also, the majority of sites provide only young patients in a restricted age range, but  as can be seen from figure \ref{fig:abide_age_site} there are some sites that acquired patients with a wide age range (MAX\_MUN or BNI\_1 for example).


Intelligence quotient, whose distribution is shown in figure \ref{fig:abidefiq}, was not provided for every participant, in fact 171 patients out of 2226, coming from sites UM1 and EMC1 (figure \ref{fig:abide_fiq_site}) lack this information.
In our further analysis, before proceding these values were replaced by the average value of all the other provided values.

The lack of a common acquisition protocol is also evident from the eye status at scan feature: as shown in figure \ref{fig:abideeyesite} each site acquired scans either with open eyes or with closed eyes, without a common procedure, and sometimes this information is not even specified.
Overall, the whole dataset consists of more than 70\% of patient acquired with open eyes, as shown in figure \ref{fig:abideeye}.

Considering all the informations above, we can limit our further analysis in order to work on a more homogeneous dataset. In this way, we try to a-priori eliminate some source of variability due to unavoidable difference due to sex or age.
We have then carried out our analysis on a dataset consisting on only male patients with an age within 5 and 40 years. Some further analysis were performed with further cut on eye status at scan, selecting only patients with open eye.




\begin{figure}
\centering
\begin{subfigure}{0.5\textwidth}
\includegraphics[width=\textwidth]{abide/abide_fiq.png}
\caption{\textbf{FIQ} distribution of the whole ABIDE I \& II dataset}
\label{fig:abidefiq}
\end{subfigure}
\hspace{2cm}
\begin{subfigure}{0.9\textwidth}
\includegraphics[width=\textwidth]{abide/fiq_site.png}
\caption{Boxplot of patient's \textbf{FIQ} per site}
\label{fig:abide_fiq_site}
\end{subfigure}
\caption{Histogram distribution and boxplot of FIQ (jointly for Controls and ASD) per site, from ABIDE I + II}
\label{}
\end{figure}






\begin{figure}
\centering
\begin{subfigure}{0.5\textwidth}
\includegraphics[width=\textwidth]{abide/abide_ages.png}
\caption{\textbf{Age} distribution of the whole ABIDE I \& II dataset}
\label{fig:abideages}
\end{subfigure}

\begin{subfigure}{0.9\textwidth}
\includegraphics[width=\textwidth]{abide/abide_age_site.png}
\caption{Boxplot of \textbf{age} of patients per site}
\label{fig:abide_age_site}
\end{subfigure}
\caption{Histogram distribution and boxplot of AGE (jointly for Controls and ASD) per site, from ABIDE I + II}
\label{}
\end{figure}





%eye plots
\begin{figure}
\begin{subfigure}{0.4\linewidth}
\includegraphics[width=\linewidth]{abide/abide_eye.png}
\caption{Overall eye status distribution on the entire ABIDE dataset}
\label{fig:abideeye}
\end{subfigure}
\begin{subfigure}{0.7\linewidth}
\includegraphics[width=\linewidth]{abide/eye_site.png}
\caption{Eye status distribution per site of the entire ABIDE dataset}
\label{fig:abideeyesite}
\end{subfigure}
\hspace{0.1 cm}
\caption{Histograms of patients with open (orange), closed (green) and unknown (blue) eye status in the whole ABIDE I \& II dataset}
\label{}
\end{figure}







\chapter{Image preprocessing}\label{chap:image_preprocessing}
\section{Common preprocessing steps}
\label{sec:preprocessing_steps}

\textcolor{red}{(Da spostare nel capitolo di MRI ?} \textcolor{blue}{ During an MRI and fMRI scan session, data are usually acquired slice by slice, and the thinner is the slice, the more spatial resolution we can accomplish.
But there's a trade-off between spatial and temporal resolution: decreasing the slice's thickness would lead to a better space resolution but at the cost of increasing repetition time to mantain the same Signal to Noise ratio (SNR).
Thie main reason is that signal is proportional to the number of hydrogen nuclei, which is proportional to the slice volume.
In order to obtain a strong BOLD signal, echo time plays a significant role as well: to obtain the maximum strenght signal echo time has to be set $TE \approx T_2^{\ast}$ that means TE around 30 ms.
Images acquired using a shorter echo time have a weaker BOLD signal because of the lack of signal to detect. \cite{jenkinson2018}
}

Data acquired after a MRI and fMRI session are not ready to be analysed to perform functional connectivity measures. In fact they are usually affected by different sources of noise and artifacts.
There are two primary noise sources: one is due to normal biologicals function of patients such as heartbeat and breathing, they are commonly referred as physiological noise. For example, breath rate, can affect BOLD signal because of the induced local motion of the brain's vessels, or the change in blood oxygenation and pressure. The second noise nource is due to movements of the patients during the scan, such as head movements or small body adjustments.
Other common artifacts usually derive from the hardware scan as well.
One of the main artifact intrinsit to the signal nature is distortion and field inhomogeneities, deriving from making the scan sensitive to the BOLD signal which intrinsecally is the detection of a signal loss due to field distortion.
This could be corrected by employing small coils inside the scan to smooth magnetic field differences, this process called shimming still left some artifacts, therefore common data preprocessing pipelines use extra acquisitions to create a field map of the remaining field inhomogeneitie and a shift and stretching voxel to correct for them.\cite{bijsterbosch2017}


In the following lines, we list the most important causes of noise and how they are corrected during common preprocessing steps implemented in the main software of analysis. Subsequently we discuss the implementation of a softwares
\begin{itemize}
\item Head motion is due to the physical movement of the patient inside the scanner. It results in a misalinement from one acquired volume to the next.
To correct effect, \textbf{motion correction} steps are performed at the beginning of image preprocessing. Motion correction works by spatially applying transformations as rotation or translation volume by volume, aiming to overlap every acquired slice to a chosen reference volume, like the first or the middle one.

\item For EPI data, \textbf{slice timing} correction is usually performed as well. It aims to correct artifacts deriving from the sequentiality of acquisition for each slice of the brain, due to which each slice is acquired at different time.
In EPI images, the entire time elapsed to acquire all the slices is called repetition-time, and it usually ranges from 1 to 3 seconds, and during this interval signal may lose its initial strenght.
Slice timing correction uses interpolation in time to shift the BOLD timeseries of each voxel, in order to align them to a reference starting time.
A downside of the use of interpolation, though, is the signal smoothing it performs that can lead to a slight loss of high frequencies information.

\item A further common step is \textbf{spatial smoothing} of both structural and functional data. It operates calculating a weighted average of each pixel over neighbored voxels. To this end, a gaussian kernel with a chosen FWHM is applied to create the weights.
Spatial smoothing is useful to avoid abrupt changes of signal between two neighbouring voxels.

\item \textbf{Band-pass temporal filtering} is commonly applyed to BOLD data, aiming to reduce artifacts from hardware such as an effect called \textquotedblleft drift \textquotedblright namely the slow change of the BOLD baseline signal over time. This is accomplished by applying a high-pass filter to remove very low frequency components. This filtering operation is usually referred as a removal of linear trends, since we are removing from a signal only components with a frequency lower than the low-frequency fluctuations of BOLD signal.
On the other hand, a low pass filter is applied to remove all frequencies above a cut off frequency, it is commonly applied in processing resting state fmri data because the physiological signal is driven by low frequences oscillation while high ones are associated to noise.

\end{itemize}
All the preprocessing steps cited above are performed for each patient, and the main objective of them is to enhance signal to noise ratio of each image. However,
if our task is to perform a group analysis, and compare different patient's images, one of the most relevant step is \textbf{registration}.
Structural (T1-weighted) data are aligned over a standard coordinates system space to universally describe location of the different brain parts, to make sure that the same voxel coordinate corresponds to the same brain area for all the subjects.
Registration occurs for functional images as well: they are firstly aligned and adapted to the structural image and then registered to the same standard template.
The most common template, provided by packages like FSL is Tailarch and Monreal Neurological Institute's MNI-152, adopted by the International Consortium of Brain Mapping (ICBM) as the international standard, replaced the previous Tailarch and Tournoux template.
Because the adoption as international standard it is also called ICBM152 but the most common name remains MNI-152.

To understand how MNI152 differs from the Tailarch and Tournoux we have to spend a few words about the latter.\cite{brett2002}

\noindent\fbox{%
    \parbox{\textwidth}{
    \textbf{From Tailarch-Tournox to MNI152 template}\hfill

    The Teilarch and Tournoux space was published in 1988 and introduced some innovative aspects to tackle the problem of the great variability in brain anatomy between people which so far had limited the accuracy of statistical analysis.
    They introduced a common coordinate system to identify different brain locations, based on some anatomical landmarks, and introduced a spatial transformations to match different brains.
    They choose two anatomical areas as reference areas: the anterior and the posterior commissure, which are relative invariant between different brains and conjuncted them with an axis.
    An horizontal plane passing through this axis was chosen such that it was perpendicular with the interemishpere axis.
    This way they created a 3D coordinate system called the Tailarch coordinate system.
    Afterwards, to match different brains they described a set of spatial transformation, one for each different brain quadrant, to transform a brain in order to match the principal anatomical structures of each other.\cite{brett2002_w}

    From this template, a first MNI template was created, called MNI305, created by manual scaling 241 brains to the Tailarch template and averaging them to obtain  new template, and then transform 305 additional scans to this new template and averaging them to create a second average and final template (MNI305).

    From this template, MNI152 was finally created and published in 2001, by registering 152 T1 scans to the MNI305 template and averaging them.

    }%
}

This template is used both for the structural and the functional registration of a MRI and fMRI scan.
%When we work with functional images though, a second registration step occurs within each patient to align EPI (functional) data to the structural registered image of that subject.

Looking at picture \ref{fig:registrationchart} we can see an example of functional and structural image registration to a MNI152 template: first the functional image is registered to the structural and next, they are both registered to MNI template.

Once the structural and functional images were registered to a standard space, is possible to extract brain region information using an \textbf{atlas}.

Atlases are in the same space as the template image (MNI or Tailarch space for example) and consist of a 3D standard brain template where different brain areas are marked with different color intensities to associate each area to a label.
This division into areas and the subsequent labeling, is called parcellization.

There are different atlases, each one including a different parcelization of the brain, which is obtained by divding it into N labeled ROIs
(Regions Of Interest) to focus on anatomical and/or functional regions, according to the study we are carrying out.
We will describe some og the most important atlases in the following section (\ref{sec:cpac})


%One of the most popular atlas is Harvard Oxford atlas obtained by manually labelling 37 MRI scans, aligning them to the standard MNI template and finally averaging each transformed label \cite{jenkinson2018}.


\begin{figure}
\centering
\includegraphics[width=.3\linewidth]{mri/registration_my.png}
\caption{Registration steps:
 starting from the first row there are the acquired functional and structural images and the MNI template respectively
 The second row shows the functional image registered on the structural space
 The third row shows the final images both registered to the MNI 152 template
}
\label{fig:registrationchart}
\end{figure}



\section{Preprocessing of ABIDE I \& ABIDE II dataset}\label{sec:cpac}

\paragraph{ABIDE-preprocessed initiative}\hfill

In 2013 Neuro Bureau Preprocessing Initiative took care of data preprocessing of ABIDE I dataset and shared its results making them publicly available \cite{cameron2013}.
Thanks to this work, ABIDE I data are available as raw data, as well as preprocessed data.
Four preprocessing software and approaches were employed each from a different group and every one publicly available \footnote{http://preprocessed-connectomes-project.org/abide/C-PAC.html} for download.
The four software for image preoprocessing employed for ABIDE I are:
\begin{itemize}
\item Connectome Computation System (CCS)
\item Configurable Pipeline for the Analysis of Connectomes (C-PAC)
\item Data Preprocess Assistant for Resting-State fMRI (DPARSF)
\item Neuroimaging Analysis Kit (NIAK)
\end{itemize}
The preprocessing steps implemented by the different softwares are similar, they differ on the programming language on which they are rely (Python, MATLAB..) the algorithm implementation, the order of preprossessing steps and their parameters.

But since this work only concerns ABIDE I dataset, and our task is to work with the entire ABIDE I + ABIDE II dataset, we to repeat the preprocessing procedures in order to obtain a dataset including both ABIDE I and II preprocessed with the same pipeline.

\paragraph{C-PAC the Configurable Pipeline for the Analysis of Connectomes}\hfill

In our work, we choose to preprocess data from ABIDE I and ABIDE II using C-PAC: a configurable, open source pipeline, based on Nipype platform.
C-PAC was run on a Docker container installed on a computer and our hardware and software setup consisted on:
\begin{itemize}
\item 16-core Intel i7-5960X processor and 64 Gb RAM
\item Ubuntu 20.04 operating System
\item C-PAC 1.8.1 installed on Docker v. 20.10.11
\end{itemize}

We were allowed to run 3 patients in parallel, reserving 4 cores for each partecipant and up to 12 Gb memory to each patient, necessary to save intermediate-steps outputs.

We choose to employ C-PAC because, basing on machine learning classification results, \cite{yang2020} C-PAC prove to be the most efficient preprocessing pipeline to preprocess ABIDE dataset.
C-PAC employs tools like AFNI, FSL and ANTS to perform image correction of structural MRI and rs f-MRI.
Data were preprocessed following the same steps as the pipeline employed with C-PAC to create the ABIDE-preprocessed dataset.
This pipeline includes both antomical and functional preprocessing.
Anatomical pipeline steps consist on:
\begin{itemize}
\item Skull removal using AFNI's 3dSkullStrip
\item Tissue segmentation using FSL-FAST, to separate gray matter, white matter and CSF, uging a thresholding probability map whose threshold's values were set the same as ABIDE-preprocessed pipeline values
\item Registration to a standard template using ANTS, with a spatial resolution of 2mm
\end{itemize}
Functional pipeline consists on the following steps
\begin{itemize}
\item Slice timing correction using AFNI-3dTshift
\item Motion estimate and correction using AFNI-3dvolreg
\item Distortion correction using PhaseDiff and AFNI 3dQWarp
\item Create a brain-only mask of the functional data using AFNI 3DAUTOMASK
\end{itemize}

\hfill

At the end of functional preprocessing steps, timeseries are extracted from each patient, making use of different atlases such as Automated Anatomical Labeled (AAL), Harvard-Oxford (HO), CC200, CC400, DesikanKilliany etc.
As mentioned in section \ref{sec:preprocessing_steps}, different atlases differ in numbers and types of ROIs. Here we list some examples

The Automated Anatomical Labeled, an atlas created in 2002 \cite{mazoyer-2002} consisted of 90 total ROIs, 45 each emisphere, since then, different modified and improved version were released, the last of which AAL3 consisting of 166 ROIs. The AAL atlas employed in C-PAC is an intermediate version and consists on 116 ROIs\footnote{http://preprocessed-connectomes-project.org/abide/Pipelines.html}.

The Desikan atlas \cite{desikan2006} originally created in 2006 and consisting on 68 regions also underwent different transformation towards a finer parcellization of the brain. C-PAC employs three different Desikan atlases, the one we found more information about is a modified intermediate version named DesikanKilliany consisting on 94 ROIs of which 32 cortical regions each side, 3 ROIs belonging to cerebellar vermis and 29 subcortical regions.

CC200 and CC400 are more recent atlases created by C. Craddock in 2012 \cite{craddock2012} for functional parcellization, consisting of 200 and 392 ROIs respectively \footnote{http://ccraddock.github.io/cluster\_roi/atlases.html}.

However, according to previous studies \cite{spera-2019} the atlas that gave best classification performances was Harvard-Oxford, so this was the atlas we choose to employ for our work.
The H-O atlas consists of a subcortical and a cortical atlas, with a total of 117 ROIs of which 21 subcortical and 96 corticals (48 for each emishpere).
The H-O atlas employed by C-PAC is provided by FSL library \footnote{https://fsl.fmrib.ox.ac.uk/fsl/fslwiki/}, and as includes both subcortical and cortical atlases, even though there's only 111 ROIs, namely 14 subcortical and 97 corticals.
The lacking ROI in the subcortical areas are L/R cerebral white matter, L/R cerebral cortex, L/R lateral ventricle and the brain stem \footnote{Subcortical ROIs https://neurovault.org/images/1700/ \quad Cortical ROIs https://neurovault.org/images/1705/ }.
%In this unified HO atlas (subcortical and cortical) the entire cerebral cortex area was removed because is replaced by the finer parcellization of the cortical areas from the cortical atlas.
Removing these regions from the 117 ROIs we should obtain 110 ROIs, while the atlas employed in C-PAC has 111 regions.
The extra cortical ROI in this H-O atlas is present in the 83th position and is named 3455; however there are no information abut this ROI neither on Harvard-Oxford documentation nor on C-PAC's/FSL's documentation, and because is only made of 2 voxels it was excluded for our further analysis.
In summary we are able to obtain 110 ROIs with Harvard-Oxford atlas, and for each ROI we are able to extract a timeseries using C-PAC.

\begin{figure}[h]
\centering
\includegraphics[width=.4\textwidth]{mri/hoatlas.png}
\caption{Horizontal section of Harvard-Oxford subcortical (sx) and cortical (dx) atlased}
\label{fig:hoatlas}
\end{figure}


As is possible to notice from figure \ref{fig:confrontoabidepreproc} timeseries extracted after our preprocessing pipeline do not exactly match those from ABIDE preprocessed, even if it appears clear that the trend is the same, but there are some local differencies between the two plots.
This is most likely due to two main reasons: the differences in software version, and the lacking of a detailed step-by-step pipeline legend showing the value of all the sub parameters employed during the C-PAC analysis pipeline.
Abide preproccessed data were obtained using one of the first version of C-PAC; nowadays, after more than 7 years, C-PAC and the libraries it relies on were upgraded several times and this may have slightly affected the final outcome of the process.
We found though the old pipeline configuration file employed on abide preprocessed \footnote{https://github.com/preprocessed-connectomes-project/abide}, but since it belongs to a previous version of C-PAC, it lacked lots of sub-parameters and sub-settings added during these years.
For this reason, in our pipeline configuration file, parameters in common between our file and abide preprocessed' were set the same as abide's, and for all the others we choose to left the C-PAC's default values.

\begin{figure}[h]
\centering
\includegraphics[width=.8\textwidth]{abide/timeseries51456.png}
\caption{Comparison of 4 timeseries between ABIDE-preprocessed and our timeseries. Data show 4 randomly chosen ROIs: ROI 0, 6, 50, 108, belonging to patient 51456 from ABIDE I. Corresponding to ROI 10, 26, 1901, 4702}
\label{fig:confrontoabidepreproc}
\end{figure}




\chapter{Connectivity measures}\label{chap:connectivity_coefficients}
As described in chapter \ref{sec:preprocessing_steps} if we choose an atlas on fMRI data, for brain parcellization, we can extract timeseries of different brain areas. If we use Harvard-Oxford atlas, we are able to extract 110 timeseries for each patient.
With these data, we want to construct a correlation matrix for each patient, by pairwise comparing timeseries and extracting a coefficient representative of functional correlation between two areas.

The total number of combination achievable with n timeseries is given by
\begin{equation}
N_{comb} = \frac{n\cdot(n-1)}{2}
\end{equation}
Therefore, employing HO atlas with 110 ROIs we obtain 5995 combination each one expressed by a correlation coefficient.

In the following section we discuss two different approaches for computing correlation coefficients: the first makes use of Pearson correlation analysis, a useful tool to quantify the linear correlation between two timeseries, and the second approach, based on wavelet analysis, performs a comparison between two signals in time-frequency domain.
We start discussing Pearson correlation coefficients in the following section and in the next section we describe the main traits of wavelet analysis and how we extracted a correlation coefficients from a time-frequency analysis of two timeseries.

\section{Pearson correlation and Z-Fisher transform}
Pearson correlation often simply called correlation coefficient is the measure of a linear relation lying between two sets of data $x_1$ and $x_2$, is defined as the covariance of (x1, x2) over the product of the standard deviation of the two sets.
Pearson correlation coefficients have the important property to be scale and magnitude invariant, since each timepoint is shifted by the average value of the timeseries and divided by its standard deviation.

\begin{equation}
Corr(x, y) = r_{xy}=  \frac{Cov\left( x, y\right)}{\sigma_x \sigma_y} = \frac{E\left[ \left( x - \mu_x \right) \left( y - \mu_y \right)\right]}{\sigma_{x} \sigma_{y}}
\end{equation}
This correlation coefficient assumes values between -1 and +1 where the extremes correspond to exact anti-correlation or correlation respectively, so that, if a linear relation lies between the two sets, a high absolute value indicates that the two series tend to be simultaneously greater or lower than their respective means. \cite{baldini2021}

Given two series x and y, of length n correlation can be easily computed by
\begin{equation}
r_{xy} = \frac{\sum_{i = 1}^n \left( x_i - \mu_{x_i}\right)\left( y_i - \mu_{y_i}\right)}{\sqrt{\sum_{i = 1}^n \left( x_i -\mu_{x_i}\right)^2}\sqrt{\sum_{i = 1}^n \left( y_i -\mu_{y_i}\right)^2}}
\end{equation}

For each patient Pearson correlation coefficient was computed for all timeseries pairs. Before further analysis, a common way to proceed is to transform each coefficient with Fisher z-transformation.\cite{spera-2019}
The reason behind this common operation appears clear when we are dealing with high correlate variables, when Pearson correlation distribution results in an highly skewed distribution, Fisher's transform sought to transform it into a normal distribution of which the standard error is approximately constant equal to $\sigma = \/\sqrt{N-3}$ where N is the total number of points, and it does not depends on the values of correlation.\cite{wicklin2017}

With this property, Fisher transform is also important to test some hypotesis about correlations, we can run the test with the transformed variables which are normal distribuited with a known variance.

Thus, this transformation allows us to obtain a variable which is normally distribuited even when Pearson correlation coefficients follow a bivariate normal distribution or they are leaning toward the extremes.
In our data, this difference in distributions is not pronounced, one example where this difference is evident is shown in figure \ref{fig:pearson_zscore_distribution_3}, but since we are not working with highly correlated or uncorrelated variables, there are many other examples where pearson correlations already follow a gaussian distribution and there is not much difference with z-score values distribution.
In any case, we are dealing with a bivariate distribution, and then performing Fisher transformation is a common practise to get a dataset more normally distributed.

\begin{equation}
z = \frac{1}{2}ln\left(\frac{1+r}{1-r}\right) = arctanh(r)
\end{equation}

At the end of this analysis, we obtain correlation matrices like the one shown as an example in fig \ref{fig:corrmatrices}, referred to patient 20243 from ABIDE I dataset.
\begin{figure}
\begin{subfigure}{0.5\textwidth}
\includegraphics[width=\textwidth]{corrmatrix.png}
\caption{}
\label{ref:corrmatrix}
\end{subfigure}
\begin{subfigure}{0.5\textwidth}
\includegraphics[width=\textwidth]{zscorematrix.png}
\end{subfigure}
\caption{Correlation and z-values matrix computed from timeseries extracted using Harvard-Oxford atlas, from patient 20243 belonging to ABIDE I dataset. \textbf{Note:} The two figures are not on the same values scale because Pearson correlation coefficients have range [-1, 1], and z-scored coefficients have range ($-\infty,\infty$)}
\label{fig:corrmatrices}
\end{figure}


\begin{figure}[h!]
\centering
\includegraphics[width=.7\textwidth]{pearson/pearson_zscore_feature_3}
\caption{Distribution of feature 3: Pearson correlation values in blue and Fisher transformed values in light orange}
\label{fig:pearson_zscore_distribution_3}
\end{figure}




\section{Wavelet analysis}
A different approach to compute a correlation coefficient is by making use of wavelet analysis.\cite{ferrante-2015} \cite{vandenberg-1999}
wavelet is a mathematical tool for analyzing time series or images, and provides a comprehensive way for investigating the bivariate relationship between timeseries both in time (or space) and frequency domain.

To understand wavelet transformation it could be helpful to compare it with Fourier analysis.
Fourier analysis allows us to expand a periodic function f(t) into a series, ideally infinite sums of weighted sines and cosines with different frequencies
\begin{equation}\label{eq:fourier_series}
f(t) = a_0 + \sum_{k = 1}^\infty (a_k cos(2\pi k t/T) + b_k sin (2\pi k t /T))
\end{equation}
In equation \ref{eq:fourier_series} terms corresponding to the k-th frequency are called harmonics and are multiples of the fundamental frequency corresponding to k = 1.

The Fourier Transform (FT) is a natural extension of the Fourier series to aperiodic functions defined over the real axis.
Aperiodic functions don't allow a discrete superposition of sines and cosines terms, and for this reason they need to be represented by a continue superposition.
Fourier transform takes a function from time or space domain and turns it into spatial or temporal frequency domain.
It is a complex function since sines and cosines terms can be represented by a complex exponential as shown below in equation \ref{eq:ctf}.
\begin{equation}\label{eq:cft}
\tilde{F}(\omega) = \int\limits_{-\infty}^{+\infty} f(t)e^{-i \omega t} dx
\end{equation}
To deal with discrete signal, for example a sequence $x_n$ obtained from a discrete sampling of a continuous signal x(t), is possible to make use of the Discrete Time Fourier Transform, wich allows us to write the $\tilde F(\omega)$ as in equation \ref{eq:dft} which represent the discrete version of \ref{eq:cft}.

The Discrete Fourier Transform, is useful to analyse discrete periodic signal, it acts on a function defined over a finite domain and returns a sequence of samples of the discrete fourier transform we can denote as $\tilde{x_k}$

\begin{equation}\label{eq:dft}
 \tilde{F}(k) = \tilde{x}_k = \sum_{n = 0}^{N-1}x_n e^{\frac{-2\pi i k n}{N}}
\end{equation}
where N is the total number of points of the timeseries x(t).
The frequencies at which samplings are computed are $\omega_k = 2\pi k/N$.
The series we obtain from relation \ref{eq:dft}: $\tilde{x}_k$ is periodic and its period is equal to N.

One of the main drawback of FT though, is the lacking of spatial information, for example for a non stationary signal, is possible that the signal contains a variable component of frequency in different space or time locations.

A time-frequency analysis provides for this lacking computing both frequency and spatial information from a signal, allowing us to analyse non stationary signals and obtain informations both on time (or space) and frequency domain.
Since we are working on signal defined over a time domain, from now on we would refer only to  \textquotedblleft time \textquotedblright  and  \textquotedblleft frequency \textquotedblright domain, but we keep in mind that all the relation are true for spatial and spatial frequency domain as well.

The process through which we exctract informations about frequencies (F) and time location (T) is a convolution of the signal x(t) with a function w(t) called windowing function usually centered at zero which rapidly decays to zero, often symmetrically.
\begin{equation}
\tilde{x}(F, T) = \int_{-\infty}^{\infty} x(t)w(t-T) e^{-2\pi i F t} dt
\end{equation}
The convolution term $w(t-T) e^{-2\pi i F t}$ changes according to the type of analysis we are performing. Parameters F and T are often simply called a and b, where b is a time translation parameter,and acts by simply shifting our convolution function throughout the entire timeseries x(t). It acts the the same way through all the type of analysis.
What differenciate the most the different analysis isthe parameter \textquotedblleft a \textquotedblright.
It is the frequency parameter and the way it is introduced in a convolution determines the developing of the transform.
The two main types of time-frequency analysis are

\begin{enumerate}
\item Windowed Fourier Transform (WFT): $\psi_{ab}(t) = e^{it/a}\phi(t-b)$ where $\phi(t)$ is a window function of constant width and the parameter a acts as frequency modulation (freq $\sim$ 1/a): the lower is a, the greater the number of oscillation inside the window $\phi(t)$.
\item Wavelet Transform (WT): $\psi_{a, b}(t) = \frac{1}{\sqrt{a}}\psi(\frac{t-b}{a})$ where $\psi$ is a function called \emph{mother wavelet}; a is a positive real number and defines the dimension of $\psi$: with a$>$1 we obtain a dilation and a$<$1 corresponds to a contraction. b is any real number (positives and negavies) and defines the location of the wavelet in time.
\end{enumerate}

In Wavelet Transform, the equivalent to the window function in WFT is a wavelet function $\psi$.

\paragraph{Wavelet} \hfill

A wavelet, literally \textquotedblleft small wave \textquotedblright is a wave function limited both in space and period: begins at zero, grows and decrease in a limited time period and returns to zero. With these properties, is a function local in the temporal domain as well as in the frequency domain.

To be defined as so, a wavelet is required to satisfy two properties: have zero mean (it must be oscillating) and unitary squared norm \cite{percival-2013}.

\begin{equation}\label{eq:waveletproperties}
\int_{-\infty}^{\infty}\psi(x)dx = 0 \qquad \qquad  \int_{-\infty}^{\infty} |\psi(x)|^2 dx = 1
\end{equation}

It is defined over the space $L^2$ of Lebesgue measurable functions that are both absolutely integrable and square integrable. In this space the two properties can be satisfied.
%This kind of transformation of going in a higher dimensional space is a redundant operation

%modificatoo
There are different wavelets that satisfy properties \ref{eq:waveletproperties}, such as Haar (fig \ref{fig:wavelet_haar}), Meyer (fig \ref{fig:wavelet_meyer}) \cite{dabauchies-1992}, Mexican hat (fig \ref{fig:wavelet_mexicanhat}) or Morlet (fig \ref{fig:wavelet_morlet}) \footnote{https://it.mathworks.com/help/wavelet/gs/introduction-to-the-wavelet-families.html}.



\begin{figure}
\begin{subfigure}{0.3\textwidth}
\includegraphics[width=\textwidth]{wavelet/haar.png}
\caption{}
\label{fig:wavelet_haar}
\end{subfigure}
\begin{subfigure}{0.3\textwidth}
\includegraphics[width=\textwidth]{wavelet/mexicanhat.png}
\caption{}
\label{fig:wavelet_mexicanhat}
\end{subfigure}
\begin{subfigure}{0.3\textwidth}
\includegraphics[width=\textwidth]{wavelet/meyer.png}
\caption{}
\label{fig:wavelet_meyer}
\end{subfigure}
\caption{Visual comparison between three different types of mother wavelets, Haar wavelet (fig \ref{fig:wavelet_haar}) is a simple square wave function, mexican hat (fig \ref{fig:wavelet_mexicanhat}) is a wavelet belonging to gaussian function family and represents the negative normalized second derivative of a gaussian function; and Meyer wavelet (fig \ref{fig:wavelet_meyer}) a wavelet with applications in fields like adaptive filters}
\label{fig:different_wavelets}
\end{figure}




What we are going to use in our analysis is Morlet wavelet, defined by equation \ref{eq:morletwavelet} and shown in figure \ref{fig:wavelet_morlet}.
For each wavelet, there's a trade-off between time ans frequency resolution: compressing the wavelet in a shorten time domain improves time resolution vut at the cost of frequency resolution and vice vice-versa.
With this wavelet this trade-off between time and frequency resolution can be controlled by the choice of $\omega_0$.\cite{muller-2004}.
We choose to run our analysis with a value of $\omega_0 = 6$ since it provides a good balance between the two resolution \cite{grinsted-2004} and besides, this value gives the best ratio between Fourier Period and the scale parameter a during a Transform, equal to $\lambda = 1.03$.
This way results in frequency domain are more interpretable since the scale parameters a used for the Transform is almost equal to the Fourier period.

%omega 0 = 6: \footnote{https://it.mathworks.com/matlabcentral/answers/824015-understanding-cwt-morlet-time-and-frequency-resolution?s_tid=srchtitle}

\begin{equation}\label{eq:morletwavelet}
\psi\left(t\right) = \pi^{-1/4}e^{iw_0 t}e^{-t^2/2}
\end{equation}

\begin{figure}[h]
\centering
\includegraphics[width=.4\linewidth]{wavelet/morlet}
\caption{Morlet wavelet}
\label{fig:wavelet_morlet}
\end{figure}

Two main type of analysis can be performed using wavelets: continuous wavelet transform (CWT) and discrete wavelet transform (DWT) and the CWT is what we employed for our analysis.

Continuous wavelet transform decomposes a time series in time-frequency domain by successively convolving the timeseries with several scaled and translated version of the mother wavelet $\psi$: $\psi_{a, b}\left(t\right) = \frac{1}{\sqrt{a}}\psi(\frac{t-b}{a})$.
If the timeseries is described by a function f, assumed to be real in the equation below, the convolution of f and $\psi_{a, b}\left(t\right)$ is

\begin{equation}
W_{a,b}(f) = \int_{-\infty}^{+\infty}  f(t) \cdot \psi_{a, b}^\ast \left(t\right)   dt
\end{equation}
Where the $\ast$ indicates the complex conjugate.
This integral is performed for different values of a and b.
It can be useful to visualize this integral in a dynamic way: set a value for $a$, a wavelet centred in $b$, is slided across the signal by changing the value of $b$ and for each of these values, a coefficient is extracted by integrating the product between the wavelet and the signal; in this way, coefficients are function of frequency (or scale) and time. This operation is repeated for different values of a and b, and allows us to assemble a matrix called \emph{scalogram}, which is basically a plot of these coefficients in time-frequency domain.

Formally, a wavelet transform, can be described as a mapping from $L^2(\Re) \rightarrow L^2(\Re^2)$.

To computationally implement CWT, a discretized version was implemented on MATLAB tools Wavelet toolbox, but it does not be confused with the Discrete wavelet transform \footnote{https://it.mathworks.com/help/wavelet/gs/continuous-and-discrete-wavelet-transforms.html}, a similar type of analysis but with some important differences.
The difference between the continuous wavelet transform implemented on MATLAB and discrete wavelet transform lies in how finely stretching and shifting parameters are sampled: the CWT discretizes scale more finely than the discrete wavelet transform.

In the CWT, parameters are discretized based on a fractional power of two, by setting a = $2^{j/\nu}$  (\footnote{This way, the discretized wavelet can be written as $\frac{1}{2^{j/\nu}} \psi\left ( \frac{t-b}{2^{j/\nu}}\right )$ }) \cite{liu-1994} \cite{dabauchies-1992} where $\nu $, j are integers.
The parameter $\nu$ is often referred to as the number of “voices per octave” because increasing the scale by an octave (namely to double the frequency) requires $\nu$ intermediate steps, for example from $f = f_0 2^{\nu/\nu} \ to \ f = f_0 2^{2\nu/\nu}$ , $\nu$ steps are required.
The larger the value of $\nu$, the finer the discretization of the scale parameter.
In the DWT, the scale parameter is always discretized to integer powers of 2, $2^j$, j=1,2,3,..., so that the number of voices per octave is always 1.
Since it employs a more rough discretization, DTW is usually used for denoising and compression of signals and images.

In our analysis we are going to use DTW implemented in matlab with the default parameters of 12 voices per octave.

%Using CWT is possible to obtain singal's information on a matrix: a time-frequency plot also called scalogram by convolving the signal, using different wavelets parameters a and b from a grid

One problem that arises from working on a timeseries, which is basically a signal with a finite support, is that when a wavelet is located at the beginning of at the end of the signal, the wavelet extends itself outside the boundary of the signal, and the convolution would require nonexistent values beyond the boundary.

To overcome this problem it would be possible to accept this data information loss and truncate the values beyond boundaries; whereas an other approach could be to artificially extend data using methods such as zero padding which assumes that the signal is zero outside its original support, or symmetrization which extend the signal symmetrically outside the boundaries or smooth padding which recovers a signal by extrapolating values from the first derivative values or from the signal itself.
Symmetrization method is the one employed for the subsequent analysis.
The confidence value obtained on the boundaries, though, is lower than other obtained for central values of time location.
Areas of the scalogram affected by these edge effects are indicated as \textquotedblleft outside the Cone of influence (COI) \textquotedblright.

Figure \ref{fig:cwt_timeseries} shows the timeserie belonging to ROI 2201: Right Angular Gyrus of patient 51056 from ABIDE I, and its corresponding Continuous Wavelet Transform.
x axis corresponds to time points and on the y axis frequencies or periods (derived from the parameter a) are represented. Just as an example, we choose to display the y axis as periods.
The COI is shown as a dotted white line and values outside the COI, where edge effect becomes effective are shown with a lighter faded.
The color is a visual representation of the magnitude of each wavelet coefficient.

\begin{figure}
\begin{subfigure}{0.5\textwidth}
\includegraphics[width=\textwidth]{wavelet/timeseries.png}
\caption{}
\label{fig:timeseries_cwt}
\end{subfigure}
\begin{subfigure}{0.5\textwidth}
\includegraphics[width=\textwidth]{wavelet/cwt_timeseries.png}
\caption{}
\label{fig:cwt}
\end{subfigure}
\caption{Timeseries (fig \ref{fig:timeseries_cwt}) of ROI 2201: Right Angular Gyrus of patient 51056 from ABIDE I, and its corresponding Continuous Wavelet Transform  (fig \ref{fig:cwt}).
x axis corresponds to time points and y axis the periods derived from the scale parameter $a$ of the wavelet convolving function.
The COI is shown as a dotted white line and values outside the COI, are shown with a lighter faded.
The color represents the magnitude of each wavelet coefficient.}
\label{fig:cwt_timeseries}
\end{figure}



\paragraph{Wavelet coherence} \hfill \newline

Continuous wavelet transform can only be used to analyse one signal at a time; if our goal is to analize and compare two signals using wavelet transform, the analysis to perform is called wavelet coherence.

From two Continuous Wavelet Transforms is possible to compute the Cross Wavelet Transform (XWT) which allow us to examine their cross-wavelet power and relative phases.

Using XWT is then possible to compute the Wavelet Coherence.

Denoting as $W^{X}(a, b)$ the continuous wavelet transform of a signal X, is defined the \emph{wavelet power spectrum} of a signal X(n) as
\begin{equation}
W^{XX}(a, b) = W^X(a, b) \left[W^X(a, b)\right]^{\ast}
\end{equation}
where the $^{\ast}$ represent the conjugate transpose.
Similarly, the \emph{wavelet cross spectrum} of two time series X and Y is defined as
\begin{equation}\label{eq:cross_wavelet_spectrum}
W^{XY}(a, b) = W^X(a, b)[W^Y(a, b)]^\ast
\end{equation}
whose module $|W^{XY}(a, b)|$ represents the amount of joint power between the two time series, and is called \emph{cross wavelet power}.
From the cross wavelet spectrum (eq \ref{eq:cross_wavelet_spectrum}), is possible to compute the the complex argument
\begin{equation}\label{eq:wavelet_phase}
\Delta \phi (a, b) = arctan\left(\frac{Im\left[ W^{XY}(a, b)\right]}{Re\left[ W^{XY}(a, b)\right]} \right)
\end{equation}
which represents the relative phase between X and Y, for each given value of the parameters $a$ and $b$, and is defined over the interval $[-\pi, \pi]$.

The wavelet coherence $R^2(a, b)$ is finally defined as reported in equation \ref{eq:WC}. This coefficient ranges in the interval (0, 1) and represents the localized correlation coefficient between X and Y in the time-frequency domain.
\begin{equation}\ref{eq:wcoherence}
R^2(a, b) = \frac{|S(W^{XY}(a,b))|^2}{S(|W^X(a,b)|^2) S(|W^Y(a,b)|^2)}
\label{eq:WC}
\end{equation}

In equation \ref{eq:wcoherence} S is a smoothing operator both in frequency and time defined as \cite{torrence-1999} \cite{grinsted-2004}
\[
S = S_{scale}S_{time}(W) \quad S_{time} = W\cdot c_1 ^{\frac{-t^2}{2a^2}} \quad S_{scale}(W) = W\cdot c_2 \Pi(0.6)
\]
where $c_1, c_2$ are normalization constants, $\Pi$ is a boxcar (rectangle) function and 0.6 is an empirically determined factor for Morlet Wavelet \cite{torrence-1998}.

An example of wavelet coherence scalogram is shown in figure \ref{fig:wcoherence_scalogram}. It represents the wavelet coherence computed from two timeseries (fig \ref{fig:wcoherence_timeseries})of patient 51056: ROI 57 corresponds to Right Angular Gyrus and ROI 65 to Right lateral occipital cortex.
The color represents the magniture of the cross-power, and phase information is represented as oriented arrows, pointing towards an imaginary 360 degrees circle, where the zero phase shift is represented by an arrow pointing right and a 180 degrees, by a left arrow.



\begin{figure}
\begin{subfigure}{0.5\textwidth}
\includegraphics[width=\textwidth]{wavelet/timeseries_for_wcoh.png}
\caption{}
\label{fig:wcoherence_timeseries}
\end{subfigure}
\begin{subfigure}{0.5\textwidth}
\includegraphics[width=\textwidth]{wavelet/wcoherence.png}
\caption{}
\label{fig:wcoherence_scalogram}
\end{subfigure}
\caption{Plot of two timeseries from patient 51056 in figure \ref{fig:wcoherence_timeseries} and their relative wavelet Coherence scalogram.
On the x axis the timepoints and on the y axis the frequencies of this time-frequency decomposition.
The color represents the magnitude of the cross-power coefficients and the relative phase shift is represented as arrows pointing to the right is phase shift is zero ad to the left is is 180 degrees. Starting from 0 degrees arrows rotate counterclockwise.
}
\label{fig:wcoherence}
\end{figure}



Our goal is to extract from this wavelet coherence matrix, a single value, interpretable as a correlation coefficient, just like we did with Pearson coefficients.
We can accomplish so, by extracting two types of information from this scalogram: the magnitude of the correlation and relative phase between the two signal.

We can extract the magnitude information of each entry of the scalogram, but before doing so, we need to assess the level of significance of this coherence matrix over the noise, this way we can estimate the statistical significance level of our values, and only collect the significative ones.

The theoretical procedure \cite{grinsted-2004} \cite{bernas-2018} is to generate, using Monte Carlo methods, a large (1000) samples of wavelet coherence matrices using red noise timeseries. \cite{hartmann-2014}
These red noise samples should be generated, for each time series, with the same 1-lag autoregressive coefficients (AR1 coefficient) of the two timeseries under examination.
Then, for each pair of red-noise timeseries, we should compute wavelet coherence matrix.
However, since AR1 coefficients have little impact on the significance level \cite{grinsted-2004}, we choose to generate a single sample of 1000 pairs of red noise and for each pair we computed the wavelet coherence matrix.

For each pair of noise, the corresponding wavelet coherence matrix from equation \ref{eq:wavelet_phase} is stacked to obtain a $A\times B \times x \times 1000$ 3D noise matrix to extract the distribution of the entries.
This null distribution of wavelet coherence coefficient is used to estimate the threshold to 5\% significance level for the subsequent analysis.
We refer to the value corresponding to this 95-th percentile as $a_{95}$.


As a second step we extract the relative phase information of each entry of the matrix, and combining these two informations together, we can estimate the time of in-phase (or out of phase) coherence which can be seen as the percentage of time synchronicity (or antisynchronicity) between the two timeseries. \cite{bernas-2018}
This time of in-phase coefficient is defined as:

\begin{equation} \label{eq:wavelet_inphase}
c_{ij} = \frac{100}{N}\sum_{a, b}^N I\left\{ R_{ij}^2(a,b) > a_{95}\right\}\cdot I\left\{-\frac{\pi}{4}<arg(W^{XY}(a, b))_{ij} < \frac{\pi}{4}  \right\}
\end{equation}

Where the indices i and j refers to the two timeseries i and j; N is the total number of points inside the cone of influence (COI) $I\left\{ ...\right\}$ is either 0 or 1 depending on if the condition inside is satisfied; $a_{95}$ is the threshold value above which the computed wavelet coherence coefficient is regarded as significative.

Similarily to this coefficient is the time of counter-phase coefficient, which can be obtained by modifying from the formula \ref{eq:wavelet_inphase} the phase condition into

\[
I\left\{arg(W^{XY}(a, b))_{ij} < -\frac{3\pi}{4} \ \lor \ arg (W^{XY}(a, b))_{ij} >\frac{3\pi}{4} \right\}
\]

In short with this analysis, we are just considering coefficients with an high significance level (above 95\%) and with a small $\in [-\pi/4, \pi/4]$, or big phase shift ( $< -\frac{3\pi}{4}  \lor  >\frac{3 \pi}{4}$).

Wavelet coherence maps were calculated using MATLAB's wavelet Toolbox, which employs the Morlet wavelet, for the wavelet decomposition, as  mentioned before, and decomposes the frequency range using 12 subscales per octave and 9 octave.

%Figure \ref{fig:wcoherence} represents the wavelet coherence scalogram obtained from two ROIs of patient 51056.

From equation \ref{eq:wavelet_inphase} both coefficients in-phase and anti-phase matrix coefficients were computed.
The correlation coefficients matrix for each subject, were then flattened and used as input to the neural networks.
An example of correlation matrix created with in-phase and out-of-phase coefficients from data of patient 51056 is shown in figure \ref{fig:win_wout}



\begin{figure}
\centering
\begin{subfigure}{0.45\textwidth}
\includegraphics[width=\textwidth]{wavelet/win_corrmatrix.png}
\caption{Coefficients of In-phase percentage}
\label{}
\end{subfigure}
\begin{subfigure}{0.45\textwidth}
\includegraphics[width=\textwidth]{wavelet/wout_corrmatrix.png}
\caption{Coefficients of Off-phase percentage}
\label{}
\end{subfigure}
\caption{Correlation matrix created with wavelet coefficients of in-phase and off-phase percentage. Patient 51056, ABIDE I }
\label{fig:win_wout}
\end{figure}



\newpage


\chapter{Multicenter data harmonization}\label{chap:harmonization_theory}
Since nw, a big percentage of neuroimaging studies have been carried out with limited data acquired from a single centre, to minimize varibility in data due to the different instrumentations employed.
In recent years, however, there's the tendency to create shared datasets, by pooling together data acquired from different centers.
A positive side of creating a large dataset putting together data from multiple centers is the possibility to work on dataset of bigger dimension, resulting in obtaining more statistically accurate analysis, and facilitates the generalization and the robustness of the model.
As a drowback, however, it introduces variability in our analysis such as differences due to scanner models, acquisition parameters, generally known as scanner effects.
A level out procedure becomes necessary to avoid the model to learn these differences deriving from inomogenity of data because of differences in data acquisition procedures.
This procedure is generally referred as harmonization.
Harmonization was tested on different datasets and has proven to be an effective way to reduce scanner effects  in different kind of datasets such as genes' microarray data, diffusion tensor imaging or structural MRI. \cite{johnson-2006} \cite{fortin-2017} \cite{lombardi2020}.


\section{Harmonization - theory}\label{sec:harmonizationtheory}


The state-of-art procedure used in genomic is called ComBat (named after Combating Batch effects), used for structural MRI but aplicable to any kind of imaging data, is used to mitigate scanner effects.
It is based on a previous method initially proposed in 2007, for gene expression studies to compute batch effect corrections, later implemented by Fortin et al \cite{fortin-2018} for the harmonization of cortical MRI volumes, and finally in its current improved version, developed by Pomponio et al. \cite{pomponio-2019} in 2019.
In this paragraph we are going to explain how ComBat technique works and after that, we'll see how it was modified and improved in the implementation made by Pomponio et al.

ComBat technique belongs to the family of Location and Scale adjustment methods, but it represents an improved version of them: since it models site-specific scaling factors and uses empirical Bayes estimate to improve the estimation of the site parameters for small-sized datasets.\cite{fortin-2018}
It aims to reduce inter-site variability while preserving biological variability such as differences due to sex, age, FIQ (Full intellective quotient), ICV (Intra Cranical Volume) and so on.
The model assumes that a feature can be modeled as a linear combination of the biological variables plus the site effect, and the errors introduced with site effect can be modeled as both a multiplicative and an additive term and that it can be standardized by adjusting the mean and the variance across the batches.

Let $y_{ijf}$ be the numeric value of the feature f for the patient i acquired with the scan (or equivalently, from the site) j, so that i = 1, 2, ..., K indexes the scanner, j = 1, 2, ..., $N_i$ indexes the total number of subjects acquired for each scanner i, and f ranges from 1 to F being F the total number of features. Besides, let's assume that this value y, can be written as
\begin{equation}
y_{ijf} = \alpha_f + x_{ij}^T \beta _f + \gamma_{if} + \delta_{if} \epsilon_{ijf}
\end{equation}
where:
\begin{itemize}
\item $\alpha_f$ is the mean value for the feature f,
\item $x_{ij}$ is the entry of the matrix X created with the covariates of interest such as age, sex,
\item $\beta_f$ is the vector of regression coefficients corresponding to X for the feature f,
\item $\gamma_{if}$ and $\delta_{if}$ represent the additive and multiplicative terms for site-i effects related to feature f respectively,
\item $\epsilon_{ijf}$ is a error term which is assumed to follow a normal distribution with zero mean and variance $\sigma_f^2$.
\end{itemize}

The final location-and-scale-adjusted data $y^{\ast}_{ijf}$ are given by
\begin{equation}\label{eq:harmonized_data}
y^{\ast}_{ijf} = \frac{y_{ijf} - \hat \alpha_f - X\cdot \hat \beta_f - \hat \gamma_{jf}}{\hat \delta_{jf}} + \hat \alpha_f + X \cdot \hat \beta_{jf}
\end{equation}
where $\hat \alpha_f , \hat \beta_f , \hat \gamma_f ,  \hat \delta_{jf}$ are estimators of the corresponding parameters, based on the model.

One of the most disadvantages of using a simple location and scale batch adjustment is that it requires a large batch size for the implementation because is not robust to outliers in small sample sizes.

ComBat uses empirical Bayes to provide a more robust adjustment for the parameters $\hat \gamma_{if} \ \hat \delta^2_{if}$, making the model able to deal with small-sized dataset as well.

%Specifically it is assumed that that the site-effect parameters follow the normal distribution and the inverse gamma distribution respectively
%\begin{equation}
%\gamma_{if} \sim N(Y_i, \tau^2_i) \qquad \delta^2_{if} \sim \frac{\beta^{\alpha}}{\Gamma (\alpha)}(1/x)^{\alpha +1}\cdot e^{-b/x}
%\end{equation}

The process ComBat employs to estimate feature-dependent parameters, and to adjust data for batch effect can be summarized in 3 steps, at the end oh which we reach obtain the results shown in equation \ref{eq:harmonized_data}

\begin{enumerate}
\item \textbf{Standardization}: Data are standardized feature-wise, so that every feature has similar overall mean and variance.
least square regression, is then performed to determine parameters $\hat \alpha_f, \hat \beta_f, \hat \gamma_{jf}$ and subsequently, $\hat \sigma^2_f = \frac{1}{n} \sum_{ji} \left ( y_{ijf} - \hat \alpha_f - X \hat \beta_f - \hat \gamma_jf \right )^2$ being n the total number of patients.


The standardized data are then calculated by equation \ref{eq:harmonization_std_data}

\begin{equation}\label{eq:harmonization_std_data}
Z_{ijf} = \frac{y_{ijf}-\hat \alpha_f - X \hat \beta_f}{\hat \sigma_f}
\end{equation}


\item \textbf{Empirical Batch parameters estimate}:
%Site effect parameters estimate using parametric empirical priors:
We assume that standardized data follow a normal distribution $Z_{ijf} \sim N(\gamma_{jf}, \delta^2_{jf})$ and we seek for a proper estimation of parameters $\gamma_{jf}, \delta^2_{jf}$.
it is also assumed that these site-effect parameters follow the normal distribution and the inverse gamma distribution respectively
\[
\gamma_{jf} \sim N(\eta_j, \tau^2_j)  \qquad \delta^2_{jf} \sim \text{Inverse Gamma}(\lambda_j, \theta_j) =  \frac{\theta^{\lambda}}{\Gamma (\lambda)}(1/x)^{\lambda +1}\cdot e^{-\theta/x}
\]

And these hyperparameters $\eta_j, \tau^2_j, \lambda_j, \theta_j$ are empirically estimated from standardized data  $Z_{ijf}$ by using the method of moments.
%Breve box to exmplain method of moments??
we then obtain improved estimation of parameters $\gamma^\ast_{jf} \ \text{and} \ \delta^\ast_{jf}$ and we use them for the third and last step, where we adjust our data using them.

\item \textbf{Adjust the data}: After calculated the site-effect parameters we are finaly able to adjust our initial data using the relation

\begin{equation}
y^{ComBat}_{ijf} = \frac{\hat \sigma_f}{\delta^\ast_{jf}}(Z_{ijf} - \gamma^\ast_{jf}) + \hat \alpha_f + X \hat \beta_f
\end{equation}
which is just an equivalent way to write equation \ref{eq:harmonized_data}, using parameters estimated in the previous passage.
\end{enumerate}


So far, this was the main idea behind ComBat technique, but we said before that the last implementation of this technique by Pomponio et al. \cite{pomponio-2019} is an improved version of it.
They differn in the modelling of the biologically covariates, which in the formulas above are expressed by the terms $\hat \alpha_f + X \hat \beta_f$ which is a simple linear model.
Pomponio substituted this linear model with a Generalized Additive Model (GAM). In this model covariates such as sex, age, FIQ, are represented by terms $x_{ij}, \ z_{ij}, \ w_{ij}$ which allow a better parametric modelling to deal with non-linear trends such as the trend of cortical thickness in relation to age.
This way, the terms $\hat \alpha_f + X \hat \beta_f$ in equation \ref{eq:harmonized_data} are substituted by a linear combination of function F of these covariates
\begin{equation}
\hat \alpha_f + X \hat \beta_f \ \longrightarrow \ F_f(x_{ij}, z_{ij}, w_{if}...) = a_f + g_f(x_{ij}) + h_f(z_{ij}) + p_f(w_{ij}) + ...
\end{equation}
Where functions $g_f, h_f, p_f$ can be either linear or non-linear functions of our covariates, according to how we want to model these covariates, or in other words, according to wether we want to specify some kind of non-linear relationship between covariates and our data.





\chapter{Domain-adversarial NN}\label{chap:domain_adversarial_theory}
%\textcolor{red}{Da rivedere tutta questa parte}
In this section we discuss a different and innovative approach to deal with multi-center datasets.
So far we presented one important harmonization procedure that removes site-related features in an analytical way; an other possible approach is to make use of a deep neural network specially designed to perform a classification unbiased by site-related features.
The construction of this network gets its main idea on the netwrok proposed by Ganin, Ustinova et al, in their article \cite{ganin2016}.
This work adresses the issue of working with two different datasets. They call them source and target dataset, and each one contains data following different distributions. Moreover, source is labeled while target is unlabeled. Data are in a total amount of N, divided into n samples in source and N-n samples in target. Their goal is to create a network able to learn relevant features  for a classification task from the source dataset (since is the only one labeled), and learn as well some distinctive feature related to the two domains. This result is achivable by constructing a model able to generalize well from one domain to another.
%the inner representation of the network contains no discriminative information about the origin of the input.

We illustrate the main aspects of this problem using a shallow neural network with a single hidden layer consisting of D nodes.
We suppose the network takes as input an m-dimensional features vector.
The hidden layer can be represented as a function $G_f : \Re^m \rightarrow \Re^D$ with a weight parameters matrix W and bias vector b which for brevity's sake we can denote as $\theta_f$.
Given an input vector $x \in \Re^m$ it acts like
\begin{equation}
G_f(x; \mathbf{W}, \mathbf{b}) = f(W\cdot x + b)
\end{equation}
%where $\textbf{W}$ and $\textbf{b}$ are the matrix and vector parameters to be optimized.
Where f is some activation function that we can represent as a sigmoid function.

Similarly the output will be a function $G_y:R^D \rightarrow \Re^Y$ where Y is the total number of classes of our data. This layer can be written as
\begin{equation}
G_y(G_f(x; W, b); V, c) = f'(V\cdot G_f(x) + c)
\end{equation}
Here, too we can denote parameters V and c with a single notation $\theta_y$.
%Where V and c are parameters to be optimized: a vector and a scalar respectively.
A usual train carried on only on the (labeled) Source domain, will therefore bring to the minimization of the loss function associated with the output, dependent on parameters $\theta_f, \theta_y$
\begin{equation}
\underset{\theta_f, \theta_y}{min} [ \frac{1}{n}\sum_{i = 1}^n L_y^i(Gy(G_f(x_i)), y_i )]
\end{equation}.

Training procedure and minimization of the loss function can be performed only on the source dataset consisting on n samples, since target data are unlabeled.
At this point, to tackle the problem of domain independence their idea is to consider the hidden layer as an internal representation of data, and use its information to create a domain regressor.

A domain regressor can be implemented as a layer $G_d$ completely similar to the output layer $G_y$, depending on parameters $V, c$ which we will denote as $\theta_d$. It takes as input the hidden layer $G_f$, and after being activated by a sigmoid function, returns an output.
We indicate the loss function associated to this output as $L_d$, and the loss previously introducted for label classification as $L_y$.

The complete optimization function can be now written as
\begin{equation}
E(\theta_f, \theta_y, \theta_d) = \frac{1}{n} \sum_{i = 1}^n L_y^i - \lambda (\frac{1}{n} \sum_{i = 1}^n L_d^i + \frac{1}{N-n} \sum_{i = n+1}^N L_d^i)
\end{equation}

Following this strategy, they perform a minimization with respect some parameters and a maximization with respect to others. More precisely they seek for a saddle point given by
\begin{equation}
\begin{split}
\hat \theta_f, \hat \theta_y = \underset{\theta_f, \theta_y}{argmin} E(\theta_f, \theta_y, \theta_d) \\
\hat \theta_d = \underset{\theta_d}{argmax} E(\theta_f, \theta_y, \theta_d)
\end{split}
\end{equation}

This task was accomplished by embedding the domain regressor $G_d$ into the neural network made by $G_f$ and $G_y$. The resulting neural network will consist on two branches: one for label classification and the other with the domain regressor.
They linked these parts using a \emph{gradient reversal layer} and exploited a classic stochastic gradient descent procedure to update weights.
A gradient reversal layer allow the training of the network in an adversarial way. It is placed at the top of the domain regressor branch, as we can see from the presentation image (fig \ref{fig:dann_original}) directly taken from their article.
Let us point out that so far, we discussed the structure of this network by using a shallow neural network consisting just on a single hidden layer, but this architecture can be generalised by adding additional layers for each branch of the network, as shown in figure \ref{fig:dann_original}


\begin{figure}[h]
\centering
\includegraphics[width=1\linewidth]{dann/dann}
\caption{A schematic representation of a Domain Adversarial Neural Network, from the presentation article \cite{ganin2016}. The network mainly consists of three parts: a feature extractor, which takes inputs and creates an internal representation of data. From it the network is splitted into two branches: a label classificator $G_y$ with associated loss $L_y$ and a domain regressor/classifier denoted as $G_d$ with associated loss $L_d$. This latter branch is linked to the feature extractor branch by making use of the gradient reversal layer.
}\label{fig:dann_original}
\end{figure}




During the backpropagation apt to parameters update, minimizing parameters are updated in the opposite direction of the gradient as a normal parameter fine tuning explained in section \ref{sec:backpropagation_theory} while maximizing parameters are modified going in the direction of gradient.

\begin{equation}
\begin{split}
 \theta_f = \theta_f - \lambda \frac{\partial L_y}{\partial \theta_f} \\
 \theta_y = \theta_y - \lambda \frac{\partial L_y}{\partial \theta_y} \\
\theta_d = \theta_d + \lambda \frac{\partial L_d}{\partial \theta_d}
\end{split}
\end{equation}

During training, classificaiton branch and domain regressor compete against each other in an adversarial way for the optimization of the parameters. For this reason this network is called Domain-Adversarial Nerual Network (DANN).



\chapter{SHAP}\label{chap:shap}


\begin{figure}[h]
\centering
\includegraphics[width=1\linewidth]{shap/shap_presentation}
\caption{}\label{fig:shap_waterfall_example}
\end{figure}


SHAP (SHapley Additive exPlanation) is a technique based on game theory that provides a method for machine learning model explanability.
It is a powerful tool to get rid of the \textquotedblleft black box \textquotedblright idea of a machine learning model and try to understand its deep mechanism and the reasons why a model gives a certain output, or a certain prediction related to an input sample such as a vector of feature or an image for example.
As is shown in figure \ref{fig:shap_waterfall_example} just as a visual example took from the SHAP documentation \footnote{https://shap.readthedocs.io/en/latest/}, a common machine learning model acts like a black box that returns an output value after some non-linear unknown calculations over some input values; with an explanatory model, we are able to quantify the contribution of each feature of our input data and assess what and why are the most important features and how much they contributed to the final outcome.

The idea behind SHAP is to use Shapley values (see chapter \ref{chap:shapley_values}) to exaplain every single feature's contribution in our machine learning model, treating the learning process as a cooperative game.
To apply the concept of Shapley values to a machine learning model we just adapt some terms we used for game theory: the payout of a coalition becomes now the model's prediction and the players are the feature in our input data.
For a single feature, its Shapley value is defined as the average marginal contribution of that feature across all possible coalitions.

There are two main classes of explainatory algorithms called local and global methods

Shap can be regarded as both a local and global explainatory method, and before it, several algorithms belonging to these categories were implemented as an attempt to create an explainatory technique for machine learning models.
In general, the objective of a local explainatory model can be defined as the attempt to explain an output f(x) after a single instance x that in our case can be identified as a vector of features.
Local methods differ from global methods, because the latter provide a global explanation of the model, across all the instances, they are able to attribute an importance to each feature to determine which one contributes the most to the output of the model.

One of the key point of SHAP is its flexibility, being both a local and global explainatory model, this way we are able to explain a single instance as well as an entire dataset and extract the most important features.

Being a local model, SHAP share with other algorithms some common implementation traits: given a single input vector x, to simplify the workflow, they introduce a coalition vector x' = $\left\{ 0, 1 \right\}^F$ : where F is the total number of features: x' is a binary vector of the same length of x, where binary means made just by zeros and ones as its entries: zero means that in our analysis we are going to withheld the corrisponding variable from x, and the one that we are including it.

To map the x' vector to the corresponding vector containing the actual value of the features, a mapping function $h_x(x')$ = x is introduced to returns the actual features vector x, unique for each vector x.
The goal of a local explainatory technique is to find a simplified explainatory model g(x') to work with, because is simpler to work with a binary vector x' and a simlified model than with the actual vector x and the model itself.
We want that this simplified model, when applied on the vector x', is able to approximate our output f(x): $ g(z') \approx f(h_x(z'))$ if z' $\approx$ x'.
The first important condition on this function g is that it has to be a linear function
\begin{equation}
\label{eq:shap_g}
g(z') = \phi_0 + \sum_{i = 0}^F \phi_i z_i'
\end{equation}
And several methods before SHAP were created that satisfy this condition such as LIME \cite{ribeiro-2016} or DeepLIFT \cite{shrikumar-2017} but they lacked additional properties that SHAP has:

\begin{enumerate}
\item \textbf{Local accuracy}:
\begin{equation}
g(x') =\phi_0 + \sum_i^F \phi_i x_i' \mathbf{=} f(x) \quad \text{when} \quad x = h_x(x')
\end{equation} and not just an approximation $f(x) \approx g(x')$. Here coefficients $\phi_i$ represent the impact of the associated feature to the model's output, and $\phi_0$ represents the coefficient corresponding to a vector x' with all entries equal to zeros.
\item \textbf{Missingness}:
\begin{equation}
\text{if} \ x_i' = 0 \Longrightarrow \ \text{then} \ \phi_i = 0
\end{equation}
Practically if a feature is equal to zero, it has no attributed impact on the model outcome
\item \textbf{Consistency}: if we have two different models g' and g. Denoting with $z' / i$ the setting where $z_i' = 0$
\begin{equation}
\text{if} \ g'(z') - g'(z' / i) \ge g(z') - g(z' / i) \forall z' \in \{0, 1\}^F \Longrightarrow \ \text{then} \ \phi_i(f', x) \ge \phi_i(f, x)
\end{equation}
This property states that if the contribution of a feature $z'_i$ increases, regardless all the other features in a model, its associated importance value increases or at least remain the same, but does not decrease.
\end{enumerate}
%The most important properties of this models is called local accuracy and states that g and f must give the same result when passing all the features: g has to match f when x = $h_x(x')$.
%We attribute a contribution $\phi_i$ to each simplified vector $z_i'$

In 1975 it was demostrated \cite{young1985} that the only coefficients satisfying all of the above properties are Shapley values, and as a consequence methods which employ different coefficients violate one of these properties, usually local accuracy and/or consistency.
For this reason SHAP employs Shapley values and takes advantage of pre-existing algorithms (some of which were cited before: LIME and DeepLIFT) that used different coefficients. It readapts and enhances them providing an unified approach to assess feature importance for different models without any violation of the axioms above.

To compute the Shapley value for a feature we have to evaluate the model on all the possible subsets S we can create with our data S$\subseteq F \setminus \left\{i\right\}$. This approach, in a problem with an high number of features would result in a huge computational work. To bypass this problem, we can choose not to calculate the exact shapley value but just an approximation of it, depending on the model we are working on. Some of them such as Tree SHAP, are able to compute the exact shap coefficients and is developed to obtain fast performances on trees or ensambles of trees.
Others just compute an aproximate shap value. Algorithms we are going to focus on are KernelSHAP which is a model agnostic explanation method, and DeepSHAP which is optimized to work faster on deep models by making use of the knowlegde of the structure of a neural network.
Each one of these algorithms is built upon previous algorithms like LIME or DeepLIFT. In the following section we are going to explain some important feature from LIME that were employed in KernelSHAP, and from DeepLIFT employed for DeepSHAP.

\section{LIME and KernelSHAP}\label{sec:lime}\hfill

LIME (Local Interpretable Model-agnostic Explanations) is a local, model-agnostic, interpretability model, made to explain the prediction of any classifier in a faithful, but only local way. This means that it can accurately explain a single prediction but it is not the most suitable to generalize to many of them \cite{ribeiro-2016}.

Since Kernel SHAP is built upon LIME algorithm, we briefly discuss how LIME works to obtain ...
it pick up an important strategy from it, to avoid the computation of all the permutations.

Starting from a single instance  $x \in \Re^n$  LIME creates a perturbed dataset X and evaluate the model over this dataset.
Perturbed dataset is created by making use of several binary vectors containing just zeros and ones. In fact, it randomly creates different binary vectors of the same lenght of x, but randomly set 0 and 1 as its entries.
From this binary dataset it is possible to come back to the space of features value, recalling that \textquotedblleft 1 \textquotedblright means that we are including the corresponding feature from x, while \textquotedblleft 0 \textquotedblright is associated to the missingness of the corresponding feature.
LIME replaces missing features in different ways according to the type of data it is dealing (Images, text data, tabular data..).
In the case of tabular data, LIME repleaces missing values by randomly sampling a value from the distribution of that feature from the training data  \cite{ferrando2018}.
KernelSHAP

KernelSHAP works in a similar way, but introduces a background dataset from which missing values are randomly picked and replaced in the perturbed vector.


From this point LIME works by using the synthetic dataset just created, and the model is evaluated on each vector.
These outputs are subsequently used to accomplish a minimization task to finally find the importance coefficients.
Minimization is performed in a regression, but before it, each output value is weighted according to the distance of the synthetic vector with the original vector x by using coefficients $\pi_x$ defined as
\begin{equation}
\pi_x = \frac{p-1}{{p \choose |z'|} \cdot |z'| (p-|z'|)}
\end{equation}
Where we indicated as p the total number of features in the original feature vector x anc with $|z'|$ the number of element in that subset, and cosequently ($p-|z'|$) is the number of features not included in the subset.

To perform a fit, the loss function to minimize is in the form $L(f, g, \pi_x) = \sum_{z\in Z} [f(h(z))-g(z')]^2\ \pi_x$
where g(z') is the simplified function expressed in equation \ref{eq:shap_g}, given by a combination of coefficients $\phi_i$ that are the parameters of the fit.

LIME is regarded as local method because it explains the prediction of a black box model by making use of a local model: in this case a regression, which is an interpretable model computed in the neighborhood of the instance we want to explain.
KernelSHAP implementation was strongly influenced by the LIME algorithm but,thanks to the introduction of Shapley values, it also comes with theoretical guarantees about consistency and local accuracy, from game theory.
%In a nutshell, the main steps of KernelSHAP algorithm are
%\begin{enumerate}
%\item Given a feature vector $\mathbf{x} = [x_1, x2, ..., x_p]$, where $x_1, .., x_p$ are the features (numbers), a model f and a background set $X_{bckg} = \{ %\mathbf{x1}, ... , \mathbf{xn}\}$, we want to explain \textbf{x}
%\item replace a subset of entries of x with values from the background dataset:
%\item create n different copies of the original input fector x, all slightly different from each other because of these replacements
%\item Evaluate the model on each copy
%\item Use these outputs to fit simple model and find shapley coefficients
%\end{enumerate}

% shap
%deepLIFT \cite{shrikumar-2017}

As mentioned above, KernelSHAP is a model-agnostic algorithm, in the sense that can easily used with every kind of model, but SHAP includes some model-specifics algorithms that make use of a previous general knowledge of a model's structure to optimize the explainer's performances and speed up the process. DeepSHAP, is one of these model-specific algorithms and it is what we are going to use in our analysis.

\section{DeepLIFT and DeepSHAP}
%deeplift github: \footnote{https://github.com/kundajelab/deeplift}
DeepSHAP alghorithm works with some similar implementation of KernelSHAP, making use of a background dataset to simulate missing values, but it is also optimized to perform better on deep models by taking advantage of the idea behind the DeepLIFT algorithm.
It can be considered as an enhanced version of DeepLIFT, thanks to the introduction of the shapley values, rather than the \textquotedblleft DeepLIFT multipliers \textquotedblright employed in DeepLIFT

The strategy implemented by DeepLIFT to explain features, is to compute the difference between the output of a deep learning model with our \textquotedblleft true  \textquotedblright data and a reference value computed as the output of our model with some reference data.
The choice of the reference data depends on what kind of data we are working on: images, or genomic data for example.
For genomic the most common way to produce reference output in DeepLIFT is to shuffle some training data, evaluate the model on them and average across all the scores.
With the implementation of DeepSHAP, this reference value is chosen from the backgound dataset and represents an uninformative value for a feature.

As we mentioned before is also shaped to perform on deep learning algorithm computing its feature importance values called DeepLIFT multiplier during the process of backpropagation.
DeepLIFT attributes to each feature $x_i$ a value $C_{\Delta x_i, \Delta y}$ that represent the effect of that input being set to a reference value rather than its actual value.

In a few words the idea behind DeepLIFT (and DeepSHAP) can be summarised as: let t represent the output of some inner neurons and let x1, x2, ... , $x_n$ be some preceding neurons necessary to compute t, if we label $t_0$ the reference output, we can compute the value $\Delta t = t-t_0$ and use them to define the DeepLIFT values $C_{\Delta x_i \Delta t}$ associated with $\Delta x_i$ which is the difference between the true feature and the reference value for that feature.
DeepLIFT coefficients are choose to follow the property called \emph{summation to delta} property
\[
\sum_{i = 1} ^n C_{\Delta x_i \Delta t} = \Delta t
\]

For a given input neuron x with difference from reference $\Delta x$ and a target neuron t, is defined the \emph{multiplier}
\[
m_{\Delta x \Delta t} = \frac{C_{\Delta x \Delta t}}{\Delta x}
\]
which represents the contribution of $\Delta x$ to $\Delta t$. Once computed the multipliers of each neuron in a layer it is possible to compute the multiplier of any target neuron during the backpropagation.
For a more detailed explanation of how DeepLIFT works we refer to its presentation article \cite{shrikumar-2017}

DeepSHAP exploits the same principles as DeepLIFT, and combines shap values for smaller components of a network to compute values for the whole network. It accomplish this by recursively passing Deep LIFT multipliers now defined in terms of SHAP values during backpropagation.
Since notation can become quite annoying, we explain the main idea of how this procedure works.
Suppose we have a single input vector x consisting only on two features: $x = (x_1, x_2)$ and a single background vector $b = (b_{x_1}, b_{x_2})$ we follow the notation on article \cite{chen2019} but we simplify for a better understanding.
We denote as $f_{x_i}$ the actual value of the feature $x_i$ and as $b_{x_i}$ the value of the entry from the backgound vector corresponding to $x_i$, and we also denote as $h_i$ the output of a neuron with the input is x, and with $b_{h_i}$ the output of the neuron when the input is b.
Giving a look at figure \ref{fig:deepshap_linear} we can write formulas for forward propagation.

\begin{figure}[h!]
\centering
\includegraphics[width=0.6\linewidth]{shap/deepshap_linear}
\caption{Representation of a linear network and important quantities involved during forward propagation}
\label{fig:deepshap_linear}
\end{figure}


\[
h_1 = w_{1, 1}x_1 + w_{2, 1}x_2 \qquad y = w_1 h_1 + w_2 h_2
\]
Giving a look at the model, we can retrieve Shap values for each node by summation of all the contribution from each path, where the path contribution to a shapley value for $x_i$ is defined as the product of the weights along the path and the difference between $x_i$ and the background value $b_{x_i}$.
In practice, if we focus on the colored blue line we have that the contribution from this path to $\phi(h_1)$ is given by
\[
\phi(h_1) = w_1 (f_{h_1} - b_{h_1})
\]
with this information, we can finally compute the contribution to the Shapley value of $x_1$ coming from this path
\begin{equation}
\phi(x_1) = (f_{x_1} - b_{x_1}) w_{1, 1} w_1 = (f_{x_1} - b_{x_1}) w_{1, 1} \frac{\phi(h_1)}{f_{h_1} - b_{h_1}}
\end{equation}
adding all contribution from all the possible paths (just two in the figure above, but much more in a real neural network), it is possible to compute the final Shapley value for a feature $x_i$ in terms of the attributions of all the intermediary nodes of the network.




\section{Shap values as feature importance}

To express the significance of a feature, we can rely on the relation between feature importance in a qualitative way and shap value: feature with high absolute shapley value are important.
The absolute importance value for a feature f is calculated as the mean of the magnitude of all the shapley value for that feature, so the mean is performed across all the n samples we used to calculate these values
\begin{equation}\label{eq:shap_magnitude}
I_f = \frac{1}{n} \sum_{i = 1}^n \|\phi_i^{(f)}\|
\end{equation}

Shap is implemented as a free Python package with MIT license, developed and mantained by Scott Lundberg \footnote{https://github.com/slundberg/shap}.
This package includes different classes such as KernelExplainer, which is the class name of KernelSHAP mentioned in section \ref{sec:lime}, TreeExplainer tuned to perform rapidly on tree and forest-like models, linearExplainer which deals with linear model and is able to compute the exact shapley values and not just an aproximation, and the class we are using: DeepExplainer, suitable to explain deep learning models and to compute an approximate value of shapley values.


\newpage





\addcontentsline{toc}{chapter}{IMPLEMENTATION \& RESULTS}
%\chapter*{IMPLEMENTATION \& RESULTS}
\chapter{Analysis workflow}\label{sec:analysis_workflow}
So far we have been discussing what type of data we used in our work without specifying what kind of analysis we carried out with those data.
We mainly performed classification using two different machine learning classifier: a deep neural network and a common Random Forest classifier.
Before running any classification procedure we tackled the problem of data harmonization. To this end in chapter \ref{chap:harmonization_results} we explain how we implemented the procedure of data harmonization and some results obtained by applying it on our data obtained from the analysis discussed in chapter \ref{chap:connectivity_coefficients} which we report for convenience:
\begin{itemize}
\item Data obtained by Fisher-transforming Pearson correlation coefficients extracted for each pair of timeseries of each patient
\item Data obtained from wavelet analysis by computing time of in-phase or out-of-phase percentage coefficients for each pair of timeseries of each patient
\end{itemize}
We report results to have a visual feedback of how harmonization modify data and how effective is this procedure in eliminating site-related information from our data.

In chapter \ref{chap:results_deeplearning} we present results obtained from classification of data using different machine learning algorithms. We seek to find the best strategy to obtain the best separability between Controls and ASD data.
Classification was carried out following different pipelines: we compared performances between a Random Forest classifier, a deep neural network and a Domain-adversarial neural network. We also discuss the effects of data harmonization on a classification procedure and we discuss the best strategy to implement harmonization.

\paragraph{Analysis pipelines}\hfill

\begin{figure}[h!]
\centering
\includegraphics[width=0.6\linewidth]{flowchart/flowchart_analysis}
\captionsetup{singlelinecheck = off}
\caption[]{
Flowchart of the analysis carried out in this work:
\begin{enumerate}
\item Coefficient selection: select whether to use Pearson-based correlation coefficients, or wavelet-based correlation coefficients
\item Patients attributes selection: select the desired attributes of patients in order to make cuts on dataset, for example use the whole ABIDE I + II dataset or just ABIDE I, limit analysis to a certain age range, sex, or eye status open or close.
\item After making cuts, and obtained raw data, classification can be performed. Classification of Controls/ASD can be fullfilled by a deep neural network, a Random Forest classifier, or the Domain-Adversarial neural network.
\item Different harmonization procedures are compared: 1) harmonization implemented inside the k-fold CV,  2)harmonization of the entire dataset before k-fold CV, 3) no harmonization 4) harmonization implemented with the Domain adversarial neural network.
\item For each of procedures, the final classification score is reported as the mean and standard deviation across all the folds of k-fold CV.
\end{enumerate}
}
\label{fig:workflowanalysis}
\end{figure}

For each classification analysis, we implemented a K-fold cross validation scheme to report our results. We determined and reported model performances as the mean AUC value and the standard deviation of the scores for each k-fold
Since we are not working with a huge amount of data ($< 1600$ samples), we choose to perform a 5-fold CV. We made this choice because with a 10 k-fold CV, the error on the mean AUC score increases, since we have less data in the test dataset, and hence more variability.
For this reason we choose to hold a greater number of data in the Test (at the cost of reducing the number of data available for train) to reduce variability among them and reduce the standard deviation of the model's score.
K-fold CV was implemented using the stratifiedKFold class provided by sklearn library for Python.


A flowchart of the analysis mentioned above is shown in figure \ref{fig:workflowanalysis}
In details the four types of analysis we carried out are:
\begin{enumerate}
\item Classification of ASD/controls using raw (not harmonized) data
\item Classification using the whole harmonized dataset, creating the harmonization model on all the control subjects and applying the model on all ASD subjects, and then split this dataset into a train and test subsets for each k-fold. The flowchart to schematise this implementation is shown in figure \ref{fig:harmon_upstream_flowchart}.
\item Classification with harmonization implemented inside the K-fold cross-validation, creating the model on controls and applying it to ASD: following this procedure data were splitted into a train and a test dataset, therefore using train data we computed the harmonization model only on control patients and then apply the model to ASD patient and both Control and ASD Test data.
The whole flowchart is shown in figure \ref{fig:harmon_kfold_flowchart}
\item Classification using the domain adversarial neural network.
\end{enumerate}

\textcolor{blue}{
Harmonizing data inside a k-fold is the best way to keep train and test data apart and avoid data leakage between train and test.
It is a common source of error to introduce some kind of data leakage between the train and the test dataset, when we harmonize on the entire dataset, the harmonization model permorms a fit using all the data, so each data is modified according to information obtained from all the other data, so in data belonging to the train dataset there's already information about the test dataset, and this leads to a misleading increase of model's performances.
Some articles perform a Leave One Site Out Cross validation. In our case we can't do that if we want to mantain Train and Test independent because if we compute the harmonization model only on the train dataset, we wouldn't have information to apply it to the test dataset since its data belong to a site not included in the model.
}


\begin{figure}[h!]
\centering
\includegraphics[width=0.6\linewidth]{ml/flowcharts/harmon_kfold}
\caption{Flowchart for the implementation of harmonization inside a k-fold CV: for each fold, data are splitted into a train and a test dataset. Harmonization model is created on control data of the train dataset, and applied to ASD data and on the entire test dataset. Harmonized control and ASD train data are merged to obtain a harmonized train dataset. Classification is performed on the new harmonized dataset: training is accomplished on the train dataset and the model is evaluated on test dataset. This entire process is repeated for each fold with different train/test data.}
\label{fig:harmon_kfold_flowchart}
\end{figure}



\begin{figure}[h!]
\centering
\includegraphics[width=0.45\linewidth]{ml/flowcharts/harmon_upstream}
\caption{Flowchart of the upstream implementation of harmonization, before the k-fold split: from the entire dataset controls are collected, and the harmonization model is built with only these controls data. The model is applied to the controls and ASD, and the harmonized dataset is created. With this harmonized dataset, a k-fold CV is performed: the dataset is split into train and test subsets and on these data, and classification is carried through.}
\label{fig:harmon_upstream_flowchart}
\end{figure}


\paragraph{Dataset selection} \hfill

With reference to the \textquotedblleft Patients attributes selection \textquotedblright in figure \ref{fig:workflowanalysis} we discuss now what selection criterias we adopted.
All the analysis mentioned above were performed using different selection criterias and thresholds on the dataset which brought us to different datasets combination.
First of all, we excluded from the analysis data from sites NYU\_2 and KUL\_3 because as noticed in chapter \ref{chap:dataset} they did not provide control data but only ASD patients, and without control data, we can't perform harmonization on these sites because to create a harmonization model, we perform a regression on control data and then apply the model to ASD data of that site.
The first and most obvious selection criteria is to use the whole dataset from ABIDE 1 and 2, consisting of only male patients aged between 5 and 40 years old, without any further constraint. So we have our first complete dataset consisting on 1470 patient data, from ABIDE I + ABIDE II datasets.
This dataset can be divided to analyse classification performances separately on ABIDE I and ABIDE II, mantaining the same constraints on ages and sex.
This choice was made because several articles dealing with controls/ASD classification using the ABIDE dataset, are fullfilled only on ABIDE I, or, more precisely, on ABIDE I preprocessed (see section \ref{sec:cpac}). Thus for an immediate comparison we choose to run classification on the two disjoint dataset as well.

To seek for a more homogeneous dataset, an other constraint can be set on the eye status at scan. The objective is to select only patients who kept open eyes throughout the entire scan session. This, as mentioned in chapter \ref{sec:intro_ml} helps obtaining a more homogeneous dataset, removing also data potentially altered by sleep.
This constraint was applied on the dataset consisting on data from ABIDE I and ABIDE II pooled together  cited earlier, and on ABIDE I and ABIDE II separately as well.
In short, considering these constraints, we obtain 6 different subdatasets, with different number of patients each, summarised in table \ref{tab:controlASD_per_subset}




\begin{table}[!htp]\centering
\scriptsize
\begin{tabular}{lcrrr}\toprule
Dataset &Constraints &Tot &Controls &ASD \\\midrule
AB I+II &eye = all, sex = M, age = (5, 40) &1470 &737 &733 \\
AB I& eye = all&841 &426 &415 \\
AB II&eye = all &629 &311 &318 \\
AB I + II &eye = open &1026 &514 &512 \\
AB I& eye=open &568 &281 &287 \\
AB II &eye = open &458 &233 &225 \\
\bottomrule
\end{tabular}
\caption{Total number of data and control/ASD amount for each subset and thresholds. To avoid repetitions in all the subdatasets, are implicit constraints regarding sex (only males) and ages (between 5 and 40).}
\label{tab:controlASD_per_subset}
\end{table}


\paragraph{Dimensionality reduction of data} \hfill

An important issue related to data we are working on concerns their dimensionality.
We are dealing with data lying in an high dimensional space but we lack an apropriate number of data to make acccurate predictions on them: this problem is usually referred as the \textquotedblleft curse of dimensionality \textquotedblright
%\footnote{https://en.wikipedia.org/wiki/Curse\_of\_dimensionality\#Machine_Learning} .
In our work, dimensionality is a relevant issue since we are dealing with 5995 features and less then 1500 patients, and for this reason we run our analysis in a situation of permantent overfitting.
One important method to tackle this problem is dimensionality reduction performing a Principal Components Analysis (PCA) as explained in section \ref{sec:pca}

% come ho calcolato le pca? solo sul training set
%non avrebbe senso calcolare la PCA solo sui controlli perchè pca cerca la direzione di massima varianza, dove i davi variano maggiornente, e quidni dove si ha una migliore distinguibilità bewteen controls and ASDs. NOn avrebbe senso farla solo sui controlli.
%Il numero di PC può essere più grande del test dataset, ma non sarà mai più grande del numero di feature in esso (per come è definito il massimo numero di pc), quindi una volta estratte le PC può essere applicato a qualsiasi test dataset

When applying PCA to a dataset we seek to explain as much variance as possible, for this reason we start our pca analysis extracting a number of principal components that explain more then 90$\%$ of variance, then we gradually reduced this number halving if.
The maximum number of principal components is limited by the number of data we can use. If we have a dataset of $n\_samples$ samples, each with $n\_features$ features, the maximum number of PC is given by $N\_pc = min\{n\_samples, n\_features\}$.
This is because in a $n-feature$-dimensional space, our points lie in a $n\_samples$-dimensional hyperplane and all the variance can be explained inside this subspace.
For a better understanding of this concept, we can imagine a dataset made of only two data, each containing 3 features. If we represent these data, we construct a 3D space where each orthogonal axes corresponds to a feature, and we insert these two points, each corresponding to a data point.
If we want to explain the maximum variance we need to find a new axis, the first principal component, but through two points passes just one straight line so we can at most limit our analysis to the plane passing through that line.

In our case $n\_samples$ is always $< \ n\_features$, so the number of principal components will be limited by the number of data in the Train dataset.
We use just the train dataset to calculate principal components because this is the correct way to proceed to avoid data leakage between Train and Test dataset.

To avoid repeating all the analysis mentioned earlier with the additional implementation of PCA, we choose to implement it only on some sub-dataset and using those coefficients that give the best classification performances.

%For each one of the analysis above, PCA was implemented to restrain the effect of overfitting.
From the initial input data consisting of 5995 features, we performed classification using different values of principal components. The search for principal components is accomplished only on the train dataset and then the same transformation is applied on the test dataset. In accordance of what we stated about harmonization when we asked whether is more correct to perform it inside or outside the k-fold splitting of train/test, calculating principal components just on the training set, allows to keep train and test set apart, and avoid data leakage of test into train.

Results are shown in chapter \ref{sec:results_pca}



\chapter{Harmonization - results}\label{chap:harmonization_results}

% data do not present a clear age-trend  (and FIQ?)
%scrivere che quando faccio l'harmonization non uso nessuna covariata specificandola come non lineare perchè non ci sono ragioni per farlo
As described in section \ref{sec:harmonizationtheory}, ComBat-based harmonization procedure simultaneously models and estimates biological and non biological terms, from data and algebraically removes the estimated addivite and multiplicative site-effect terms.

Harmonization procedure was tested on the datased created by using patients from ABIDE I and II after the cuts on sex and age discussed in section \ref{sec:analysis_workflow}.
From these patients we tested harmonization using Pearson-based correlation coefficients and wavelet coefficients of in phase time percentage.
We run this analysis to have a quantitative and a visual representation of how features are modified after being processed with harmonization pipeline.

To test harmonization procedure and evaluate its performances in making feature uniform in relation to sites, we used NeuroHarmonize \footnote{https://github.com/rpomponio/neuroHarmonize} package for Pyhton, developed by Pomponio et al. \cite{pomponio-2019} which implemented and added some features to the previous NeuroComBat Python package \footnote{https://github.com/Jfortin1/ComBatHarmonization}.

To use Neuroharmonize package we need to input the feature data to harmonize and the corresponding covariate information with site, age, sex, and other biological quantities we desire to take into account.
For our data these information were provided by a csv file on the ABIDE website. In this file, for each patient, are collected several medical and biologiclal information useful to run a great number of different analysis.
A regression model is thus created with parameters estimated with the procedure explained in section \ref{sec:harmonizationtheory}. This model can be applied to the same input data we used to generate it, or to new, unseen data acquired from the same sites we used to create the model.

We tested harmonization procedure on 1470 male patients coming from ABIDE I and ABIDE II of which 737 controls and 733 cases, choosing as covariate to preserve the age, and the FIQ (Full intellective quotient) to mantain important possible biolgical trends in the data and avoid overcorrection.

The central idea is to create the model on control subjects only, to operate without the influence of informations related to ASD patients whose feature might follow a different distribution compared to control.
Once created the model on control subject, it is applied to both control and ASD subjects, following this procedure site-related information are eliminated from our data.
To test the effectiveness of this harmonization procedure, we used a Random Forest Regression to test the site classification performances before and after the harmonization procedure.
In this analysis, we split the model into a Train and Test subsets: using only control subjects from the train dataset, we created the harmonization model as explained before.

To actually harmonize data, the model is applied to all the data: Controls and cases of Train dataset and Control and Cases of the Test.
We remind that this approach, allows train and test dataset to remain independent, because harmonization model is only created on train subjects and once learnt the parameters, is applied to the Test dataset. Thus, data we use to train the random forest, are kept separated from the test set, and there's no information leakage of the test data into the training data.

Two random forest classifiers were trained. One using raw (non-harmonized) data, and the second using harmonized data.
Their task was to make a binary classification of data: site vs site, for each pair of sites.


\begin{notes}
\item As mentioned in the previous paragraph, two sites namely \texttt{NYU\_2} and \texttt{KUL\_3} provided only ASD patients, and, since we are estimating the mean and the variance for each feature across all sites, when we train a model only on control cases, and we apply it on patients belonging to new sites, we don't have information to scale and modify those data so the model would output NaN values.
This is the reason why in our analysis we excluded these sites since we can't perform a proper harmonization of them.
\end{notes}

Figures \ref{fig:heatmap_harmonization} and \ref{fig:heatmap_harmonization_w} show the results of classification site vs site using the two coefficients described above:  Pearson and wavelet correlations. Sites along the x and y axes are ordered by incresging average age of patients and scores are reported in terms of AUC,

Specifically, fig \ref{fig:heatmap_harmonization_noharmon} and \ref{fig:heatmap_harmonization_noharmon_w} show the AUC score of the model trained with the raw, not harmonized dataset, while in figures \ref{fig:heatmap_harmonization_harmon} and \ref{fig:heatmap_harmonization_harmon_w} show the results obtained with harmonized data.



% HEATMAPS
\begin{figure}
\centering
\begin{subfigure}[b]{0.8\textwidth}
   \includegraphics[width=1\linewidth]{/home/federico/magistrale/Imagess/harmonization/pearson/matrix}
   \caption{Heatmap with AUC score of a binary classification site vs site with raw data}
   \label{fig:heatmap_harmonization_harmon}
\end{subfigure}

\begin{subfigure}[b]{0.8\textwidth}
   \includegraphics[width=1\linewidth]{/home/federico/magistrale/Imagess/harmonization/pearson/harmon_matrix}
   \caption{Heatmap with AUC score of a binary classification site vs site with harmonized data}
   \label{fig:heatmap_harmonization_noharmon}
\end{subfigure}

\caption{Comparison of two heatmaps with AUC score of binary classification site vs site with \textbf{Pearson}-based correlation coefficients of raw (fig \ref{fig:heatmap_harmonization_harmon}) and harmonized data (fig \ref{fig:heatmap_harmonization_noharmon}) classified using a Random Forest Classifier
}
\label{fig:heatmap_harmonization}
\end{figure}



\begin{figure}
\centering
\begin{subfigure}[b]{0.70\textwidth}
   \includegraphics[width=1\linewidth]{/home/federico/magistrale/Imagess/harmonization/wavelets/matrix}
   \caption{Heatmap with AUC score of a binary classification site vs site with raw data}
   \label{fig:heatmap_harmonization_harmon_w}
\end{subfigure}

\begin{subfigure}[b]{0.70\textwidth}
   \includegraphics[width=1\linewidth]{/home/federico/magistrale/Imagess/harmonization/wavelets/harmon_matrix}
   \caption{Heatmap with AUC score of a binary classification site vs site with harmonized data}
   \label{fig:heatmap_harmonization_noharmon_w}
\end{subfigure}

\caption{Comparison of two heatmaps with AUC score of binary classification site vs site of \textbf{wavelet}-based correlation coefficients of raw (fig \ref{fig:heatmap_harmonization_harmon_w}) and harmonized data (fig \ref{fig:heatmap_harmonization_noharmon_w}) classified using a Random Forest Classifier
}
\label{fig:heatmap_harmonization_w}
\end{figure}





To have a visual representation of how much each feature is modified in respect of its sourced site, in fig \ref{fig:features_raw-harmo} are shown two features among the 5995: \emph{feature 324} and \emph{feature 2800} as a function of sites. This figures show features computed with Pearson coefficients while in figure \ref{fig:wavelet_features_raw-harmo} are shown the same two features with wavelet coefficients.
Just for a clearer idea of what is shown, feature 324 represents the correlation between right and left inferior frontal gyrus, and feature 2800 that between left precuneous cortex and the left inferior frontal gyrus. There is not a particular reason for choosing these two feature, they are just randomly drawn from the 5995 possible combinations.
Their values are plotted as a box along the y axis and sites are indicated along the x-axis, it appears clear how the mean value of each feature for each site is shifted and features are stretched or shrinked to make them follow a more uniform distribution.

\begin{figure}
\centering
\begin{subfigure}[b]{1.\textwidth}
   \includegraphics[width=1\linewidth]{/home/federico/magistrale/Imagess/harmonization/pearson/Feature324raw-harmo}
   \caption{}
   \label{fig:feature324}
\end{subfigure}
\begin{subfigure}[b]{1.\textwidth}
   \includegraphics[width=1\linewidth]{/home/federico/magistrale/Imagess/harmonization/pearson/Feature2800raw-harmo}
   \caption{}
   \label{fig:feature2800}
\end{subfigure}
\caption{Pearson-based connectivity coefficients per site for features 324 (fig \ref{fig:feature324}) and 2800 (fig \ref{fig:feature2800}) before and after harmonization}
\label{fig:features_raw-harmo}
\end{figure}



 \begin{figure}
 \centering
 \begin{subfigure}[b]{1.\textwidth}
    \includegraphics[width=1\linewidth]{/home/federico/magistrale/Imagess/harmonization/wavelets/Feature324raw-harmo}
    \caption{}
    \label{fig:wavelet_feature324}
 \end{subfigure}
 \begin{subfigure}[b]{1.\textwidth}
    \includegraphics[width=1\linewidth]{/home/federico/magistrale/Imagess/harmonization/wavelets/Feature2800raw-harmo}
    \caption{}
    \label{fig:wavelet_feature2800}
 \end{subfigure}
 \caption{wavelet-based connectivity coefficients of in-phase percentage, for features 324 (fig \ref{fig:wavelet_feature324}) and 2800 (fig \ref{fig:wavelet_feature2800}) before and after harmonization}
 \label{fig:wavelet_features_raw-harmo}
 \end{figure}



 If we split this plot into control and ASD patients to see the effect of harmonization separately on these subsets, \ref{fig:features_control-ASD_raw-harmo} shows feature 324 computed using Pearson coefficients and figure \ref{fig:feature324-control-ASD_w} with wavelet coefficients.

 \begin{figure}
 \centering
 \begin{subfigure}[b]{1.\textwidth}
    \includegraphics[width=1\linewidth]{harmonization/pearson/Feature324control-ASD_raw-harmo}
    \caption{Feature 324 of Pearson correlation coefficients}
    \label{fig:feature324-control-ASD}
 \end{subfigure}
 \begin{subfigure}[b]{1.\textwidth}
    \includegraphics[width=1\linewidth]{harmonization/wavelets/Feature324control-ASD_raw-harmo}
    \caption{Feature 324 of wavelet-based correlation coefficients}
    \label{fig:feature324-control-ASD_w}
 \end{subfigure}
 \caption{Pearson-correlation and wavelet coefficients for features 324 before and after harmonization with a separated plot for controls and ASD}
 \label{fig:features_control-ASD_raw-harmo}
 \end{figure}




\section{Comments on harmonization}

%C'è da dire che anche dopo l'armonizzazione, il sito SU_2 rimane distinguibile, anche se dalle feature sembra essere stato allineato con gli altri. Quindi stessa storia potrebbe essere avvenuta per i wavelet.

Figure \ref{fig:heatmap_harmonization_w} shows a binary classification procedure with a random forest, for site discrimination of wavelet-based coefficients. It appears clear that with wavelet coefficients there is not the same outcome as with Pearson coefficients (figure \ref{fig:heatmap_harmonization}), in fact with wavelets sites still remain distinguishable after harmonization.

%This may be due to the loss of some information linked to the site througout all the process of extracting a correlation coefficient starting from a time-frequency analysis of two timeseries.
%Accordind to this hypotesis, if we lost site information, the particular trend of a feature we see for example in figure \ref{fig:wavelet_features_raw} is just a casual trend and taking into account all of the features, there is not a specific bias connected to a site.
%This way, during procedure of harmonization, we are creating a bias towards each site and for this reason after harmonization, sites remain distinguishable, or actually, they become even more recognisable
Nevertheless, if we plot the mean value of a feature computed with wavelet-based correlation across sites (figure \ref{wavelet_features_raw}) we note that there is a certain trend related to sites, but it is possible either that we obtain this trend just for this choice of feature or that even if the mean values of these features underwent a level out procedure, they still remain recognisable because of the values they assume which is not in the same range for all the sites and this can help the classificator recognising sites differences.

\hfill

Both for wavelet-based coefficients in figure \ref{fig:wavelet_features_raw} and for Pearson-based coefficients in figure \ref{fig:features_raw-harmo} we can notice a certain messy trend of features of raw data in respect to site (plots on the left side), which is not due to the presence of an unbalanced Controls/ASD number of patients per site, because from figure \ref{fig:features_control-ASD_raw-harmo} we observe as well, a messy trend both for Controls and ASD.
We can therefore conclude that in our data there's some kind of bias linked to the acquisition site.
After harmonization procedure both Pearson-based coefficients and wavelet-based coefficients assume a more regular trend (plots on the right side of figures  \ref{fig:wavelet_features_raw} and \ref{fig:features_raw-harmo})
This is a visual confirm that harmonization effectively removes site-related, but if for Pearson coefficients this is enough to remove inter-site variability, for wavelet coefficients

%This bias is more evident with Pearson-based correlation coefficients (fig \ref{fig:wavelet_features_raw}) than with wavelet-based coefficients (fig \ref{fig:features_raw-harmo}) that appear to assume a more uniform trend.



\newpage
\chapter{Deep neural models} \label{chap:deep_models}
%\section{Model's structure}
\paragraph{Deep neural network}\hfill

In this work we performed a classification task, using a shallow neural network built using Keras library for Python.
Since we are working in an unavoidable overfitting condition, we tried to build a network with as less feature as possible, but still preserving good classification performances.
We then started the research of our newtork with a network comprising only 3 layers and we searched the minimum number of neurons to employ in each layer.
We started with a trivial network made by just 3 neuron in the first layer, 2 nodes in the second and at last a single-output layer. Even if this was just an attempt to put a lower limit to the number of neurons, we noted that even with this configuration, the network tends to overfit our data.
This is clear if we take a look at the learning curves in figure \ref{fig:learningcurve} which shows the training and validation AUC curves for two models: the first one of them (\ref{fig:auc_no3-2-1}) even if with the simplified structure just mentioned, shows the classical trend of an overfitting state, however, it performes a bit worse than the other two more complex models.
We continued adding neurons to each layer and tried different configurations. With a configuration of 8-8-1 performances were slightly higher, so we set the second layer to 8 neurons and tried changing the number of the neurons in the first layer, performances seemed to increase as the number of neurons increased untill they reached a stable value. Different attempts and the related classification scores are reported in table \ref{tab:different_model_structures} in appendix.
We decided to pick the configuration comprising the minimum number of neurons, beyond which performances were stable, and the addition of more neurons to create a more complex network was unjustified.

\begin{figure}
\centering
\begin{subfigure}[b]{0.3\textwidth}
   \includegraphics[width=1\linewidth]{ml/auc_no64-8-1}
   \caption{AUC learning curve corresponding to a model with a structure 64-8-1 nodes trained on the entire not-harmonized dataset}
   \label{fig:auc_no64-8-1}
\end{subfigure}
\begin{subfigure}[b]{0.3\textwidth}
   \includegraphics[width=1\linewidth]{ml/auc_no8-8-1}
   \caption{AUC learning curve corresponding to a model with a structure 8-8-1 nodes trained on the entire not-harmonized dataset}
   \label{fig:auc_no8-8-1}
\end{subfigure}
\begin{subfigure}[b]{0.3\textwidth}
   \includegraphics[width=1\linewidth]{ml/auc_no3-2-1}
   \caption{AUC learning curve corresponding to a model with a structure 3-2-1 nodes trained on the entire not-harmonized dataset}
   \label{fig:auc_no3-2-1}
\end{subfigure}
\caption{}
\label{fig:learningcurve}
\end{figure}


In our work we employed the neural network schematized in figure \ref{fig:model_structure}.
This network consists of 2 hidden layers made up of 264, and 8 neurons respectively, both activated by a ReLU function and separated by a Batch-Normalization and a Dropout layer with a dropout chance of 30\%.
After the 8-node layer we added a second Batch-Normalization layer, before the output layer.
At the end we put a single output layer with a single neuron for the classification output.
This last layer is activated by a sigmoid function, which outputs a real number between 0 and 1.



\begin{figure}[h!]
\centering
\includegraphics[width=0.4\textwidth]{ml/models/model_structure}
\caption{Schematic structure of the deep neural network employed in this work for Control/ASD classification, with a scheme 264-8-1.
Each box contains the layer's name, layer's task (Input, Dense, BatchNormalization etc..), activation function, input shape (number of neurons) and output shape.}
\label{fig:model_structure}
\end{figure}


To create the network we set the subsequent hyperparameters after a grid search on learning rate and number of epochs, setting a validation set of 25\% of Train data, and studying the evolution of the learning curves, to find the minimal parameters configuration which allows high classification scores.
We trained the model using binary crossentropy loss function and optimizing parameters using Adam optimizer with a learning rate of $10^{-4}$.
Adam optimizer was chosen over Stochastic Gradient Descent optimizer since with Adam we empirically achieved similar classification performances with a lower error than with SGD.
%Model's classification performances were assessed with a 5 k-fold Cross validation, collecting the AUC score for each fold and then computing mean and std deviation of these results.

%Adversarial model:
\paragraph{Domain-adversarial neural network}\hfill

As mentioned in chapter \ref{sec:analysis_workflow} an other kind of analysis was performed using a Domain adversarial neural network.
This domain-adversarial or site-adversarial network is a model able to earn from data both class and site information, and avoid some sort of bias due to site, when performing classification.
The main idea behind this network comes from a revisited and customized version of the model described in chapter{chap:adversarial}, but before building this network we tried a different aproach.

The first attempt to implement a model able to predict category label without any influence of site's information is a model consisting on two different branches, one for the output of control/ASD prediction, and one for the output of the predicted site.
With these two informations, a loss is created by combining two different loss functions, one for the class and the other for site: the idea is similrar to that proposed in the article \ref{guan2021}.
This new loss is in the form
\begin{equation}
L = L_1 - \lambda L_2
\end{equation}
Where $L_1$  is in our case the loss for the binary classification Control/ASD that we want to minimize (binary crossentropy), while $L_2$ is the loss for site classification: categorical crossentropy we aim to maximize in order to avoid learning any information related to sites. $\lambda $ is a parameter empirically set, to control the contribution of the sites loss $L_2$ to the overall loss.
This way, the minimum of the total loss is reached with parameters that minimize the category classification loss and maximize the site classification loss.
Performances of this kind of model are strictly linked to the value of the parameter $\lambda$, in a sense that if a certain value gave an optimal performance on a certain dataset, on an other dataset, for example one created with some more constraints on patients, the value for which one had the best scores was slightly different.
And also, since when we first employed this model it gave slightly worse performances than simple deep model, we looked for an other strategy and carried out the analysis with this second type of model.
Thus we reached to the domain-adversarial neural network.
Even though this network is based on the same principles of the neural network with combination of two losses (minimize the error on classification of ASD/controls, and maximizing the one on site classification), it accomplish its task in a different way.
Our task is in fact to avoid the model to learn some site-related pattern from data and we accomplished so with a model whose inner weights are updated moving towards a minimization of the gradient for classification, and in the opposite direction for site classification, as explained in section \ref{chap:domain_adversarial_theory}.

To costruct our domain-adversarial neural network, we followed the same structure of the deep neural network implemented before, and we added the domain-adversarial branch.
We can describe the structure of this model as consisting on three parts: a feature extractor branch takes input data and creates an inner representation of them within its structure; at this point the network is forked into two branches: a label classificator for control/ASD cassification and the second branch for site classification. At the top of the site-classification branch, the gradient reversal layer was placed.

%with a composition of the two losses, the structure of this model can be divided into three main components, two of which are exactly the same of the deep neural model explained at the beginning of this section.
In details the three components that make up this model are: the first feature extraction branch comprising the input layer, the 264-nodes and the 8-node layer. Between the last two, we put a Batch-normalization and a 0.3 dropout layer, as we did in the first DNN.
From this point on the network is splitted into two branches: the categorical classifier, similarily to the DNN consists of a last layer with a single neuron activated by a sigmoid function, for the output of control/ASD classification.
The other branch, the site classifier consists of a gradient reversal layer as the first layer followed by a layer with 16 neurons, a Batch-Normalization layer and the last layer: a multi-output layer with N nodes, being N the total number of sites input data belong to. The effect of the gradient reversal layer can be regularized with a factor to weight the contribution of the site loss on the feature extractor parameters. We empirically set this value on 0.3.
The structure of this network is illustrated in figure \ref{fig:adv_model_structure}


\begin{figure}[h!]
\centering
\includegraphics[width=0.6\linewidth]{ml/models/adv_model_structure}
\caption{Structure of the domain-adversarial neural network with two outputs: a single-neuron output for binary classification controls/ASD and a multi-class classification for site classification.
Each box contains the layer's name, layer's task (Input, Dense, BatchNormalization etc..), activation function, input shape (number of neurons) and output shape.
}
\label{fig:adv_model_structure}
\end{figure}


The final layer of the site-classification branch is activated by a softmax function, and the loss function employed is the categorical crossentropy, commonly employed for multi-class classification tasks as explained in \ref{sec:loss_functions}.

%implemented using a custom function to deal with a possible low crossentropy score when by chance the site classifier, categorize one or more input data site label correctly, which would have low impact on the whole branch classifier making the adversarial strategy less effective.

\begin{notes}
\item To work with this loss we need to encode each site numerically before before use its label into our model because functions like these can't deal with string variables such as site names. For this reason we have to associate each name to a number: site\_a = 0, site\_b = 1 and so on. If we stop to this point and use integer numbers $\{0, ..., s\}$ as labels (where s corresponds to the integer associated to the last sites : N\_total\_sites - 1), our classification model would end up considering sites with higher number as \textquotedblleft larger \textquotedblright than others. This should be avoided in our case since all the sites are \textquotedblleft equally important \textquotedblright. % not-optimal results.
To avoid this a common strategy is to transform these variables according to a technique called One-hot-encoding.
This common approach converts each value to a vector of len s, containing all zeros but in the entry corresponding to that site, where it puts 1. With this transformation, each site label becomes a binary vector, with just a single 1 places in correspondence to that specific site.
\end{notes}


We tested both these different adversarial models (the model with loss combination and this with gradient reversal) and we noticed that they had more or less similar performances, but we choose to carry out the following analysis with this second model, because, even if, like the model with loss compositions, in the gradient reversal layer, a parameter $\lambda$ controls the weights updating during the backpropagation, results obtained with this model are more stable in respect to the the value of this parameter.
Moreover, there are other studies \cite{kamath2019} that compare the performances obtained with different implementation of adversarial network, and, even if they work with different data (text data) from ours, they find the DANN the best performing implementation; since this network appears to be more promising than the simple loss combination networks, we choose to employ this network in the following analysis.



\newpage
\chapter{Results of classification} \label{chap:results_deeplearning}
In the following section we report results of classification controls/ASD obtained by following different strategies of harmonization as explained in chapter \ref{sec:analysis_workflow} using the neural network and the adversarial neural network presented in chapter \ref{chap:deep_models}.

From now on, in all the tables we denote as \textquotedblleft Harmon k-fold \textquotedblright the pipeline where harmonization is implemented inside the k-fold and harmonization model is created only on control train data. \textquotedblleft Harmon. upstream \textquotedblright denote the pipeline where harmonization is implemented upstream, before the k-fold partition of dataset into train and test; with this pipeline, harmonization model is created using data of all controls.  \textquotedblleft No harmon. \textquotedblright denote the pipeline of classification of raw data, while  \textquotedblleft Adversarial \textquotedblright denotes the pipeline where we use the domain-adversarial neural network giving as input raw data.

And when listing datasets, we denote as ABIDE I+II / ABIDE I / ABIDE II datasets with cuts on covariates such as sex and ages as discussed in chapter \ref{sec:analysis_workflow} and when we select only patients with open eye we expressly specify it.

\section{Results with Pearson coefficients}\label{sec:pearson_results}
The first classification analysis was run on Pearson-related correlation coefficients. To recall the different analysis pipeline we can take a look at flowchart \ref{fig:workflowanalysis}, and in this analysis \textquotedblleft Coefficients selection \textquotedblright  translates as we are selecting coefficients based on Pearson correlation.

Results of classification are reported in terms of mean AUC across all the 5 folds.
We listed in table \ref{tab:pearson_results} results obtained with our DNN model followind different harmonization implementation and with the domain-adversarial network.

In table \ref{tab:comparison_dnn_rf_pearson} and we compared some of these results with a Random Forest classifier to assess whether is convenient to use a deep model instead of a conventional machine learning algorithm such as Random Forest.
Comparison with Random Forest were performed on the whole dataset ABIDE I+II, and on ABIDE I with open eye, since on this dataset we obtained the best classification performances. Results of DNN and RF concern the three pipelines of: hamonization inside K-fold, harmonization upstream, and raw data (non-harmonized).


%If the table is too wide, replace \begin{table}[!htp]...\end{table} with
%\begin{adjustwidth}{-2.5 cm}{-2.5 cm}\centering\begin{threeparttable}[!htb]...\end{threeparttable}\end{adjustwidth}

\begin{table}[!htp]\centering
\scriptsize
\begin{tabular}{lcccc}\toprule
Dataset &Harmon K-fold &Harmon. upstream &No harmon. &Adversarial \\\midrule
AB I+II &71$\pm$1 &74$\pm$2 &73$\pm$3 &70$\pm$3 \\
AB I &71$\pm$3 &74$\pm$2 &72$\pm$3 &71$\pm$4 \\
AB 2 &63 $\pm$5 &68$\pm$ 6 &64$\pm$ 4 &66 $\pm$ 4 \\
AB I + II, eye open &72$\pm$3 &75$\pm$4 &72$\pm$3 &71$\pm$3 \\
AB I, eye open &73$\pm$ 1 &76$\pm$2 &72$\pm$1 &72$\pm$4 \\
AB II, eye open &66$\pm$3 &70$\pm$6 &69$\pm$6 &68$\pm$6 \\
\bottomrule
\end{tabular}
\caption{AUC score obtained with the deep neural network, using Pearson-related coefficients and following different harmonization procedures.}
\label{tab:classification_pearson}
\end{table}



\begin{table}[h]
    \begin{subtable}[h]{0.45\textwidth}
        \centering
        \begin{tabular}{|lcc|}\toprule
        \multicolumn{3}{|c|}{AB I + II} \\\midrule
        &DNN &RF \\
        K-fold &71$\pm$1 &66+2 \\
        Upstream &74$\pm$2 &72+1 \\
        Non-harmon &73$\pm$3 &65+1 \\
        \bottomrule
        \end{tabular}
    \end{subtable}
    %\hfill
   \begin{subtable}[h]{0.45\textwidth}
       \centering
       \begin{tabular}{|lcc|}\toprule
       \multicolumn{3}{|c|}{AB I eye = open} \\\midrule
       &DNN &RF \\
       K-fold &73$\pm$ 1 &70+3 \\
       Upstream &76$\pm$2 &71+2 \\
       Non-harmon &72$\pm$1 &68+3 \\
       \bottomrule
       \end{tabular}
    \end{subtable}
    \caption{Classification score of a DNN and a RF classifier, using Pearcon correlation coefficients, for dataset: ABIDE I + II and ABIDE I with open eyes}
    \label{tab:comparison_dnn_rf_pearson}
\end{table}


\newpage

\section{Results with wavelet coefficients}\label{sec:wavelet_results}

The same analysis performed with Pearson-related coefficients were carried out with wavelet-related coefficients. If we refer once again to the flowchart \ref{fig:workflowanalysis}, now with  \textquotedblleft Coefficients selection \textquotedblright we denote four different dataset created using wavelet coefficients, whose classification results are reported in the respective tables.

\begin{enumerate}
\item In phase wavelet coefficients, with results reported in table \ref{tab:classification_win}
\item In phase and counter phase coefficient stacked together in a single array long twice the in-phase coefficients array. Results reported on table \ref{tab:classification_win+wout}
\item Counter phase wavelet coefficients, with results reported in appendix in table \ref{tab:classification_wout}
\item Coefficients resulting from the subtraction of the two: In phase - Counter phase: denoted as w\_in - w\_out, reported in appendix, in table \ref{tab:classification_win-wout} \textcolor{red}{Ha senso questa analisi? Come si giustifica?}
\end{enumerate}

We choose to put some tables in appendix to avoid to weight down this section with tables. We choose to put in appendix results less relevant both from a physical point of view and for classification scores.

\begin{table}[!htp]\centering
\scriptsize
\begin{tabular}{lrrrrr}\toprule
Dataset &Harmon K-fold &Harmon. upstream &No harmon. &Adversarial \\\midrule
AB I$\pm$II &65$\pm$ 2 &74$\pm$ 2 &66$\pm$ 2 &67 $\pm$2 \\
AB I &67$\pm$ 3 &73 $\pm$2 &68$\pm$ 3 &68$\pm$ 2 \\
AB 2 &62 $\pm$4 &68$\pm$ 4 &63$\pm$ 4 &62$\pm$ 4 \\
AB I $\pm$ II - eye open &65 $\pm$4 &71 $\pm$3 &66 $\pm$3 &66$\pm$ 3 \\
AB I open eye &69$\pm$ 4 &74$\pm$ 6 &67$\pm$ 3 &67 $\pm$4 \\
AB II eye open &66 $\pm$2 &69 $\pm$3 &67$\pm$ 2 &68 $\pm$2 \\
\bottomrule
\end{tabular}
\caption{AUC score obtained with the deep neural network, using wavelet coefficients w\_in and w\_out stacked together, following different harmonization procedures.}\label{tab:classification_win+wout}
\end{table}

As we did with Pearson-related coefficients, we compared results obtained using a DNN with results obtained from a Random Forest. We run this comparison only with coefficients that gave the best classification performances between the four coefficients (or combination of coefficients) listed at the beginning of this section. Results are compared in table \ref{tab:comparison_dnn_rf_win+wout}

\begin{table}[h]
    \begin{subtable}[h]{0.45\textwidth}
        \centering
        \begin{tabular}{|lcc|}\toprule
        \multicolumn{3}{|c|}{ABIDE I + II} \\\midrule
        &DNN &RF \\
        K-fold &65$\pm$ 2 &60+2 \\
        Upstream &74$\pm$ 2 &74+3 \\
        Non-harmon &66$\pm$ 2 &59+3 \\
        \bottomrule
        \end{tabular}
    \end{subtable}
    %\hfill
   \begin{subtable}[h]{0.45\textwidth}
       \centering
       \begin{tabular}{|lcc|}\toprule
       \multicolumn{3}{|c|}{AB I eye = open} \\\midrule
       &DNN &RF \\
       K-fold &69$\pm$ 4 &62+5 \\
       Upstream &74$\pm$ 6 &69+3 \\
       Non-harmon &67$\pm$ 3 &61+5 \\
       \bottomrule
       \end{tabular}
    \end{subtable}
    \caption{Classification scores comparison between the DNN model and a Random Forest classifier, on wavelet coherence coefficients W\_in and W\_out, for dataset: ABIDE I + II and ABIDE I with open eyes}
    \label{tab:comparison_dnn_rf_win+wout}
\end{table}
\newpage




\begin{table}[!htp]\centering
\scriptsize
\begin{tabular}{lrrrrr}\toprule
Dataset &Harmon K-fold &Harmon. upstream &No harmon. &Adversarial \\\midrule
AB I$\pm$II &65$\pm$ 2 &71$\pm$ 2 &62$\pm$ 2 &64$\pm$ 1 \\
AB I &66$\pm$ 3 &73 $\pm$3 &65$\pm$ 3 &66$\pm$ 2 \\
AB 2 &62 $\pm$3 &65 $\pm$3 &61 $\pm$4 &60 $\pm$2 \\
AB I $\pm$ II - eye open &64$\pm$ 4 &70 $\pm$3 &66$\pm$ 3 &65 $\pm$3 \\
AB I open eye &64 $\pm$5 &68 $\pm$5 &64 $\pm$3 &64 $\pm$4 \\
AB II eye open &66 $\pm$2 &68$\pm$ 3 &66$\pm$ 4 &65 $\pm$5 \\
\bottomrule
\end{tabular}
\caption{Classification scores: AUC, only w\_in coefficients}\label{tab:classification_win}

\end{table}


\newpage

\section{Results with PCA}\label{sec:results_pca}

As mentioned in section \ref{sec:analysis_workflow} we choose to run PCA analysis using coefficients that gave the best classification results in the previous analysis namely Pearson correlation coefficients. Furthermore, to not weigth down this table, we limit our dataset choice to the whole ABIDE I + II dataset and to ABIDE I only open eye, because on it we obtained a bit higher classification results.

We started our analysis choosing the proper number of PC to explain $> 90\%$ of variance and then gradually reduced them as reported in table \ref{tab:pca_explained}

\begin{table}[h]
\centering
\begin{tabular}{ |l|c c| }
\hline
\multicolumn{3}{|c|}{Explained variance} \\
 \hline
  N PC & ABIDE I + II [\%] & AB I eye open [\%] \\
  \hline
 800 & 93 & -- \\
 400 & 78 & 98 \\
 200 & 62 & 77 \\
 100 &48 & 58 \\
 50 & 36 & 41 \\
 20 & 24 & 26\\
 \hline
\end{tabular}
\caption{Explained variance [\%] for different numbers of principal components, and for the two dataset used in the analysis}
\label{tab:pca_explained}
\end{table}


We followed the same pipelines as we did for Pearson and wavelet coefficients, following the four different harmonization procedures.
Results obtained from this analysis are reported in table \ref{tab:classification_pearson_pca} and shown in figure \ref{fig:classification_pearson_pca}

\begin{table}[!htp]\centering
\scriptsize
\begin{tabular}{lrrrrrr}\toprule
PC &Dataset &Harmon K-fold &Harmon. upstream &No harmon. &Adversarial \\\midrule
800 &AB I+II &69$\pm$2 &73$\pm$3 &71$\pm$1 &71$\pm$2 \\
\hline
\multirow{2}{*}{400} &AB I+II &67$\pm$3 &73$\pm$5 &72$\pm$3 &73$\pm$3 \\
&AB I eye open &71$\pm$3 &72$\pm$3 &71$\pm$3 &72$\pm$4 \\
\hline
\multirow{2}{*}{200} &AB I+II &68$\pm$3 &74$\pm$2 &72$\pm$2 &71$\pm$2 \\
&AB I eye open &69$\pm$3 &73$\pm$2 &72$\pm$4 &73$\pm$3 \\
\hline
\multirow{2}{*}{100} &AB I+II &67 $\pm$2 &72$\pm$2 &69$\pm$1 &70$\pm$2 \\
&AB I - eye open &70$\pm$2 &73$\pm$2 &71$\pm$4 &72$\pm$2 \\
\hline
\multirow{2}{*}{50} &AB I+II &66$\pm$4 &70$\pm$6 &68$\pm$3 &69$\pm$2 \\
&AB I - eye open &70$\pm$4 &72$\pm$3 &69$\pm$3 &72$\pm$3 \\
\hline
\multirow{2}{*}{20} &AB I+II &65$\pm$5 &67$\pm$4 &64$\pm$5 &66$\pm$1 \\
&AB I - eye open &68$\pm$6 &71$\pm$2 &71$\pm$6 &71$\pm$3 \\
\bottomrule
\end{tabular}
\caption{AUC score obtained with the deep neural networks, using a decreasing number of principal components, following different harmonization procedures}
\label{tab:classification_pearson_pca}
\end{table}





\begin{figure}
\centering
\begin{subfigure}{0.4\textwidth}
   \includegraphics[width=1\linewidth]{ml/auc_vs_pc_all}
   \caption{Classification on the entire dataset ABIDE I+II}
   \label{}
\end{subfigure}
\begin{subfigure}{0.4\textwidth}
   \includegraphics[width=1\linewidth]{ml/auc_vs_pc_ab1}
   \caption{Classification on dataset ABIDE I - only open eyes}
   \label{}
\end{subfigure}
\caption{AUC scores obtained from control/ASD classification of data using decreasing numbers of principal components for Pearson-based coefficients.
Results obtained with different analysis are represented with different colors: AUC scores with data harmonized inside k.fold are marked \textbf{blue}, AUC score with data harmonized upstream \textbf{orange}, AUC scores with raw data (not harmonized) \textbf{green} and AUC scores classification with the adversarial network \textbf{red}.
}
\label{fig:classification_pearson_pca}
\end{figure}

\section{Discussion on classification results}\label{sec:classification_discussion}

\paragraph{Comparison between Pearson and wavelets}
From results reported on section \ref{sec:pearson_results} and \ref{sec:wavelet_results} we can observe that average results obtained with Pearson correlation coefficients are systematically higher than wavelet-based correlation coefficients.
We can therefore assume that working with wavelet-based correlation coefficient obtained through processes described in section \ref{sec:wavelet_theory} does not bring any advantages to this kind of analysis.
A possible reason for this outcome could be the loss of information during the extraction of these coefficients because of the introduction of redundancy and randomness into this analysis. Starting from two timeseries wavelet transform compute a 2D matrix for each timeseries and uses them to calculate the cross-power spectrum as explained in the dedicated section.
Doing so we create a redundant representation of timeseries data moving from one dimension to two-dimensional data.
We furthermore compare the cross-power spectrum obtained from the wavelet transform of each timeseries, with a red-noise cross-power spectrum to assess the significativity of our data.
In the end we shrink again our dimensions to reduce to a single data point representing a correlation.

Since differences of traits between controls and ASD are not very strong, this process can cause the loss of this weak information, which does not happen with a more direct and linear correlation calculus like with Pearson coefficients.

This lower scores is found both on wavelet coefficients of in-phase and counter-phase time percentage, and not even the combination of this two in a double-sized array, or with a coefficient made by the difference between these two coefficients brings any benefits, as an evidence that this procedure is not the best suitable to our classification goals.

\paragraph{Effects of a smaller and more regular dataset}\hfill

Let's focus for a moment just on table \ref{tab:classification_pearson} (even if these considerations also apply to wavelet results) to discuss the benefits of reducing the dataset size making cuts on covariates to run analysis on a more homogeneous dataset.
We can state that the eye status of patients during the scan has a non-negligible impact on the outcome of classification since limiting the dataset only on subjects with open eyes systematically improves classification scores of more than 1 \% for every dataset (ABIDE I+II, ABIDE I only, ABIDE II only).
As introduced in chapter \ref{chap:autism} there can be differences in brain functional areas between patients with open eyes and patients with eyes closed.
This is related to activation of different cortex and gyri areas or because during the scan, some patients may be fallen asleep, and this heavily modify the functional connectivity matrix.
For this reason, removing these source of variability could bring to a cleaner distinction of relevant patterns between controls data and ASD.

We can also notice that results on the dataset made by ABIDE I + II are mostly driven to the data belonging to ABIDE I than those belonging to ABIDE II, and this is true either if we restrict our analysis on open eyes patients only or we do not.
In fact, putting together the two dataset does not lead to a great improvement of scores than limiting just on ABIDE I.
It is possible that this trend is due to two main factors: the greater number of ABIDE I patient with respect to those in ABIDE II left after all the cuts, and the presence in ABIDE I of a great site: NYU containing a great percentage of patient which are also balanced between controls and ASD.
This could monopolise the results since a great percentage of train and test data would belong to this center leading to a lower variability due to site-related features as demonstrated in studies like \cite{spera-2019}.
\textcolor{red}{Ma allora perchè questo trend lo vediamo anche sui dati armonizzati?}


\paragraph{Comparison between a deep neural network and a random forest classifier}hfill

As a general rule of thumb is alway a good practice to compare results obtained with a complex model, to results obtained with a simpler machine learning classifier like Random Forest.
Instead of immediately opting for a complex model, simpler methods should be tried to establish a baseline and allow a meaningful comparison.
The principle of Occam's razor, when applied to machine learning demands that if two model perform quite the same, the simpler of the two should be picked \cite{domingos-1999}.
This is true expecially for our data since we start from a situation of overfitting and there is no need to choose a complex model a priori if this choice is not supported by data.
This is the same principle that lead us during the choice of the best structure and the best number of layers and neurons of the deep neural network.

Comparing the results obtained by the deep model with those obtained with a Random Forest, (table \ref{tab:comparison_dnn_rf_pearson} and \ref{tab:comparison_dnn_rf_win+wout}) it is possible to deduce that a deep model is more adapt to find characteristic patterns between controls and ASD than a simpler Ranfom Forest classifier.
Scores obtained with a deep model are systematically much higher than those obtained with a random forest classifier.
This trend is true, with the exception of data classified with the upstream harmonization pipeline, but as explained in the paragraph below, this procedure leads to biased data and as a consequence results obtained with this pipeline are not reliable.
We are still going to use Random Forest classifier in some of the remaining anslysis just as a comparison, even though we assert that a deep model is more suitable to understand differences between the two classes of interest.

\paragraph{Effect of harmonization}
An other common trend that stands out from all the analysis is the systematic improvement of AUC score, when we use data harmonized with the pipeline of upstream harmonization (explained in section \ref{sec:analysis_workflow} and outlined on flowchart in figure \ref{fig:harmon_upstream_flowchart}.
This improvement disappears when we implement harmonization inside the k-fold cross validation procedure (sketched in flowchart \ref{fig:harmon_kfold_flowchart}).
These two implementation differ on the order by which harmonization is implemented: with upstream harmonization we harmonize the whole dataset and after we split it into train and test to run the cross validation procedure; conversely, with harmonization implemented inside the cross validation procedure, we run harmonization only after the division of the dataset into a train and test subsets.
We can state that these high results are an evidence that the upstream harmonization is the wrong way to proceed, since classification results are affected by a bias due to data leakage which occur if we implement harmonization upstream.
It is also possible that some good results on other similar studies, obtained with harmonization implemented in this same way (\cite{ingalhalikar-2021}) may be due to a similar data leakage as we have in this analysis.
As a general definition, we have data leakage when information outside the training set is used to create and train models. This additional information allows the model to know something additional about the data that otherwise it would have not known.
This leads to an improvement of model's prediction and an overestimation of performances.
In this case we have data leakage when we harmonize the entire dataset before splitting it into train and test, because, according to this way to proceed, when we use train data, they already contain informations about test data.
In fact, harmonized train and test data they have been created using covariates, features and other informations belonging to the whole dataset.
We can then assert that the right way to implement harmonization is inside the k-fold cross validation, otherwise, if we don't run a cross validation, it is important to implement it after the splitting of dataset into a train and a test goups. This way, we can create the harmonization model only on control subjects belonging to train dataset, and we train the model with this dataset so there are not external information in it during training.
We can then harmonize data belonging to test applying the harmonization model, with parameters created only on the train dataset, on test data.
For this reason, from now on, when we refer to \textquotedblleft harmonized data \textquotedblright, we are referring to the procedure according to which data are harmonized inside the k-fold CV.

\hfill

Comparing results of harmonized data with results on raw data we don't notice any particular improvements. This is likely due to the weak information contained within our data. Distinction between healthy and ASD subjects based only on functional connectivity data is not a straigthforward task, and it is possible that these results are driven by all the information available. Removing site-related information from our data does not bring relevant improvements because this is all the information we can retrieve from our data, and with this procedure we are not able to bring out new information from our data.
However harmonization procedure can become a usefull tool to remove some noise from data and obtain a cleaner assessment of what feature are the most discriminative between controls and ASD. This theme is entirely addressed in chapter \ref{chap:shap_results} talking about feature importance.


\paragraph{Results with adversarial}
Looking at tables \ref{tab:classification_pearson} we notice that the adversarial model perform slightly worse working on Pearson-based coefficients while for wavelet coefficients (tables \ref{tab:classification_win} and \ref{tab:classification_win+wout}) we obtain on average the same values between DNN and adversarial.
A reason for this is the possible confusion in weight updates introduced by thhe adversarial network that sometimes may generate noise between weights belonging to the feature extraction branch. So far, in different papers, \cite{ganin2016}, \cite{kamath2019} or \cite{guan2021} (where, however adversarial is introduced just by combining losses), adversarial networks are employed with only two domains, which can be considered as the equivalent of our sites. In our data we deal with 36 sites when we consider ABIDE I+II, or, best case scenario, about 15 sites when we reduce to ABIDE I or II with just open eyes.
This way even if the site-predictor branch is encouraged to learn site-distinctive traits, it is not always able to distinguish between 36 sites, so this can cause confusion during backpropagation on weigths of the feature extractor branch. This noise introduced by this process may be the cause of this slight worsening of classification performances.
For this reason it is possible that when dealing with more than two domains (sites) this implementation of an adversarial network is not the most suitable way to proceed to remove site-related patterns from data.

%scrivere che quando faccio l'harmonization non uso nessuna covariata specificandola come non lineare perchè non ci sono ragioni per farlo
%A possible reason, in light of the results obtained with harmonization of wavelet coefficients which is not able to make sites-related feature unrecognisable, is that when computing wavelet coefficients, linear site-dependent relationship is lost and maybe a non linear relation is created.
%Same thing does not happen with pearson coefficients where linear relations remain. harmonization is able to remove linear relations on pearson but not on wavelets, and a classifier is more likely to get these non-linear relations with site and perform better on wavelets.


\paragraph{Effect of dimensionality reduction}
The same alertness we paid for the implementation of harmonization was applied when extracting PCA. As mentioned in the overview chapter \ref{sec:analysis_workflow}, it is possible to make the mistake of computing princial components on the whole dataset, before its separation into Train and Test sets.
For this reason we implemented PCA extracting PCs from the training dataset and applying the transformation to the both of Train and Test dataset.
\hfill

At a first glance we can notice that reducing the number of PCs leads to a gradual reduction of classification performances.
This trend is more evident in the whole datataset, while in the ABIDE I only open eyes dataset, we obtain a flatter performances curve.
A reduction from 5995 to 800 or 400 feature is a substanctial reduction since we are lowering our dimensionality by an order of magnitude still preserving an high classification AUC comparable to that obtained without PCA.
In this analysis as well, data leakage due to the prior harmonization of dataset plays an important role as it leads to a rise in classification performances for any choice of number of PCs.

An other aspect that leap out comparing results with PCA with results without it, is a light improvement of performances of the adversarial learning.
An explaination to this can be that a reduced source of error (deriving from this dimensionality reduction) brings to a reduced variability in site-related features making these confounding feature more recognizable. This could lead to a smoother weight updates that doesn't affect classification weigths as much as it does with raw data.

What arises from this analysis, is that PCA represent an important strategy to tackle the problem of dimensionality.
It is plausible tbat among all the 5995 features, a lots of them are uninformative for Control/ASD classification and they just increase dimensions of data without any benefit, or worse, just adding noise.
This can be linked to a biological reason: we can in fact imagine that there are some brain areas that just don't significantly change their activity between healthy and ASD people.
Following this explaination, all the coefficients representing a link between them are just similar between the two groups and don't bring any benefit in this discrimination.
For this reason, reducing the source of noise due to this data-points leads to the creation of an equally informative dataset, but with a reduced source of error.

\newpage


\chapter{SHAP - Implementation and results}\label{chap:shap_results}

In this chapter we are going to give a look at how we implemented a feature explainatory model using the DeepSHAP algorithm described in chapter \ref{chap:shap}, and we compared its results with a more conventional random forest feature importance analysis.

Here we present a brief summary of all the analysis we performed using feature importance informations.

We choose to run this analysis only on Pearson correlation coefficients since they gave the best classification results which means achieve the best separability between controls and ASD.
We used the whole dataset consisting of ABIDE I + ABIDE II patients with the same characteristics as we choose for classification: only males and with an age range of 5-40.
In addition we put in appendix some results obtained on the dataset ABIDE I only open eye.

\textcolor{red}{Metto una lista numerate? Devo essere più preciso in questa overview?}
In this section we are looking for what are the most important features that guide the prediction made by a model.
These features will be representative of some altered functional connections between brain areas.
From the study of these altered connection we should be able to identify biological areas relevant for the distinction between an healty and with ASD subjects.

%We are creating a gerarchy of features from the most to the less important in order to analyse if
What we are looking for in this section is whether there are some features whose contribution was greater than others during the distinction between controls and ASD data, and if these feature are recurrent, regardless the type of analysis we are carrying out, or the machine learning classifier we use.
%for the most important features computed on the entire test dataset and even if for each test data they contributed poorly, there is a certain gerarchy of importance between them.

As a first, itroductory analysis, in section \ref{sec:feature_importance_dnn} we start analyzing the most important feature that lead the classification of the DNNs models, using SHAP.
We extract the most important features for each of the pipeline discussed in section \ref{sec:analysis_workflow}, our goal is to check if the presence of harmonization, implemented by different strategies, significatively alter the contribution of features in the output of the model.
In section \ref{sec:feature_importance_rf} we repeat the same analysis earlier done with DNNs, with a random forest classifier. For this machine learning model, feature extraction is implemented by the use of the feature\_extraction methods provided by sklearn.
In these analysis our goal is to understand if a different machine learning algorithm gives results relying on different features than a deep network model.
We also check the impact of harmonization just as we did with the deep models
%We start asking what are the most important feature for each of the analysis we discussed in section \ref{sec:analysis_workflow} both with the DNNs and with the Random Forest classifier.
These, however are just a first, preliminary inspections and for this reason we limit our focus to the first 20 important feature for each pipeline.
We limit choose this number because we can easily visualize them and have a visual comparison to find differences between the diffent analysis.

A more meaningful analysis, is executed afterwards.
We select just 5 pipelines among all the one considered earlier, because we regard at these 5 as the most significative analysis.
We will explain more in details in section \ref{sec:feature_importance_cross} what these analysis are and why we excluded the others and we will present results on common important features between these 5 analysis.
From them we examine the biological meaning of these feature and the brain areas they involve, to create an histogram of the most involved areas in discrimination of healthy and ASD subjects.
%if these most important features are the same across the different pipelines we followed or if each pipeline  more specifically we want to check if before and after the harmonization procedure, the most important features still remain the same or if they change depending on the kind of analysis we are carrying out.
%We compared important features extracted with SHAP (related to the DNNs) with those important for a Random Forest classifier to evaluate if there are feature that stand out regardless the algorithm used to compute them since we used SHAP to explain feature from the DNN and the feature\_importance class provided by scikit learn to assess feature in a random forest classifier.

\hfill

To instantiate the desired class with SHAP: DeepExplainer in our case, we require the trained deep model and a fraction of the training data to use as background.
We can choose as many background data as we want keeping in mind that the bigger is this background subset $N_{bkg}$, the more accurate is the computing of shapley values, but the more computationally expansive this process will be.
In particular using a big amount of background data, would occupy a vast amount of RAM memory and eventually run out of it;
However, since the error we make on shap values linked to this choice is $\sim 1/\sqrt{N_{bkg}}$, a background dataset made of 100 samples is already a good compomise to have a relatively small error.
We choose though, a background size of 500 samples for the entire dataset and a size of 100 samples when working on ABIDE I only open eyes, since we had a reduced number of train data to collect background data from.

Once the explainer is created on our model, we need to input as many test dataset as we want to explain, and for each one of them, the explainer will output an array containing a shap coefficient for all the n\_features of each test data.
Once inputed a batch of test data, we obtain a matrix of the same size of the test dataset we used, shaped n\_samples x n\_features, containing all the shapley values for each feature of each test data.
We choose to input all the test dataset for each fold to obtain a more statistically accurate estimate of feature importance.

Since we are presenting our results following a k-fold cross validation scheme, we implemented this procedure inside the k-fold CV.
To do so and collect a final and general result, we computed the shap values for all the test samples of each k-fold and we stacked them together.
At the end of the CV, we have a matrix of shap values, and we can proceed to visualize the results.

\paragraph{Different types of SHAP plots} \hfill

We can visualize, for each test data, the contribution of each feature on pushing or pulling the predicted output from the baseline output of our model, but to produce this kind of plot we need to save the test datasets as well.
An example is shown in figure \ref{fig:shap_waterfall}.
This type of plot, called \emph{waterfall plot}, shows the result of each feature on a single test data.
We have the baseline value of the model in the lower right corner and on the top left the output of the model with this test data.
For each feature, positive shapley values, representing positive contribution to our output are colored red and negative ones are blue.
Even if this is just an example taken from a single test data, it representative of all the analysis where there is not a feature whose contribution is way greater than the others.
In fact each feature makes a small contribution to the final outcome, but the sum of them push the model towards an output.


\begin{figure}[h]
\centering
\includegraphics[width=0.6\textwidth]{shap/waterfall_plot}
\caption{Example of a waterfall plot obtained with a single test data. For each row is displayed the feature's name and value and its contribution to the final output: negative contribution are colored blue and positive ones red. It is also marked the reference output in the bottom right corner, and the output of this instance on the top left.
}\label{fig:shap_waterfall}
\end{figure}


If we want look at the overall result on all the test data, it would be inconvenient to create a plot like \ref{fig:shap_waterfall} for each data. For this reason we have to look at an other plot called \emph{force plot}

The force plot of feature's importance can involve either the module of shap values or shap values themselves (positive or negative), an example of the latter is figure \ref{fig:shap_dot_kfold}: this plot shows for each instance (for each sample from the test data)
%\footnote{https://shap.readthedocs.io/en/latest/example\_notebooks/tabular\_examples/tree\_based\_models/Scatter\%20Density\%20vs.\%20Violin\%20Plot\%20Comparison.html},
\footnote{https://shap.readthedocs.io/en/latest/}
the contribution of one particular feature on the output computed with that test data.
In other words, for each test data, we have an array of shapley values: one for each feature, and these shap values for each features are represented by a dot, placed to the right or to the left of the thin black line corresponding shapley values with a value zero (a zero value indicates that for a certain test data, that corresponding feature did not contributed to the output).
This position in respect to the black vertical line, indicates the sign of that contribution: on the right we found positive coefficients that gave a positive contribution pushing the output towards values greater than the reference output, and on the left are placed those that had a negative contribution.
Just to be clear, positive and negative adjectives are to be intended in a mathematical sense, as greater or smaller than zero, and pushing the output towards greater values (closer to 1 than to 0), means that the input data is being classified as autism.
This process of placing shap values for each feature in a plot, is repeated for each test data to obtain a scatter plot of shapley values for each feature, where features are ordered by importance.
Each spot is also colored red or blue. Color indicates the magnitude of that feature calculated in respect of the mean value of that feature across all the test dataset. With this additional information we are able to understand if a great or a small value of a feature pushes the model towards one output or the opposite (0 or 1).
As a practicle example to understand this concept of colors, with our data, if we ask \textquotedblleft is a strong or a weak correlation between two areas, that contributed the most to the prediction of my model towards high values (and then more close to a label 1 which means autism)?\textquotedblright.


% as a single colored spot, the color indicated the strength of the contribution of that feature to the final outcome, and
However, for our data, this kind of plot, is not the best suitable for readibility because of the large amount of features and the small contribution of each feature to the final outcome, in respect to other feature.
This results in a smoothed scatter plot for each feature, containing many but unnecessary details which make this kind of plot not the most clear to assess how much a feature is important for that model.

We can choose to visualize the total amount of \textquotedblleft contribution \textquotedblright of each feature, by considering the absolute value of the shap values of figure \ref{fig:shap_dot_kfold} to obtain figure \ref{fig:shap_bar_kfold} representing a bar plot of all the first 20 important feature from the most to the least important.
Each bar represents a feature importance, computed from equation \ref{eq:shap_magnitude}.

For a better readability we prefer reporting the results like this, in terms of absolute shapley value, where we just consider the module of the shap values, regardless its positive or negative contribution to the output, just to understand the absolute contribution of a single feature for a given analysis pipeline.

From this plot we have a more immediate understanding of what are the most important features, and, as we noticed from the example in figure \ref{fig:shap_waterfall}, there is not a standing out feature, or group of feature that have a contribution way greater than the others, but in fact feature's importance has a quite smoothed descending trend.
An example to visualize this trend is shown in figures \ref{fig:shap_bar_kfold} representing the first 20 most important features extracted from DNN trained on data harmonized inside k-fold CV.

\begin{figure}[h!]
\centering
\begin{subfigure}[t]{.45\linewidth}
   \fbox{\includegraphics[width=1\linewidth]{shap/ab_all/shap_dot_kfold}}
   \caption{Plot of dotted shapley values: for each feature each instance (each test sample) is represented by a dot, the position indicates a positive or negative contribution to the output, while the color represents the magnitude of the feature, compared with the average value of that feature across all the test dataset.}
   \label{fig:shap_dot_kfold}
\end{subfigure}
\hspace{3mm}
\begin{subfigure}[t]{.439\linewidth}
  \fbox{ \includegraphics[width=1\linewidth]{shap/ab_all/shap_bar_kfold}}
   \caption{Bar plot of shapley values: for each feature is represented the mean absolute value of all the shap values computed for each instance}
   \label{fig:shap_bar_kfold}
\end{subfigure}
\caption{Two different but closely related force plots of shap values regarding the first 20 most important features obtained from data where harmonization procedure is implemented inside the k-fold CV scheme
}
\label{fig:shap_features_kfold_both}
\end{figure}




\newpage
\section{Plots of the most relevant features from DNNs and random forest}
\subsection*{Feature importance with DNN}\label{sec:feature_importance_dnn} \hfill

If we plot the first twenty important feature for DNNs, obtained from each pipeline described in section \ref{sec:analysis_workflow}, we can notice that some features seem to be persistent through all the four pipelines; we find that: feature 592 and 1521 appear, even if in different order of importance, in all our four analysis;
(We replicated the plot in figure \ref{fig:shap_bar_kfold} putting it again in figure \ref{fig:shap_bar_kfold_} for a more immediate visual comparison of all the twenty important features for each analysis).
If we consider only features common between harmonization in k-fold, harmonization upstream and raw data, two more features appear common:  1513 and 3334.
We run this latter analysis excluding the adversarial network to allow comparison with results of the following section \ref{sec:feature_importance_rf} extracted from random forest.

Since each feature is representative of a correlation between two brain areas, we list what are these ares involed:
\begin{itemize}
\item Feature 592: correlation between Right Middle Temporal Gyrus ( anterior division) and Left Superior Temporal Gyrus ( anterior division)
\item Feature 1521: correlation between Left Angular Gyrus  and Right Middle Temporal Gyrus ( posterior division)
\item Feature 1513: correlation between Left Angular Gyrus  and Right Temporal Pole
\item Feature 3334: correlation between Right Parahippocampal Gyrus ( posterior division) and Right Accumbens

\end{itemize}


\begin{figure}[h!]
\centering
\begin{subfigure}[c]{.45\linewidth}
  \includegraphics[width=1\linewidth]{shap/ab_all/shap_bar_kfold}
   \caption{Bar plot of shap values and related important features extracted with harmonized data, with harmonization procedure implemented inside \textbf{k-fold} CV}
   \label{fig:shap_bar_kfold_}
\end{subfigure}
\begin{subfigure}[c]{.45\textwidth}
   \includegraphics[width=1\linewidth]{shap/ab_all/shap_bar_upstream}
   \caption{Bar plot of shap values and related important features extracted with harmonized data, with harmonization procedure implemented \textbf{upstream}, before the k-fold}
   \label{}
\end{subfigure}
\hspace{3mm}
\begin{subfigure}[c]{.45\textwidth}
   \includegraphics[width=1\linewidth]{shap/ab_all/shap_bar_no}
   \caption{Bar plot of shap values and related important features extracted from \textbf{raw}, not-harmonized data}
   \label{}
\end{subfigure}
\begin{subfigure}[c]{.45\textwidth}
   \includegraphics[width=1\linewidth]{shap/ab_all/shap_bar_adv}
   \caption{Bar plot of shap values and related important features extracted from raw data using the \textbf{Adversarial} neural network,}
   \label{}
\end{subfigure}
\caption{First twenty most important features plotted with a bar plot of mean absolute shap values. Extracted from each of the four analysis we run with DNNs and Adversarial network}
\label{fig:shap_abide_all}
\end{figure}





\subsection*{Feature importance with Random Forest}\label{sec:feature_importance_rf}
%they are computed as the mean and standard deviation of accumulation of the impurity decrease within each tree \footnote{https://scikit-learn.org/stable/auto_examples/ensemble/plot_forest_importances.html}
In this section we repeat the same plot we obtained with a dnn in figure \ref{fig:shap_abide_all}, with a different machine learning algorithm: a random forest classifier.
We plot the first 20 most important features obtained from the three main analysis of harmonization in k-fold, harmonization upstream, and raw data.
Features are extracted an plots are created using the function provided by Scikit-Learn library.
Results are shown in figure \ref{fig:shap_abide_rf_all}. From the three plots we notice that 7 different features are common to the three analysis: feature 710, 1127, 1703, 748, 2735, 1400 and 2811.
As we did for DNNs we create a list with information about the brain areas involved in these correlations.


\begin{figure}
\centering
\begin{subfigure}[b]{.45\textwidth}
   \includegraphics[width=1\linewidth]{shap/ab_all/rf_feature_importance_kfold}
   \caption{kfold}
   \label{}
\end{subfigure}
\begin{subfigure}[b]{.45\textwidth}
   \includegraphics[width=1\linewidth]{shap/ab_all/rf_feature_importance_upstream}
   \caption{upstream}
   \label{}
\end{subfigure}
\begin{subfigure}[b]{.45\textwidth}
   \includegraphics[width=1\linewidth]{shap/ab_all/rf_feature_importance_no}
   \caption{Non-harmonized}
   \label{}
\end{subfigure}
\caption{Feature importance obtained with a Random Forest classifier: results on the entire dataset.}
\label{fig:shap_abide_rf_all}
\end{figure}



\begin{itemize}
\item Feature 710: correlation between Right Middle Temporal Gyrus ( temporooccipital part) and Right Thalamus
\item Feature 1127: correlation between Left Postcentral Gyrus  and Right Postcentral Gyrus
\item Feature 1703: correlation between Right Lateral Occipital Cortex ( inferior division) and Right Supramarginal Gyrus ( anterior division)
\item Feature 748: correlation between Left Middle Temporal Gyrus ( temporooccipital part) and Right Thalamus
\item Feature 2735: correlation between Right Precuneous Cortex  and Right Middle Temporal Gyrus ( anterior division)
\item Feature 1400: correlation between Left Supramarginal Gyrus ( posterior division) and Right Inferior Frontal Gyrus ( pars triangularis)
\item Feature 2811: correlation between Left Precuneous Cortex  and Right Middle Temporal Gyrus ( posterior division)
\end{itemize}


\subsection*{Comments on feature importance}
Comparing results obtained using different classifier: a DNN and a random forest classifier, we notice from plots \ref{fig:shap_abide_all} and \ref{fig:shap_abide_rf_all} that there are no common features between the first 20. This is most likely due to the difference in the inner algorithm that guide throughout the decisional procecss. A DNN and a random forest have very different modes of operation. This can lead the two of them to find different patterns within data whose results do not (or just partially) overlap.
We are prone to consider the DNNs most reliable than the random forest because of the higher classification perfocmance achieved with that models, but as we can notice from plots, random forest feature importance is able to determine some feature that stand out over the others.
The same trend is not visible with feature extracted from the DNNs except for the pipeline of harmonization inside k-fold where it seems that some feature have a greater contribution than others and the descending in feature importance is less smooth.
A possible reason for this trend, in accordance of what we said during the discussion of classification results, is that the harmonization procedure, even if does not bring up new information, helps in data cleaning, and remove some noise due to site-distinctive patterns.
Harmonization thus helps to obtain a better and a more defined assessment of what the most important features are, in discrimination of controls/ASD.

Even if there are not overlapping features between analysis with DNN and anslysis with random forest, if we take a look at the lsits of brain areas involved in correlation, we can notice that certain brain areas such as the temporal gyri are in common between these two models.
This can suggest that even if the feature itself does not determine in the same way the output ot these two models, maybe some brain area represented by features, are recurrent in this kind of discrimination.
In this regard, the following analysis aims to study this aspect and determine what are the most important brain areas that allow a discrimination between healthy and ASD patients.

\newpage
\section{Cross analysis of important features}\label{sec:feature_importance_5}

So far, we discussed results linked to images just as a rapid and visual comparison of feature importance between DNN and random forest.
Now we go deeper to fulfill a more meaningful analysis of what are important feature and what they represent.

Since as explained in section \ref{sec:classification_discussion} we decided to exclude from our analysis the procedure of upstream harmonization because it creates a bias in results, from now on, we focus only on the other pipelines and when we refers to harmonized data, we are referring to data with harmonization procedure implemented inside kfold.

To assess what are the most relevant recurrent feature between these different types of analysis, we choose to limit our analysis to the 1\% important features among 5995; to this end we collected the first 60 important features and we limited our analysis to only 5 classification procedures listed below.
We limit to these pipelines since they are the most correct and significant, and we can extract non-biased results.


\begin{enumerate}
\item Classification of harmonized data using the DNN \label{proc:dnn_kfold}
\item Classification of harmonized data using the Random Forest classifier \label{proc:rf_kfold}
\item Classification of raw data using the DNN \label{proc:dnn_no}
\item Classification of raw data using the Random Forest classifier \label{proc:rf_no}
\item Classification using raw data with the adversarial neural network \label{proc:adv}
\end{enumerate}



Firstly we checked if and how many common features we find among the first 60 important features between these five classification.
But turns out that there are no common features between all these 5 pipelines taking into account only the first 60.
%If we exclude, though, from our analysis the adversarial learning classification we find that features 1513, 3058, 1067 are common between the remaining classification methods.
%They represent:
%\begin{itemize}
%\item Feature 1513: correlation between Left Angular Gyrus  and Right Temporal Pole
%\item Feature 3058: correlation between Right Frontal Orbital Cortex  and Left Angular Gyrus
%\item Feature 1067: correlation between Right Postcentral Gyrus  and Right Superior Temporal Gyrus (posterior division)
%\end{itemize}

Now, we carry out analysis on subgroups of the previous classification methods and we report the number of common important features among the first 60 most relevant as a number and as percentage.
This kind of analysis is useful to assess whether the harmonization procedure significatively changes important features.
It is also aimed to quantify the common features between a DNN and a random forest classifier.

\begin{itemize}
\item Focusing just on \textbf{DNN} we compared pipelines \ref{proc:dnn_kfold}. and \ref{proc:dnn_no}. finding that 22 features (37 \%) out of 60 are common between harmonized and raw data.

\item Focusing just on \textbf{Random Forest} classifier, we compared pipelines \ref{proc:rf_kfold}. and \ref{proc:rf_no}. finding that 33 features out of 60 are common (55\%) between harmonized and raw data.

\item Comparing results on  \textbf{harmonized data} obtained with \textbf{DNN} (pipeline \ref{proc:rf_kfold}.) and \textbf{Random Forest} (\ref{proc:dnn_kfold}.) we obtain 13 common features (22 \%)

\item Comparing results on \textbf{raw data} obtained with \textbf{DNN} (pipeline \ref{proc:dnn_no}.) and \textbf{Random Forest} (\ref{proc:rf_no}.) we obtain 7 common features  (12\%)

\item Comparing results of the \textbf{Adversarial network} (\ref{proc:adv}.) with the \textbf{harmonized data} with DNN (\ref{proc:dnn_kfold}.) we obtain 21 common features (35\%)

\item Comparing results of the \textbf{Adversarial network} (\ref{proc:adv}.) with DNN classification of \textbf{raw data} (\ref{proc:dnn_no}.) we obtain 28 common features (47\%)
\end{itemize}



\section{Brain areas related to important features} \hfill

As mentioned before, each feature is representative of a correlation between two brain areas, and, since for brain parcellization we used the Harvard-Oxford atlas with 110 ROIs, each brain area is involved in 109 features.
In this section we answer to the question we asked before, whether or not some brain areas are recurrent in healthy/ASD discrimination, and what are these areas.

For this purpose, we plotted an histogram to represent the number of occurrences for each brain areas among the first 60 features for each classification procedure, and subsequently we created the histogram of the overall important brain areas pooling all of these results into a single histogram .

Figure \ref{fig:histograms_60} shows for each pipeline, what are the brain areas recurrent among the 60 most important feature.
In figure \ref{fig:important_areas_features_all} is shown the overall histogram comprising all the results.


\begin{figure}
\centering
\begin{subfigure}[b]{1\columnwidth}
   \includegraphics[width = 1\columnwidth, height = 6cm]{shap/ab_all/important_areas_features_no}
   \caption{Important areas from Raw data classified with DNN}
   \label{}
\end{subfigure}

\begin{subfigure}[b]{1\columnwidth}
   \includegraphics[width = 1\columnwidth, height = 6cm]{shap/ab_all/important_areas_features_adv}
   \caption{Important areas from Raw data classified with Adversarial network}
   \label{}
\end{subfigure}

\begin{subfigure}[b]{1\columnwidth}
   \includegraphics[width = 1\columnwidth, height = 6cm]{shap/ab_all/important_areas_features_kfold}
   \caption{Important areas from harmonized data classified with DNN}
   \label{}
\end{subfigure}
\end{figure}
\begin{figure}\ContinuedFloat
%\medskip
\begin{subfigure}[b]{1\columnwidth}
   \includegraphics[width = 1\columnwidth, height = 6cm]{shap/ab_all/important_areas_features_rf_kfold}
   \caption{Important areas from harmonized data classified with Random Forest}
   \label{}
\end{subfigure}

\begin{subfigure}[b]{1\columnwidth}
   \includegraphics[width = 1\columnwidth, height = 6cm]{shap/ab_all/important_areas_features_rf_no}
   \caption{Important areas from Raw data classified with Random Forest}
   \label{}
\end{subfigure}
\caption{Histograms of important brain areas extracted from the first 60 important features of different classification procedures}
\label{fig:histograms_60}
\end{figure}


Since it is immediate to notice that there are recurrent brain areas across all the analysis, we are willing to visualize (fig \ref{fig:important_areas_features_all}) the overall histogram putting together results obtained from each analysis.
To this end, as mentioned above, we computed in figure \ref{fig:important_areas_features_all} the overall histogram of the most recurrent brain areas between these common feature.

\begin{figure}[h]
\centering
  \includegraphics[angle = 270, origin = c, width=0.75\linewidth]{shap/ab_all/important_areas_features_all}
\caption{Histogram of the most important brain areas putting together results obtained from all the five analysis listed above.}
\label{fig:important_areas_features_all}
\end{figure}

\section{Discussion on common features and important brain areas}
\textcolor{red}{Manca discussione su zone collegate ad autismo}
\paragraph{Common important features}
With this analysis to check the common features between different analysis, we can quantify the amount of common features to have a better comparison on how much different harmonization strategies or classificator have an impact on the assessment of feature importance.
Looking at the results listed in section \ref{sec:feature_importance_5} we can assert that a DNN and a random forest definitely have different way to use features to make prediction and to assess their importance.
We find that just a low percentage of features are common between a DNN and a random forest both with raw and harmonized data.
This has a non-negligible impact when choosing the best classificator for any classification task.
We should keep in mind that there is not the best classifier in an absolute sense, but each classifier is able to find different patternt among data and we should pick the best only in relation to a specific dataset.

We can also state that harmonization procedure changes, expecially in a DNN the first important features, as we pointed out when discussing the plots of feature importance, harmonization is able to remove noise from data and reveals in a cleaner way what feature are really important for the network, without the contribution of site-related noise.
It is possible that the number of feature in a random forest between harmonized and non harmonized data is bigger because random forest is a simpler algorithm.
In fact a DNN is able to learn more complicated pattern, and in doing so, it is more sensible to noise and to the discovery of more subtile pattern that can drive to this difference in feature ranking.
This finer searching for patterns, though, is what lead a DNN to achieve better classification performances with respect to a random forest.
Thus, harmonization changes the most important features, but we believe that this chance happen in a positive way, since feature are less affected by noise and results are more reliable.

This change resulting from a better definition of feature due to harmonization, also appears when comparing results with the adversarial network.
In fact we observe that
The number of common feature between adversarial network and raw data is similar to the number between harmonized and raw data determined using the DNN.
The resaon for this lies in the mechanism of the adversarial network: it results in a reduction of some site-related noise thanks to the adversarial branch, but at the same time, the introduction of some confusion due to the flawed definition of the correct site.

%We obtain indeed a better similarity between raw data and adversarial network tha harmonized data and adversarial network.
Even if between raw data and adversarial network we should have obtained a result similar to ras data and harmonized data ($\approx 37\%$), we obtain a greater similarity of feature importance ($\approx 47\%$).
This means that data are modified and hamronized according to some inner processes, but in a really different way than the analytical harmonization.

\paragraph{Important areas} \hfill

\newpage

\addcontentsline{toc}{chapter}{APPENDIX}

\appendix


\chapter{Other classification results}

\begin{table}[!htp]\centering
\scriptsize
\begin{tabular}{lrrrrr}\toprule
Dataset &Harmon K-fold &Harmon. upstream &No harmon. &Adversarial \\\midrule
AB I$\pm$II &60$\pm$1 &71$\pm$3 &63$\pm$2 &60$\pm$1 \\
AB I &61$\pm$2 &69$\pm$3 &62$\pm$2 &61$\pm$2 \\
AB 2 &56$\pm$3 &69$\pm$4 &60$\pm$4 &59$\pm$4 \\
AB I $\pm$ II - eye open &63$\pm$2 &68$\pm$3 &64$\pm$2 &63$\pm$3 \\
AB I open eye &63$\pm$5 &74$\pm$3 &71$\pm$4 &72$\pm$2 \\
AB II eye open &63$\pm$5 &70$\pm$3 &63$\pm$3 &62$\pm$3 \\
AB I $\pm$ II AGE=all, eye=all &65$\pm$3 &71$\pm$2 &66$\pm$2 &63$\pm$2 \\
\bottomrule
\end{tabular}
\caption{Classification scores using only out-of-phase wavelet coefficients}
\label{tab:classification_wout}
\end{table}


\begin{table}[!htp]\centering
\scriptsize
\begin{tabular}{lrrrrr}\toprule
Dataset &Harmon K-fold &Harmon. upstream &No harmon. &Adversarial \\\midrule
AB I$\pm$II &67$\pm$ 3 &70$\pm$ 3 &66 $\pm$3 &66 $\pm$3 \\
AB I &70$\pm$ 3 &72 $\pm$2 &68 $\pm$3 &68$\pm$ 3 \\
AB 2 &61$\pm$ 4 &65$\pm$ 4 &64 $\pm$5 &65 $\pm$4 \\
AB I $\pm$ II - eye open &70$\pm$ 1 &73$\pm$ 2 &71 $\pm$1 &68 $\pm$2 \\
AB I open eye &71$\pm$ 4 &74 $\pm$3 &71 $\pm$3 &68 $\pm$2 \\
AB II eye open &62 $\pm$4 &66 $\pm$3 &68 $\pm$3 &64$\pm$ 5 \\
\bottomrule
\end{tabular}
\caption{Classifiaction scores using (W\_in - W\_out) wavelet coefficients coefficients}
\label{tab:classification_win-wout}
\end{table}





\begin{table}[!htp]\centering
\begin{tabular}{lccc}
\toprule
Structure &k-fold &upstream &no harmonization \\
\midrule
3-2-1 &60$\pm$2 &63$\pm$3 &60$\pm$4 \\
8-8-1 &64$\pm$1 &65$\pm$4 &64$\pm$5 \\
64-8-1 &69$\pm$1 &72$\pm$2 &68$\pm$1 \\
64-32-8-1 &68$\pm$1 &72$\pm$2 &69$\pm$1 \\
128-8-1 &70$\pm$2 &71$\pm$3 &69$\pm$2 \\
128-64-1 &70$\pm$1 &72$\pm$3 &69$\pm$2 \\
%\cellcolor[HTML]{ffff00}264-8-1 &\cellcolor[HTML]{ffff00}70$\pm$2 &\cellcolor[HTML]{ffff00}73$\pm$3 &\cellcolor[HTML]{ffff00}70$\pm$2 \\
\textcolor{ForestGreen}{264-8-1}&\textcolor{ForestGreen}{70$\pm$2} & \textcolor{ForestGreen}{73$\pm$3} & \textcolor{ForestGreen}{70$\pm$2} \\
512-8-1 &71$\pm$2 &73$\pm$2 &70$\pm$2 \\
1024-8-1 &70$\pm$1 &74$\pm$2 &71$\pm$2 \\
1024-32-1 &70$\pm$2 &74$\pm$2 &71$\pm$3 \\
\bottomrule
\end{tabular}
\caption{Different model structures and related performances for the three main analysis we carried out during this entire work. Colored the structure we choose to employ through all our analysis}\label{tab:different_model_structures}
\end{table}

\section{Feature importance results on ABIDE I open eye dataset}

The same tests we did to assess feature importance on ABIDE I + II dataset, we carried out on ABIDE I with only patient with open eye.
Here in figure \ref{fig:shap_ab1} we report the results obtained from shap, related to the first twenty most relevant features.
We can notice at a first glance that there are less feature common to all the four analysis:
There are no common features among them that we can find across all these four analysis, if we limit our study on the first twenty.
We have to search among the first thirty features to find something in common, which is still a good procedure since we are dealing with 5995 features and the first 30 are just the 0.5\% of them.

We find that, among the first 30 features, only feature  762 is common to the four analysis, while if we exclude the adversarial network we add feature 1417 and 748




\begin{itemize}
\item Feature 762: correlation between Left Middle Temporal Gyrus ( temporooccipital part) and Left Middle Frontal Gyrus
\item Feature 1417: correlation between Left Supramarginal Gyrus ( posterior division) and Left Middle Temporal Gyrus ( temporooccipital part)
\item Feature 748: correlation between Left Middle Temporal Gyrus ( temporooccipital part) and Right Thalamus (nan)
\end{itemize}



\begin{figure}[h!]
\centering
\begin{subfigure}[c]{.45\linewidth}
  \includegraphics[width=1\linewidth]{shap/ab1_openeye/shap_bar_kfold}
   \caption{Bar plot of shap values and related important features extracted with harmonized data, with harmonization procedure implemented inside \textbf{k-fold} CV}
   \label{fig:shap_bar_kfold_}
\end{subfigure}
\begin{subfigure}[c]{.45\textwidth}
   \includegraphics[width=1\linewidth]{shap/ab1_openeye/shap_bar_upstream}
   \caption{Bar plot of shap values and related important features extracted with harmonized data, with harmonization procedure implemented \textbf{upstream}, before the k-fold}
   \label{}
\end{subfigure}
\hspace{3mm}
\begin{subfigure}[c]{.45\textwidth}
   \includegraphics[width=1\linewidth]{shap/ab1_openeye/shap_bar_no}
   \caption{Bar plot of shap values and related important features extracted from \textbf{raw}, not-harmonized data}
   \label{}
\end{subfigure}
\begin{subfigure}[c]{.45\textwidth}
   \includegraphics[width=1\linewidth]{shap/ab1_openeye/shap_bar_adv}
   \caption{Bar plot of shap values and related important features extracted from raw data using the \textbf{Adversarial} neural network,}
   \label{}
\end{subfigure}
\caption{Results obtained on ABIDE I only open eye dataset. First twenty most important features plotted with a bar plot of mean absolute shap values. Extracted from each of the four analysis we run with DNNs and Adversarial network. }
\label{fig:shap_abide_all}
\end{figure}





With a random forest we obtain more coherence between important features: searching among the first 20 features, we find that
are common to the 3 pipelines: 2690, 1127, 2313, 2314, 748, 2712, 2735, 2582, 1400



\begin{itemize}
\item Feature 2690: correlation between Left Cingulate Gyrus ( posterior division) and Right Frontal Medial Cortex (nan)
\item Feature 1127: correlation between Left Postcentral Gyrus (nan) and Right Postcentral Gyrus (nan)
\item Feature 2313: correlation between Right Paracingulate Gyrus (nan) and Left Middle Temporal Gyrus ( anterior division)
\item Feature 2314: correlation between Right Paracingulate Gyrus (nan) and Right Middle Temporal Gyrus ( posterior division)
\item Feature 748: correlation between Left Middle Temporal Gyrus ( temporooccipital part) and Right Thalamus (nan)
\item Feature 2712: correlation between Right Precuneous Cortex (nan) and Right Hippocampus (nan)
\item Feature 2735: correlation between Right Precuneous Cortex (nan) and Right Middle Temporal Gyrus ( anterior division)
\item Feature 2582: correlation between Right Cingulate Gyrus ( posterior division) and Right Precentral Gyrus (nan)
\item Feature 1400: correlation between Left Supramarginal Gyrus ( posterior division) and Right Inferior Frontal Gyrus ( pars triangularis)
\end{itemize}


\begin{figure}
\centering
\begin{subfigure}[b]{.45\textwidth}
   \includegraphics[width=1\linewidth]{shap/ab1_openeye/rf_feature_importance_no}
   \caption{Non-harmonized}
   \label{}
\end{subfigure}
\begin{subfigure}[b]{.45\textwidth}
   \includegraphics[width=1\linewidth]{shap/ab1_openeye/rf_feature_importance_upstream}
   \caption{upstream}
   \label{}
\end{subfigure}
\begin{subfigure}[b]{.45\textwidth}
   \includegraphics[width=1\linewidth]{shap/ab1_openeye/rf_feature_importance_kfold}
   \caption{kfold}
   \label{}
\end{subfigure}
\caption{Feature importance obtained from a Random Forest classifier: results on ABIDE I only open eyes.}
\label{fig:rf_feature_importance_ab1}
\end{figure}



The same cross analysis we did on the entire dataset:

\begin{itemize}
\item Focusing on random forest classifier, we compared important features between \ref{proc:rf_kfold} and \ref{proc:rf_no} and found that 34 features out of 60 are common (57\%).
\item Focusing on DNN we compared pipelines \ref{proc:dnn_kfold} and \ref{proc:dnn_no} finding that 19 features (32 \%) out of 60 are common.
\item Comparing the same analysis pipeline with DNN and Random Forest we find that with harmonized data: procedure \ref{proc:rf_kfold} and \ref{proc:dnn_kfold} there are 11 common features (18 \%)
\item Comparing results on raw data obtained with DNN \ref{proc:dnn_no} and Random Forest \ref{proc:rf_no} we obtain 13 common features  (22\%)
\item Comparing the adversarial results \ref{proc:adv} with the harmonized data with DNN \ref{proc:dnn_kfold} we obtain 19 common features (32\%)
\item Comparing the adversarial results \ref{proc:adv} with DNN classification of raw data \ref{proc:dnn_no} we obtain 23 common features (38\%)
\end{itemize}


\begin{figure}
\centering
\begin{subfigure}[b]{1\columnwidth}
   \includegraphics[width = 1\columnwidth, height = 6cm]{shap/ab1_openeye/important_areas_features_no}
   \caption{Important areas from Raw data classified with DNN}
   \label{}
\end{subfigure}

\begin{subfigure}[b]{1\columnwidth}
   \includegraphics[width = 1\columnwidth, height = 6cm]{shap/ab1_openeye/important_areas_features_adv}
   \caption{Important areas from Raw data classified with Adversarial network}
   \label{}
\end{subfigure}

\begin{subfigure}[b]{1\columnwidth}
   \includegraphics[width = 1\columnwidth, height = 6cm]{shap/ab1_openeye/important_areas_features_kfold}
   \caption{Important areas from harmonized data classified with DNN}
   \label{}
\end{subfigure}
\end{figure}
\begin{figure}\ContinuedFloat
%\medskip
\begin{subfigure}[b]{1\columnwidth}
   \includegraphics[width = 1\columnwidth, height = 6cm]{shap/ab1_openeye/important_areas_features_rf_kfold}
   \caption{Important areas from harmonized data classified with Random Forest}
   \label{}
\end{subfigure}

\begin{subfigure}[b]{1\columnwidth}
   \includegraphics[width = 1\columnwidth, height = 6cm]{shap/ab1_openeye/important_areas_features_rf_no}
   \caption{Important areas from Raw data classified with Random Forest}
   \label{}
\end{subfigure}
\caption{Histograms of important brain areas extracted from the first 60 important features of different classification procedures}
\label{fig:histograms_60_ab1_openeye}
\end{figure}


\begin{figure}[h]
\centering
  \includegraphics[angle = 270, origin = c, width=0.75\linewidth]{shap/ab1_openeye/important_areas_features_all}
\caption{Histogram of the most important brain areas putting together results obtained from all the five analysis listed above.}
\label{fig:important_areas_features_all_ab1_openeye}
\end{figure}

\newpage

\bibliographystyle{plain}
\bibliography{library.bib}



\end{document}
